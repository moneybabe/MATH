\documentclass{article}
\newcommand*{\Var}{\ensuremath{\mathrm{Var}}}
\newcommand*{\Cov}{\ensuremath{\mathrm{Cov}}}
\newcommand*{\Corr}{\ensuremath{\mathrm{Corr}}}
\newcommand*{\Bias}{\ensuremath{\mathrm{Bias}}}
\newcommand*{\MSE}{\ensuremath{\mathrm{MSE}}}

\newcommand*{\range}{\ensuremath{\mathrm{range}}\,}
\newcommand*{\spann}{\ensuremath{\mathrm{span}}\,}
\newcommand*{\nul}{\ensuremath{\mathrm{null}}\,}
\newcommand*{\dom}{\ensuremath{\mathrm{dom}}\,}

\newcommand*{\pdv}[3][]{\frac{\partial^{#1}}{\partial#3^{#1}}#2}
\renewcommand*{\implies}{\ensuremath{\Longrightarrow}}
\renewcommand*{\impliedby}{\ensuremath{\Longleftarrow}}
\renewcommand*{\l}{\left}
\renewcommand*{\r}{\right}
\renewcommand{\leq}{\leqslant}
\renewcommand{\geq}{\geqslant}
\newcommand*{\tb}{\textbf}
\renewcommand*{\t}{\text}

\newcommand*{\Z}{\ensuremath{\mathbb{Z}}}
\newcommand*{\Q}{\ensuremath{\mathbb{Q}}}
\newcommand*{\R}{\ensuremath{\mathbb{R}}}
\newcommand*{\F}{\ensuremath{\mathbb{F}}}
\newcommand*{\C}{\ensuremath{\mathbb{C}}}
\newcommand*{\N}{\ensuremath{\mathbb{N}}}
\newcommand*{\E}{\ensuremath{\mathds{E}}}
\renewcommand*{\P}{\ensuremath{\mathds{P}}}
\newcommand*{\p}{\ensuremath{\mathcal{P}}}
\renewcommand*{\L}{\mathcal{L}}


% deliminators
\DeclarePairedDelimiter{\abs}{\lvert}{\rvert}
\DeclarePairedDelimiter{\norm}{\|}{\|}
\DeclarePairedDelimiter{\inner}{\langle}{\rangle}
\DeclarePairedDelimiter{\ceil}{\lceil}{\rceil}
\DeclarePairedDelimiter{\floor}{\lfloor}{\rfloor}
\DeclarePairedDelimiter{\round}{\lfloor}{\rceil}

\let\oldleq\leq % save them in case they're every wanted
\let\oldgeq\geq

\newcommand*{\img}[3][]{
    \begin{figure}[htb!]
         \centering
         \includegraphics[scale=#1]{#2}
         \caption{#3}
    \end{figure}
}
\newcommand*{\imgs}[5][]{
    \begin{figure}[htb]
        \qquad
        \begin{minipage}{.4\textwidth}
            \centering
            \includegraphics[scale=#1]{#2}
            \caption{#3}
        \end{minipage}    
        \qquad
        \begin{minipage}{.4\textwidth}
            \centering
            \includegraphics[scale=#1]{#4}
            \caption{#5}
        \end{minipage}
    \end{figure} 
}


\title{Math 104 HW12}
\author{Neo Lee}
\date{12/01/2023}

\begin{document} 

\maketitle 

\subsection*{Exercise 31.2}
Find the Taylor series for the functions $\sinh =\frac{1}{2}\l(e^x-e^{-x}\r)$ 
and $\cosh =\frac{1}{2}\l(e^x+e^{-x}\r)$ and indicate why they converge to 
$\sinh$ and $\cosh$ respectively for all $x\in\R$.
\begin{proof}[Solution]
    The even degree derivatives of $\sinh$ centered at 0 are 
    all 0, and the odd degree derivatives are all 1. Thus, the Taylor series 
    for $\sinh$ is
    \[
        \sinh(x) = \sum_{n=0}^\infty \frac{x^{2n+1}}{(2n+1)!}.
    \]
    On the other hand, the even degree derivatives of $\cosh$ centered at 0
    are all 1, and the odd degree derivatives are all 0. Thus, the Taylor series
    for $\cosh$ is
    \[
        \cosh(x) = \sum_{n=0}^\infty \frac{x^{2n}}{(2n)!}.
    \]
    
    For all $x\in\R$, $x\in(-|x|-1,|x|+1)$, and both $\sinh^{(n)}$ and $\cosh^{(n)}$ 
    are bounded on this interval (by $e^{|x|+1}$) for all $n\in\N$. Thus, by \emph{Corollary 31.4},
    the remainder term of Taylor series for both $\sinh$ and $\cosh$ tend to 0 
    as $n\to\infty$ for all $x\in\R$. Thus, the Taylor series for both $\sinh$ 
    and $\cosh$ converges to $\sinh$ and $\cosh$ respectively for all $x\in\R$.
\end{proof}

\newpage
\subsection*{Exercise 31.5}
Let $g(x)=e^{-1/x^2}$ for $x\neq 0$ and $g(0)=0$.
\begin{enumerate}[label=\tb{(\alph*)}]
    \item Show $g^{(n)}(0)=0$ for all $n=0,1,2,\dots$.
    \item Show the Taylor series for $g$ about 0 agrees with $g$ only at $x=0$.
\end{enumerate}
\begin{proof}[Solution]\indent
    \begin{enumerate}[label=\tb{(\alph*)}]
        \item
        Let \(f(x)=e^{-1/x}\), then $g(x)=f(x^2)$ for all $x\in\R$. Since $f$ is differentiable at 0,
        and $x^2$ is differentiable at 0. By the Chain rule, $g$ is differentiable at 0, and
        \[
            g'(x)=f'(x^2)\cdot 2x
        \]
        for all $x=0$. Since $f^k$ is differentiable at 0 for all $k\in\Z^{\ge}$, and $x^j$ is 
        differentiable at 0 for all $j\in\N$, we can apply Chain rule and Product rule repeatedly 
        to obtain higher degree derivatives of $g$ at 0. 

        The formula for derivatives of $g$ at 0 is
        \[
            g^{(n)}(x)=\sum_{k=0}^n a_kf^{(k)}(x^2)\cdot x^{b_k}
        \]
        for some non-negative integers $a_k$ and $b_k$. 
        From \emph{example 3}, we know $f^{(n)}(0)=0$ for $n\in\Z^\ge$. Hence, $g^{(n)}(x)$ evaluated at 0 is 0 for 
        all $n\in\Z^\ge$.
        
        \item
        The Taylor series of $g$ about 0 is 
        \[
            \sum_{n=0}^\infty \frac{g^{(n)}(0)}{n!}x^n=0.
        \]
        Clearly, $g(x)=0$ for all $x\neq 0$, but $g(0)=0$. Thus, the Taylor series
        for $g$ about 0 agrees with $g$ only at $x=0$.
    \end{enumerate}    
\end{proof}

\newpage
\subsection*{Exercise 32.2}
Let $f(x)=x$ for rational $x$ and $f(x)=0$ for irrational $x$.
\begin{enumerate}[label=\tb{(\alph*)}]
    \item 
    Calculate the upper and lower Darboux integrals for $f$ on $[0,b]$.
    \item
    Is $f$ integrable on $[0,b]$?
\end{enumerate}
\begin{proof}[Solution]\indent
    \begin{enumerate}[label=\tb{(\alph*)}]
        \item For any partition $P$ of $[0,b]$, the supremum of $f$ on any subinterval
        of $P$ is the right endpoint of the subinterval, and the infimum of $f$ on any
        subinterval of $P$ is 0 because rationals and irrationals are dense in $\R$.

        Therefore, with any partition $P$ of $[0,b]$, 
        \[
          U(f,P) = \sum_{k=1}^{n}x_k\cdot (x_k-x_{k-1}).
        \]
        In fact, $f$ has the same upper Darboux integrals as $g(x)=x$ on $[0,b]$ because they have 
        the same $U(f,P)$ for any $P$. 
        From calculus, we know that $g$ is integrable on $[0,b]$ and 
        \[
            \int_0^b g(x)dx = \frac{b^2}{2}.
        \] 
        Hence, the upper Darboux integral of $g$ is $\frac{b^2}{2}$, and the upper Darboux 
        integral of $f$ is also $\frac{b^2}{2}$.

        On the other hand, for any partition $P$ of $[0,b]$,
        \[
            L(f,P) = \sum_{k=1}^{n}0\cdot (x_k-x_{k-1})=0.  
        \]
        Hence, the lower Darboux integral of $f$ is 0.

        \emph{Alternatively for $g(x)=x$,} we can show that it is integrable on $[0,b]$ 
        by evaluating its upper and lower Darboux integrals by partitioning $[0,b]$ into
         $\l[\frac{(k-1)b}{n}, \frac{kb}{n}\r]$ and letting $n\to\infty$.

        \item
        Since the upper and lower Darboux integrals of $f$ are not equal, $f$ is not integrable.
    \end{enumerate}    
\end{proof}


\newpage
\subsection*{Exercise 32.7}
Let $f$ be integrable on $[a, b]$, and suppose $g$ is a function on $[a,b]$ such that 
$g(x)=f(x)$ except for finitely many $x\in[a,b]$. Show that $g$ is integrable on $[a,b]$
and $\int_a^b g(x)dx=\int_a^b f(x)dx$.
\begin{proof}[Solution]
    Let $S$ be the set of points $x\in[a,b]$ such that $g(x)\neq f(x)$. Then for any partition $P$,
    define 
    \[
        P' = P\cup \{\cdots, t_k, x-\frac{1}{n}, x, t_{k+1},\cdots\} \qquad \t{for } x\in S.
    \]
    where the $t$'s belong to the original $P$ and $n$ is chosen such that $x-\frac{1}{n}>t_k$.
    Since $P\subseteq P'$, by \emph{Lemma 32.2}, 
    \[
        L(f,P)\le L(f,P')\le U(f,P')\le U(f,P),  
    \]
    we can just focus on the partition $P'$. 

    Now for any partition $P'$, 
    $U(f,P')$ and $U(g,P')$ only differ by the finitely many subintervals that contain points in $S$.
    In fact, we can write 
    \[
        U(g,P') = U(f,P') - \sum_{x\in S}M(f, [x-\frac{1}{n}, x])\cdot \frac{1}{n} + 
        \sum_{x\in S}M(g, [x-\frac{1}{n}, x])\cdot \frac{1}{n}
    \]
    Since $S$ is finite, the two terms on the right hand side equal 0 when $n\to \infty$, and hence 
    \[
        U(g,P') = U(f,P').
    \]
    Similarly, we can show that $L(g,P') = L(f,P')$. Thus, $g$ and $f$ have the same upper and lower
    Darboux integrals, and $g$ is integrable on $[a,b]$ with $\int_a^b g(x)dx=\int_a^b f(x)dx$.

    \emph{Note:} We have not considered the case where $M(f, [x-\frac{1}{n},x])$ or 
    $M(g, [x-\frac{1}{n},x])$ is infinite. However, since $f$ and $g$ only differ by finite terms
    on that interval, if any of the two is infinite, the other must also be infinite. In that case, 
    we don't do any manipulation within that interval, and the proof still holds. 
    The same is true for the case where $m(f, [x-\frac{1}{n},x])$ or
    $m(g, [x-\frac{1}{n},x])$ is infinite.
\end{proof}


\newpage
\subsection*{Exercise 33.4}
Give an example of a function $f$ on $[0,1]$ that is not integrable for which 
$|f|$ is integrable.
\begin{proof}[Solution]
    Let $f(x)=-1$ for rationals and $f(x)=1$ for irrationals. Then $|f(x)|=1$ for all $x\in[0,1]$.
    From calculus (or we can go through the tedious upper and lower Darboux argument), we know 
    that $|f|$ is integrable on $[0,1]$ and 
    \[
        \int_0^1 |f(x)|dx = 1.
    \]
    
    However, for any partition $P$, 
    \[
      m(f, p) = -1 \qquad \t{and} \qquad M(f, p) = 1,  
    \]
    where $p$ is any subinterval of $P$. 
    Hence, 
    \[
        L(f,P) = \sum_{k=1}^{n} -1\cdot (x_k-x_{k-1}) = -1 \qquad \t{and} \qquad 
        U(f,P) = \sum_{k=1}^{n} 1\cdot (x_k-x_{k-1}) = 1.    
    \]
    Thus, $f$ is not integrable on $[0,1]$.
\end{proof}

\end{document}
