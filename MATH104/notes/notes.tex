\documentclass{book}
%--------------------------
% This amazing template is from 
% https://github.com/SeniorMars/dotfiles/tree/master/latex_template
%--------------------------

%%%%%%%%%%%%%%%%%%%%%%%%%%%%%%%%%
% PACKAGE IMPORTS
%%%%%%%%%%%%%%%%%%%%%%%%%%%%%%%%%


\usepackage[tmargin=2cm,rmargin=1in,lmargin=1in,margin=0.85in,bmargin=2cm,footskip=.2in]{geometry}
\usepackage{amsmath,amsfonts,amsthm,amssymb,mathtools}
\usepackage[varbb]{newpxmath}
\usepackage{xfrac}
\usepackage{indentfirst}
\usepackage[T1]{fontenc}
\usepackage[makeroom]{cancel}
\usepackage{bookmark}
\usepackage{enumitem}
\usepackage{hyperref,theoremref}
\hypersetup{
	pdftitle={Assignment},
	colorlinks=true, linkcolor=doc!90,
	bookmarksnumbered=true,
	bookmarksopen=true
}
\usepackage[most,many,breakable]{tcolorbox}
\usepackage{xcolor}
\usepackage{varwidth}
\usepackage{varwidth}
\usepackage{etoolbox}
\usepackage{authblk}
\usepackage{nameref}
\usepackage{multicol,array}
\usepackage{tikz-cd}
\usepackage[ruled,vlined,linesnumbered]{algorithm2e}
\usepackage{comment} % enables the use of multi-line comments (\ifx \fi) 
\usepackage{import}
\usepackage{xifthen}
\usepackage{pdfpages}
\usepackage{transparent}
\usepackage{setspace}
\setstretch{1.15}
\fontfamily{cmr}\selectfont
\DeclareSymbolFont{letters}     {OML}{cmm} {m}{it}


\newcommand\mycommfont[1]{\footnotesize\ttfamily\textcolor{blue}{#1}}
\SetCommentSty{mycommfont}
\newcommand{\incfig}[1]{%
    \def\svgwidth{\columnwidth}
    \import{./figures/}{#1.pdf_tex}
}

\usepackage{tikzsymbols}
\renewcommand\qedsymbol{$\Laughey$}


%%%%%%%%%%%%%%%%%%%%%%%%%%%%%%
% SELF MADE COLORS
%%%%%%%%%%%%%%%%%%%%%%%%%%%%%%


\definecolor{myg}{RGB}{56, 140, 70}
\definecolor{myb}{RGB}{45, 111, 177}
\definecolor{myr}{RGB}{199, 68, 64}
\definecolor{mytheorembg}{HTML}{F2F2F9}
\definecolor{mytheoremfr}{HTML}{00007B}
\definecolor{mylemmabg}{HTML}{FFFAF8}
\definecolor{mylemmafr}{HTML}{983b0f}
\definecolor{mypropbg}{HTML}{f2fbfc}
\definecolor{mypropfr}{HTML}{191971}
\definecolor{myexamplebg}{HTML}{F2FBF8}
\definecolor{myexamplefr}{HTML}{88D6D1}
\definecolor{myexampleti}{HTML}{2A7F7F}
\definecolor{mydefinitbg}{HTML}{E5E5FF}
\definecolor{mydefinitfr}{HTML}{3F3FA3}
\definecolor{notesgreen}{RGB}{0,162,0}
\definecolor{myp}{RGB}{197, 92, 212}
\definecolor{mygr}{HTML}{2C3338}
\definecolor{myred}{RGB}{127,0,0}
\definecolor{myyellow}{RGB}{169,121,69}
\definecolor{myexercisebg}{HTML}{F2FBF8}
\definecolor{myexercisefg}{HTML}{88D6D1}


%%%%%%%%%%%%%%%%%%%%%%%%%%%%
% TCOLORBOX SETUPS
%%%%%%%%%%%%%%%%%%%%%%%%%%%%

\setlength{\parindent}{1cm}
%================================
% THEOREM BOX
%================================

\tcbuselibrary{theorems,skins,hooks}
\newtcbtheorem[number within=section]{Theorem}{Theorem}
{%
	enhanced,
	breakable,
	colback = mytheorembg,
	frame hidden,
	boxrule = 0sp,
	borderline west = {2pt}{0pt}{mytheoremfr},
	sharp corners,
	detach title,
	before upper = \tcbtitle\par\smallskip,
	coltitle = mytheoremfr,
	fonttitle = \bfseries\sffamily,
	description font = \mdseries,
	separator sign none,
	segmentation style={solid, mytheoremfr},
}
{th}

\tcbuselibrary{theorems,skins,hooks}
\newtcbtheorem[number within=section]{theorem}{Theorem}
{%
	enhanced,
	breakable,
	colback = mytheorembg,
	frame hidden,
	boxrule = 0sp,
	borderline west = {2pt}{0pt}{mytheoremfr},
	sharp corners,
	detach title,
	before upper = \tcbtitle\par\smallskip,
	coltitle = mytheoremfr,
	fonttitle = \bfseries\sffamily,
	description font = \mdseries,
	separator sign none,
	segmentation style={solid, mytheoremfr},
}
{th}


\tcbuselibrary{theorems,skins,hooks}
\newtcolorbox{Theoremcon}
{%
	enhanced
	,breakable
	,colback = mytheorembg
	,frame hidden
	,boxrule = 0sp
	,borderline west = {2pt}{0pt}{mytheoremfr}
	,sharp corners
	,description font = \mdseries
	,separator sign none
}

%================================
% Corollery
%================================
\tcbuselibrary{theorems,skins,hooks}
\newtcbtheorem[number within=section]{Corollary}{Corollary}
{%
	enhanced
	,breakable
	,colback = myp!10
	,frame hidden
	,boxrule = 0sp
	,borderline west = {2pt}{0pt}{myp!85!black}
	,sharp corners
	,detach title
	,before upper = \tcbtitle\par\smallskip
	,coltitle = myp!85!black
	,fonttitle = \bfseries\sffamily
	,description font = \mdseries
	,separator sign none
	,segmentation style={solid, myp!85!black}
}
{th}
\tcbuselibrary{theorems,skins,hooks}
\newtcbtheorem[number within=section]{corollary}{Corollary}
{%
	enhanced
	,breakable
	,colback = myp!10
	,frame hidden
	,boxrule = 0sp
	,borderline west = {2pt}{0pt}{myp!85!black}
	,sharp corners
	,detach title
	,before upper = \tcbtitle\par\smallskip
	,coltitle = myp!85!black
	,fonttitle = \bfseries\sffamily
	,description font = \mdseries
	,separator sign none
	,segmentation style={solid, myp!85!black}
}
{th}


%================================
% LEMMA
%================================

\tcbuselibrary{theorems,skins,hooks}
\newtcbtheorem[number within=section]{Lemma}{Lemma}
{%
	enhanced,
	breakable,
	colback = mylemmabg,
	frame hidden,
	boxrule = 0sp,
	borderline west = {2pt}{0pt}{mylemmafr},
	sharp corners,
	detach title,
	before upper = \tcbtitle\par\smallskip,
	coltitle = mylemmafr,
	fonttitle = \bfseries\sffamily,
	description font = \mdseries,
	separator sign none,
	segmentation style={solid, mylemmafr},
}
{th}

\tcbuselibrary{theorems,skins,hooks}
\newtcbtheorem[number within=section]{lemma}{Lemma}
{%
	enhanced,
	breakable,
	colback = mylemmabg,
	frame hidden,
	boxrule = 0sp,
	borderline west = {2pt}{0pt}{mylemmafr},
	sharp corners,
	detach title,
	before upper = \tcbtitle\par\smallskip,
	coltitle = mylemmafr,
	fonttitle = \bfseries\sffamily,
	description font = \mdseries,
	separator sign none,
	segmentation style={solid, mylemmafr},
}
{th}


%================================
% PROPOSITION
%================================

\tcbuselibrary{theorems,skins,hooks}
\newtcbtheorem[number within=section]{Prop}{Proposition}
{%
	enhanced,
	breakable,
	colback = mypropbg,
	frame hidden,
	boxrule = 0sp,
	borderline west = {2pt}{0pt}{mypropfr},
	sharp corners,
	detach title,
	before upper = \tcbtitle\par\smallskip,
	coltitle = mypropfr,
	fonttitle = \bfseries\sffamily,
	description font = \mdseries,
	separator sign none,
	segmentation style={solid, mypropfr},
}
{th}

\tcbuselibrary{theorems,skins,hooks}
\newtcbtheorem[number within=section]{prop}{Proposition}
{%
	enhanced,
	breakable,
	colback = mypropbg,
	frame hidden,
	boxrule = 0sp,
	borderline west = {2pt}{0pt}{mypropfr},
	sharp corners,
	detach title,
	before upper = \tcbtitle\par\smallskip,
	coltitle = mypropfr,
	fonttitle = \bfseries\sffamily,
	description font = \mdseries,
	separator sign none,
	segmentation style={solid, mypropfr},
}
{th}


%================================
% CLAIM
%================================

\tcbuselibrary{theorems,skins,hooks}
\newtcbtheorem[number within=section]{claim}{Claim}
{%
	enhanced
	,breakable
	,colback = myg!10
	,frame hidden
	,boxrule = 0sp
	,borderline west = {2pt}{0pt}{myg}
	,sharp corners
	,detach title
	,before upper = \tcbtitle\par\smallskip
	,coltitle = myg!85!black
	,fonttitle = \bfseries\sffamily
	,description font = \mdseries
	,separator sign none
	,segmentation style={solid, myg!85!black}
}
{th}



%================================
% Exercise
%================================

\tcbuselibrary{theorems,skins,hooks}
\newtcbtheorem[number within=section]{Exercise}{Exercise}
{%
	enhanced,
	breakable,
	colback = myexercisebg,
	frame hidden,
	boxrule = 0sp,
	borderline west = {2pt}{0pt}{myexercisefg},
	sharp corners,
	detach title,
	before upper = \tcbtitle\par\smallskip,
	coltitle = myexercisefg,
	fonttitle = \bfseries\sffamily,
	description font = \mdseries,
	separator sign none,
	segmentation style={solid, myexercisefg},
}
{th}

\tcbuselibrary{theorems,skins,hooks}
\newtcbtheorem[number within=section]{exercise}{Exercise}
{%
	enhanced,
	breakable,
	colback = myexercisebg,
	frame hidden,
	boxrule = 0sp,
	borderline west = {2pt}{0pt}{myexercisefg},
	sharp corners,
	detach title,
	before upper = \tcbtitle\par\smallskip,
	coltitle = myexercisefg,
	fonttitle = \bfseries\sffamily,
	description font = \mdseries,
	separator sign none,
	segmentation style={solid, myexercisefg},
}
{th}

%================================
% EXAMPLE BOX
%================================

\newtcbtheorem[number within=section]{Example}{Example}
{%
	colback = myexamplebg
	,breakable
	,colframe = myexamplefr
	,coltitle = myexampleti
	,boxrule = 1pt
	,sharp corners
	,detach title
	,before upper=\tcbtitle\par\smallskip
	,fonttitle = \bfseries
	,description font = \mdseries
	,separator sign none
	,description delimiters parenthesis
}
{ex}

\newtcbtheorem[number within=section]{example}{Example}
{%
	colback = myexamplebg
	,breakable
	,colframe = myexamplefr
	,coltitle = myexampleti
	,boxrule = 1pt
	,sharp corners
	,detach title
	,before upper=\tcbtitle\par\smallskip
	,fonttitle = \bfseries
	,description font = \mdseries
	,separator sign none
	,description delimiters parenthesis
}
{ex}

%================================
% DEFINITION BOX
%================================

\newtcbtheorem[number within=section]{Definition}{Definition}{enhanced,
	before skip=2mm,after skip=2mm, colback=red!5,colframe=red!80!black,boxrule=0.5mm,
	attach boxed title to top left={xshift=1cm,yshift*=1mm-\tcboxedtitleheight}, varwidth boxed title*=-3cm,
	boxed title style={frame code={
					\path[fill=tcbcolback]
					([yshift=-1mm,xshift=-1mm]frame.north west)
					arc[start angle=0,end angle=180,radius=1mm]
					([yshift=-1mm,xshift=1mm]frame.north east)
					arc[start angle=180,end angle=0,radius=1mm];
					\path[left color=tcbcolback!60!black,right color=tcbcolback!60!black,
						middle color=tcbcolback!80!black]
					([xshift=-2mm]frame.north west) -- ([xshift=2mm]frame.north east)
					[rounded corners=1mm]-- ([xshift=1mm,yshift=-1mm]frame.north east)
					-- (frame.south east) -- (frame.south west)
					-- ([xshift=-1mm,yshift=-1mm]frame.north west)
					[sharp corners]-- cycle;
				},interior engine=empty,
		},
	fonttitle=\bfseries,
	title={#2},#1}{def}
\newtcbtheorem[number within=section]{definition}{Definition}{enhanced,
	before skip=2mm,after skip=2mm, colback=red!5,colframe=red!80!black,boxrule=0.5mm,
	attach boxed title to top left={xshift=1cm,yshift*=1mm-\tcboxedtitleheight}, varwidth boxed title*=-3cm,
	boxed title style={frame code={
					\path[fill=tcbcolback]
					([yshift=-1mm,xshift=-1mm]frame.north west)
					arc[start angle=0,end angle=180,radius=1mm]
					([yshift=-1mm,xshift=1mm]frame.north east)
					arc[start angle=180,end angle=0,radius=1mm];
					\path[left color=tcbcolback!60!black,right color=tcbcolback!60!black,
						middle color=tcbcolback!80!black]
					([xshift=-2mm]frame.north west) -- ([xshift=2mm]frame.north east)
					[rounded corners=1mm]-- ([xshift=1mm,yshift=-1mm]frame.north east)
					-- (frame.south east) -- (frame.south west)
					-- ([xshift=-1mm,yshift=-1mm]frame.north west)
					[sharp corners]-- cycle;
				},interior engine=empty,
		},
	fonttitle=\bfseries,
	title={#2},#1}{def}



%================================
% Solution BOX
%================================

\makeatletter
\newtcbtheorem{question}{Question}{enhanced,
	breakable,
	colback=white,
	colframe=myb!80!black,
	attach boxed title to top left={yshift*=-\tcboxedtitleheight},
	fonttitle=\bfseries,
	title={#2},
	boxed title size=title,
	boxed title style={%
			sharp corners,
			rounded corners=northwest,
			colback=tcbcolframe,
			boxrule=0pt,
		},
	underlay boxed title={%
			\path[fill=tcbcolframe] (title.south west)--(title.south east)
			to[out=0, in=180] ([xshift=5mm]title.east)--
			(title.center-|frame.east)
			[rounded corners=\kvtcb@arc] |-
			(frame.north) -| cycle;
		},
	#1
}{def}
\makeatother

%================================
% SOLUTION BOX
%================================

\makeatletter
\newtcolorbox{solution}{enhanced,
	breakable,
	colback=white,
	colframe=myg!80!black,
	attach boxed title to top left={yshift*=-\tcboxedtitleheight},
	title=Solution,
	boxed title size=title,
	boxed title style={%
			sharp corners,
			rounded corners=northwest,
			colback=tcbcolframe,
			boxrule=0pt,
		},
	underlay boxed title={%
			\path[fill=tcbcolframe] (title.south west)--(title.south east)
			to[out=0, in=180] ([xshift=5mm]title.east)--
			(title.center-|frame.east)
			[rounded corners=\kvtcb@arc] |-
			(frame.north) -| cycle;
		},
}
\makeatother

%================================
% Question BOX
%================================

\makeatletter
\newtcbtheorem{qstion}{Question}{enhanced,
	breakable,
	colback=white,
	colframe=mygr,
	attach boxed title to top left={yshift*=-\tcboxedtitleheight},
	fonttitle=\bfseries,
	title={#2},
	boxed title size=title,
	boxed title style={%
			sharp corners,
			rounded corners=northwest,
			colback=tcbcolframe,
			boxrule=0pt,
		},
	underlay boxed title={%
			\path[fill=tcbcolframe] (title.south west)--(title.south east)
			to[out=0, in=180] ([xshift=5mm]title.east)--
			(title.center-|frame.east)
			[rounded corners=\kvtcb@arc] |-
			(frame.north) -| cycle;
		},
	#1
}{def}
\makeatother

\newtcbtheorem[number within=section]{wconc}{Wrong Concept}{
	breakable,
	enhanced,
	colback=white,
	colframe=myr,
	arc=0pt,
	outer arc=0pt,
	fonttitle=\bfseries\sffamily\large,
	colbacktitle=myr,
	attach boxed title to top left={},
	boxed title style={
			enhanced,
			skin=enhancedfirst jigsaw,
			arc=3pt,
			bottom=0pt,
			interior style={fill=myr}
		},
	#1
}{def}



%================================
% NOTE BOX
%================================

\usetikzlibrary{arrows,calc,shadows.blur}
\tcbuselibrary{skins}
\newtcolorbox{note}[1][]{%
	enhanced jigsaw,
	colback=gray!20!white,%
	colframe=gray!80!black,
	size=small,
	boxrule=1pt,
	title=\textbf{Note:},
	halign title=flush center,
	coltitle=black,
	breakable,
	drop shadow=black!50!white,
	attach boxed title to top left={xshift=1cm,yshift=-\tcboxedtitleheight/2,yshifttext=-\tcboxedtitleheight/2},
	minipage boxed title=1.5cm,
	boxed title style={%
			colback=white,
			size=fbox,
			boxrule=1pt,
			boxsep=2pt,
			underlay={%
					\coordinate (dotA) at ($(interior.west) + (-0.5pt,0)$);
					\coordinate (dotB) at ($(interior.east) + (0.5pt,0)$);
					\begin{scope}
						\clip (interior.north west) rectangle ([xshift=3ex]interior.east);
						\filldraw [white, blur shadow={shadow opacity=60, shadow yshift=-.75ex}, rounded corners=2pt] (interior.north west) rectangle (interior.south east);
					\end{scope}
					\begin{scope}[gray!80!black]
						\fill (dotA) circle (2pt);
						\fill (dotB) circle (2pt);
					\end{scope}
				},
		},
	#1,
}


%%%%%%%%%%%%%%%%%%%%%%%%%%%%%%%%%%%%%%%%%%%
% TABLE OF CONTENTS
%%%%%%%%%%%%%%%%%%%%%%%%%%%%%%%%%%%%%%%%%%%

\usepackage{tikz}
\definecolor{doc}{RGB}{0,60,110}
\usepackage{titletoc}
\contentsmargin{0cm}
\titlecontents{chapter}[3.7pc]
{\addvspace{30pt}%
	\begin{tikzpicture}[remember picture, overlay]%
		\draw[fill=doc!60,draw=doc!60] (-7,-.1) rectangle (-0.5,.5);%
		\pgftext[left,x=-3.5cm,y=0.2cm]{\color{white}\Large\sc\bfseries Chapter\ \thecontentslabel};%
	\end{tikzpicture}\color{doc!60}\large\sc\bfseries}%
{}
{}
{\;\titlerule\;\large\sc\bfseries Page \thecontentspage
	\begin{tikzpicture}[remember picture, overlay]
		\draw[fill=doc!60,draw=doc!60] (2pt,0) rectangle (4,0.1pt);
	\end{tikzpicture}}%
\titlecontents{section}[3.7pc]
{\addvspace{2pt}}
{\contentslabel[\thecontentslabel]{2pc}}
{}
{\hfill\small \thecontentspage}
[]
\titlecontents{subsection}[3.7pc]
{\addvspace{-1pt}\small\qquad}
{\contentslabel[\thecontentslabel]{2pc}}
{}
{\ --- \small\thecontentspage}
[]

\makeatletter
\renewcommand{\tableofcontents}{%
	\chapter*{%
	  \vspace*{-20\p@}%
	  \begin{tikzpicture}[remember picture, overlay]%
		  \pgftext[right,x=15cm,y=0.2cm]{\color{doc!60}\Huge\sc\bfseries \contentsname};%
		  \draw[fill=doc!60,draw=doc!60] (13,-.75) rectangle (20,1);%
		  \clip (13,-.75) rectangle (20,1);
		  \pgftext[right,x=15cm,y=0.2cm]{\color{white}\Huge\sc\bfseries \contentsname};%
	  \end{tikzpicture}}%
	\@starttoc{toc}}
\makeatother


%%%%%%%%%%%%%%%%%%%%%%%%%%%%%%
% SELF MADE COMMANDS
%%%%%%%%%%%%%%%%%%%%%%%%%%%%%%

\newcommand{\thm}[2]{\begin{Theorem}{#1}{}\setlength{\parindent}{0.5cm}#2\end{Theorem}}
\newcommand{\cor}[2]{\begin{Corollary}{#1}{}\setlength{\parindent}{0.5cm}#2\end{Corollary}}
\newcommand{\mlemma}[2]{\begin{Lemma}{#1}{}\setlength{\parindent}{0.5cm}#2\end{Lemma}}
\newcommand{\mprop}[2]{\begin{Prop}{#1}{}\setlength{\parindent}{0.5cm}#2\end{Prop}}
\newcommand{\clm}[2]{\begin{claim}{#1}{}\setlength{\parindent}{0.5cm}#2\end{claim}}
\newcommand{\wc}[2]{\begin{wconc}{#1}{}\setlength{\parindent}{1cm}#2\end{wconc}}
\newcommand{\thmcon}[1]{\begin{Theoremcon}{\setlength{\parindent}{0.5cm}#1}\end{Theoremcon}}
\newcommand{\ex}[2]{\begin{Example}{#1}{}\setlength{\parindent}{0.5cm}#2\end{Example}}
\newcommand{\dfn}[2]{\begin{Definition}[colbacktitle=red!75!black]{#1}{}\setlength{\parindent}{0.5cm}#2\end{Definition}}
\newcommand{\dfns}[2]{\begin{definition}[colbacktitle=red!75!black]{#1}{}\setlength{\parindent}{0.5cm}#2\end{definition}}
\newcommand{\qs}[2]{\begin{question}{#1}{}\setlength{\parindent}{0.5cm}#2\end{question}}
\newcommand{\sol}[1]{\begin{solution}{\setlength{\parindent}{0.5cm}#1}{}\end{solution}}
\newcommand{\pf}[2]{\begin{myproof}[#1]\setlength{\parindent}{0.5cm}#2\end{myproof}}
\newcommand{\nt}[1]{\begin{note}\setlength{\parindent}{0.5cm}#1\end{note}}

\newcommand*\circled[1]{\tikz[baseline=(char.base)]{
		\node[shape=circle,draw,inner sep=1pt] (char) {#1};}}
\newcommand\getcurrentref[1]{%
	\ifnumequal{\value{#1}}{0}
	{??}
	{\the\value{#1}}%
}
\newcommand{\getCurrentSectionNumber}{\getcurrentref{section}}
\newenvironment{myproof}[1][\proofname]{%
	\proof[\bfseries #1: ]%
}{\endproof}

\newcommand{\mclm}[2]{\begin{myclaim}[#1]#2\end{myclaim}}
\newenvironment{myclaim}[1][\claimname]{\proof[\bfseries #1: ]}{}

\newcounter{mylabelcounter}

\makeatletter
\newcommand{\setword}[2]{%
	\phantomsection
	#1\def\@currentlabel{\unexpanded{#1}}\label{#2}%
}
\makeatother




\tikzset{
	symbol/.style={
			draw=none,
			every to/.append style={
					edge node={node [sloped, allow upside down, auto=false]{$#1$}}}
		}
}




% % redefine matrix env to allow for alignment, use r as default
% \renewcommand*\env@matrix[1][r]{\hskip -\arraycolsep
%     \let\@ifnextchar\new@ifnextchar
%     \array{*\c@MaxMatrixCols #1}}


%\usepackage{framed}
%\usepackage{titletoc}
%\usepackage{etoolbox}
%\usepackage{lmodern}


%\patchcmd{\tableofcontents}{\contentsname}{\sffamily\contentsname}{}{}

%\renewenvironment{leftbar}
%{\def\FrameCommand{\hspace{6em}%
%		{\color{myyellow}\vrule width 2pt depth 6pt}\hspace{1em}}%
%	\MakeFramed{\parshape 1 0cm \dimexpr\textwidth-6em\relax\FrameRestore}\vskip2pt%
%}
%{\endMakeFramed}

%\titlecontents{chapter}
%[0em]{\vspace*{2\baselineskip}}
%{\parbox{4.5em}{%
%		\hfill\Huge\sffamily\bfseries\color{myred}\thecontentspage}%
%	\vspace*{-2.3\baselineskip}\leftbar\textsc{\small\chaptername~\thecontentslabel}\\\sffamily}
%{}{\endleftbar}
%\titlecontents{section}
%[8.4em]
%{\sffamily\contentslabel{3em}}{}{}
%{\hspace{0.5em}\nobreak\itshape\color{myred}\contentspage}
%\titlecontents{subsection}
%[8.4em]
%{\sffamily\contentslabel{3em}}{}{}  
%{\hspace{0.5em}\nobreak\itshape\color{myred}\contentspage}




\usepackage{amsfonts, amsmath, amssymb, amsthm, dsfont, mathtools}
\usepackage{enumitem}
\usepackage{graphicx}
\usepackage{setspace}
\usepackage{indentfirst}
\usepackage[margin=1in]{geometry}
\graphicspath{{./images/}}
\setstretch{1.15}

\newtheorem{thm}{Theorem}
\newtheorem{proposition}[thm]{Proposition}
\newtheorem{corollary}[thm]{Corollary}
\newtheorem{lemma}[thm]{Lemma}

\newcommand*{\Var}{\ensuremath{\mathrm{Var}}}
\newcommand*{\Cov}{\ensuremath{\mathrm{Cov}}}
\newcommand*{\Corr}{\ensuremath{\mathrm{Corr}}}
\newcommand*{\Bias}{\ensuremath{\mathrm{Bias}}}
\newcommand*{\MSE}{\ensuremath{\mathrm{MSE}}}

\newcommand*{\range}{\ensuremath{\mathrm{range}}\,}
\newcommand*{\spann}{\ensuremath{\mathrm{span}}\,}
\newcommand*{\nul}{\ensuremath{\mathrm{null}}\,}
\newcommand*{\dom}{\ensuremath{\mathrm{dom}}\,}

\newcommand*{\pdv}[3][]{\frac{\partial^{#1}}{\partial#3^{#1}}#2}
\renewcommand*{\implies}{\ensuremath{\Longrightarrow}}
\renewcommand*{\impliedby}{\ensuremath{\Longleftarrow}}
\renewcommand*{\l}{\left}
\renewcommand*{\r}{\right}
\renewcommand{\leq}{\leqslant}
\renewcommand{\geq}{\geqslant}

\newcommand*{\Z}{\ensuremath{\mathbb{Z}}}
\newcommand*{\Q}{\ensuremath{\mathbb{Q}}}
\newcommand*{\R}{\ensuremath{\mathbb{R}}}
\newcommand*{\F}{\ensuremath{\mathbb{F}}}
\newcommand*{\C}{\ensuremath{\mathbb{C}}}
\newcommand*{\N}{\ensuremath{\mathbb{N}}}
\newcommand*{\E}{\ensuremath{\mathds{E}}}
\renewcommand*{\P}{\ensuremath{\mathds{P}}}
\newcommand*{\p}{\ensuremath{\mathcal{P}}}
\renewcommand*{\L}{\mathcal{L}}


% deliminators
\DeclarePairedDelimiter{\abs}{\lvert}{\rvert}
\DeclarePairedDelimiter{\norm}{\|}{\|}
\DeclarePairedDelimiter{\inner}{\langle}{\rangle}
\DeclarePairedDelimiter{\ceil}{\lceil}{\rceil}
\DeclarePairedDelimiter{\floor}{\lfloor}{\rfloor}


\let\oldleq\leq % save them in case they're every wanted
\let\oldgeq\geq

\newcommand*{\img}[3][]{
    \begin{figure}[htb!]
         \centering
         \includegraphics[scale=#1]{#2}
         \caption{#3}
    \end{figure}
}
\newcommand*{\imgs}[5][]{
    \begin{figure}[htb]
        \qquad
        \begin{minipage}{.4\textwidth}
            \centering
            \includegraphics[scale=#1]{#2}
            \caption{#3}
        \end{minipage}    
        \qquad
        \begin{minipage}{.4\textwidth}
            \centering
            \includegraphics[scale=#1]{#4}
            \caption{#5}
        \end{minipage}
    \end{figure} 
}


\title{\Huge{MATH 104 Notes}\\\emph{Book: Elementary Analysis - Ross}}
\author{\huge{Neo Lee}}
\date{Fall 2023}

\begin{document}
\text{Computer Modern}\selectfont
\maketitle
\let\cleardoublepage\clearpage
\pdfbookmark[section]{\contentsname}{toc}
\tableofcontents

\chapter{Metric Spaces}
\section{Some Topological Concepts in Metric Spaces}
\dfn{Metric and Metric Space}{
	Let $S$ be a set, and suppose $d$ is a function defined for all paris $(x,y)$ of elements 
	from $S$ satisfying 
	\begin{enumerate}
		\item [\textbf{D1.}] $d(x,x)=0$ for all $x\in S$ and $d(x,y)>0$ for distinct $x,y$ in $S$.
		\item [\textbf{D2.}] $d(x,y)=d(y,x)$ for all $x,y\in S$.
		\item [\textbf{D3.}] $d(x,z)\le d(x,y)+d(y,z)$ for all $x,y,z\in S$.
	\end{enumerate}

	We call such function $d$ the \emph{distance function} or a \emph{metric}  on $S$. A 
	\emph{metric space} $S$ is a set $S$ together with a metric on it. Properly speaking, the 
	metric space is the pair $S,d$ since a set $S$ may have more than one metric on it.
}

\ex{}{
	Let dist$(a,b)=|a-b|$ for $a,b\in\R$. Then dist is a metric on \R. Notice $\textbf{D3}$ follows 
	direcly from the triangular inequality for real numbers. Therefore, for any metric $d$, we 
	call the property $\textbf{D3}$ the \emph{triangular inequality}.
}
\ex{}{
	The space of all $k$-tuples
	$$\tb{x}=(x_1,x_2,\dots,x_k)\qquad \text{where}\qquad x_j\in\R\qquad\text{for }j=1,2,\dots,k,$$
	is called \emph{k-dimensional Euclidean space} and written $\R^k$. 

	$R^k$ has multiple metrics on it, and the most familiar metric is the \emph{Euclidean norm}:
	$$d(\tb{x},\tb{y})=\sqrt{(x_1-y_1)^2+(x_2-y_2)^2+\dots+(x_k-y_k)^2}\qquad \text{for }
	\tb{x},\tb{y}\in\R^k.$$
}

\dfn{Convergence in Metric Space}{
	A sequence $(s_n)$ in a metric space $S,d$ \emph{converges} to $s\in S$ if 
	$$\lim_{n\to\infty}d(s_n,s)=0.$$ A sequence $(s_n)\in S$ is a \emph{Cauchy sequence} if
	for each $\epsilon>0$ there is an integer $N$ such that 
	$$m,n>N \implies d(s_m,s_n)<\epsilon.$$
	
	\nt{
		The metric space $(S,d)$ is \emph{complete} if every Cauchy sequence in $S$ is convergent.
	}
}
\mlemma{}{
	A sequence $(\tb{x}^{(n)})\in \R^k$ converges if and only if each of its component sequences
	$(x_j^{(n)})$ converges in \R. A sequence $(\tb{x}^{(n)})\in\R^k$ is a Cauchy sequence if and
	only if each of its component sequences $(x_j^{(n)})$ is a Cauchy sequence in \R.
}
\thm{}{
	Euclidean $k$-space $\R^k$ is complete.
	\nt{
		In other words, every Cauchy sequence in $\R^k$ is convergent.
	}
}
\dfn{Boundedness of $\R^k$}{
	A set $S\in\R^k$ is bounded if there exists $M>0$ such that 
	$$\max\{|x_j|:j=1,2,\dots,k\}\le M$$ for all $\tb{x}=(x_1,x_2,\dots,x_k)\in S.$
}
\thm{Bolzano-Weierstrass Theorem for Euclidean $k$-space}{
	Eveery bounded sequence in $\R^k$ has a convergent subsequence.
}

\dfn{Open Set}{
	Let $(S,d)$ be a metric space. Let $E$ be a subset of $S$. An element $s_0\in E$ is interior to 
	$E$ if for some $r>0$ we have 
	$$\{s\in S:d(s,s_0)<0\}\subseteq E.$$

	\nt{
		In other words, $s_0$ is interior to $E$ if there exists an open ball with radius $r>0$ 
		centered at $s_0$ such that all points $\in S$ and in the ball are still in $E$.
	}

	We denote $E^\circ$ as the set of points in $E$ that are interior to $E$. The set $E$ is
	\emph{open} on $S$ if every point of $E$ is interior to $E$, i.e., if $E=E^\circ$.
}
\thm{Property of Open Set}{
	\begin{enumerate}[label=\textbf{(\roman*)}]
		\item $S$ is open in $S$.
		\item The empty set $\emptyset$ is open in $S$.
		\item The union of any collection of open sets in $S$ is also open in $S$.
		\item The intersection of finitely many open sets in $S$ is also open in $S$.
	\end{enumerate}
}
\nt{
	Metric spaces are fairly general and useful objects, in the sense that when one is interested 
	in convergence of certain objects (such as points or functions), there is often a metric 
	that assists in the study of the convergence.

	However, sometimes no metric will work but there is still some sort of convergence notion. 
	Frequently, the toolbox we need for studying such behavior is what is called a \emph{topology}. 
	
	Topologies are more general than metrics, and they are defined in terms of open sets rather
	than distance functions. In particular, the open sets defined by a metric form a topology.
	Because of this abstract theory of topology, concepts that can be defined in terms of open sets 
	are called topological. Hence, we are slowly entering the realm of topology by studying open
	sets in metric spaces.
}
\dfn{Closed Set}{
	Let $(S,d)$ be a metric space. A subset $E$ of $S$ is \emph{closed} if its complement
	$S\setminus E$ is an open set in $S$. Because of \emph{Theorem 1.3 (iii)}, the intersection of any 
	collection of closed sets in $S$ is also closed in $S$. 
	
	The closure $E^-$ of a set $E$ is the intersection of all closed sets in $S$ that 
	contain $E$. 

	The \emph{boundary} of a set $E$ is the set of $E^-\setminus E^\circ$; points in the boundary
	set are called \emph{boundary points} of $E$.

	\nt{
		For every bounary point $s$ of $E$, all open balls centered at $s$ must contain some point in $E$
		and some point in $S\setminus E$. Because if $s$ has an open ball that does not contain any points
		in $E$, then $s$ is not in a limit point of $E$ and thus not in $E^-$. On the other hand, if
		$s$ has an open ball that does not contain points in $S\setminus E$, then that open ball is 
		contained in $E$ and thus $s$ is interior to $E$.
	}
}
\thm{Property of Closed Set}{
	\begin{enumerate}[label=\textbf{(\roman*)}]
		\item The set $E$ is closed if and only if $E^-=E$.
		\item The set $E$ is closed if and only if it contains the limits of all convergent
		sequences of points in $E$.
		\item An element is in $E^-$ if and only if it is the limit of some sequence of points in $E$.
		\item A point $s$ is in the boundary of $E$ if and only if it belongs to both the closure of 
		$E$ and the closure of $S\setminus E$.
	\end{enumerate}
}
\ex{Open Intervals and Closed Intervals in \R}{
	In \R, open intervals $(a,b)$ are open sets. Closed intervals $[a,b]$ are closed sets. 
	The interior of $[a,b]$ is $(a,b)$. The boundary of both $(a,b)$ and $[a,b]$ is the 
	two-element set $\{a,b\}$.

	Every open set in $\R$ is the union of a disjoint sequence of open intervals. However, 
	a closed set in $\R$ need not be the union of a disjoint sequence of closed intervals and points.
}
\ex{Open Balls and Closed Balls}{
	In $\R^k$, open balls $\{\tb{x}:d(\tb{x},\tb{x}_0)<r\}$ are open sets, and closed balls 
	$\{\tb{x}:d(\tb{x},\tb{x}_0)\le r\}$ are closed sets. The boundary of each of these sets is
	the sphere $\{\tb{x}:d(\tb{x},\tb{x}_0)=r\}$.

	In the plane $\R^2$, the sets 
	$$\{(x,y):x>0\}\qquad\t{and}\qquad\{(x,y):x>0, y>0\}$$
	are open. If $>$ is replaced by $\ge$, we obtain clsoed sets. Many sets are neither open nor 
	closed. For example, $[0,1)$ are neither open nor closed in \R, and 
	$\{(x,y):x>0, y\ge 0\}$ is neither open nor closed in $\R^2$.
}
\thm{}{
	Let $F_n$ be a decreasing sequence $[i.e., F_1 \supseteqq F_2\supseteq\cdots]$ of closed, 
	bounded, and nonempty sets in $\R^k$. Then $\bigcap_{n=1}^\infty F_n$ is also closed, bounded, 
	and nonempty.

	\nt{
		\begin{itemize}
			\item 
			We can think of this theorem as a generalization of the Nested Interval Property in \R.
			\item 
			Also, we can substitute the word "closed and bounded" with "compact" in the theorem.
		\end{itemize}
	}
}
\ex{Cantor Set}{
	Cantor set is a nonempty, closed, and bounded set in \R. It is constructed by removing the
	middle third of the interval $[0,1]$, then removing the middle thirds of the two remaining
	pieces, and so on. The Cantor set is the set of all points that remain after this process is
	repeated infinitely many times.

	The sum of the intervals removed at the $n$th stage is $\l(\frac{2}{3}\r)^n$, and this tends to 
	zero as $n\to\infty$. Yet, the points remaining after the infinite process is so large that 
	it cannot be written as a sequence; in set-theoretic terms it is "uncountable". The interior 
	of the Cantor set is empty, and hence its boundary is the set itself.
}
\dfn{}{
	Let $(S,d)$ be a metric space. A family $\mathcal{U}$ of open sets is said to be an 
	\emph{open cover} for a set $E$ if each point of $E$ belongs to at least one set in 
	$\mathcal{U}$, i.e.,
	$$E\subseteq\{U:U\in\U\}.$$

	A \emph{subcover} of $\U$ is any subfamily of $\U$ that still covers $E$. A cover or subcover 
	is finite if it contains only finitely many sets; yet the sets themselves can be infinite.

	A set $E$ is \emph{compact} if every open cover of $E$ has a finite subcover.

	\nt{
		The word "family" here emphasizes it is a collections of sets, usually indexed loosely. 
		So a family is a set of sets.
	}
}
\thm{Heine-Borel Theorem}{
	A subset $E$ of $\R^k$ is compact if and only if it is closed and bounded.
}
\dfn{$k$-cell}{
	A \emph{$k$-cell} $F$ in $\R^k$ is a set of the form 
	$$\{x\in\R^k:a_j\le x_j\le b_j\text{ for }j=1,2,\dots,k\}$$
	where $a_j,b_j\in\R$ and $a_j\le b_j$ for $j=1,2,\dots,k$.

	The \emph{diameter} of $F$ is 
	$$\delta=\sqrt{(b_1-a_1)^2+(b_2-a_2)^2+\dots+(b_k-a_k)^2},$$
	that is, $\delta = \sup\{d(\tb{x},\tb{y}):\tb{x},\tb{y}\in F\}.$
	\nt{
		$F$ is a union of $2^k$ $k$-cells each having diameter $\delta/2$.
	}
}
\ex{}{
	A 2-cell in $\R^2$ is a rectangle. A 3-cell in $\R^3$ is a rectangular box.
}
\thm{}{
	Every $k$-cell in $\R^k$ is compact.
}
\ex{}{
	Let $E$ be a nonempty subset of a metric space $(S,d)$. Consider the function 
	$d(E,x)=\inf\{d(y,x):y\in E\}$ for $x\in S$. This function satifies $|d(E,x_1)-d(E,x_2)|\le d(x_1,x_2)$.

	We want to show that if $E$ is compact and if $E\subseteq U$ for some open subset $U\subseteq S$,
	then for some $\delta>0$, we have 
	$$\{y\in S:d(E,y)<\delta\}\subseteq U,$$ in other words, any point in $S$ that is within a 
	distance $\delta$ of $E$ is in $U$.

	\pf{Proof}{
		Since $U$ is open, for each $x\in E$, which is also in $U$, there exists some 
		$r_x$ such that $$\{y\in S:d(y,x)<r_x\}\subseteq U\t{ for some } r_x>0.$$

		The open balls $\{y\in S:d(y,x)<\frac{r_x}{2}\}$ cover $E$. Since $E$ is compact, a finite 
		subfamily also covers $E$. That is there are $x_1,\dots,x_n\in E$ so that 
		$$E\subseteq \bigcup_{i=1}^n\{y\in S:d(y,x_i)<\frac{r_{x_i}}{2}\}.$$
		Let $\delta=\frac{1}{2}\min\{r_{x_1},\dots,r_{x_n}\}$.

		Now consider an arbitrary $y\in S$ and $d(E,y)<\delta$. Then for some $x\in E$, 
		we have $d(y, x)<\delta$. Moreover, $d(x,x_k)<\frac{r_{x_k}}{2}$ for some $k\in
		\{1,2,\dots,n\}$. Therefore, by "passing through" this closest point $x$ to reach 
		$x_k$, 
		$$d(y,x_k)\le d(x,y)+d(x,x_k)<\delta+\frac{r_k}{2}\le\frac{r_k}{2}+\frac{r_k}{2}=r_k.$$
		Hence, $y$ is within the "safe" open ball centered at $x_k$, which is contained in $U$.
	}
}




\section{Metric Spaces: Continuity}
\section{Metric Spaces: Connectedness}
We generealize the notion of "connectedness" from $\R$ to metric spaces.

\dfn{Disconnected}{
	Let $E$ be a subset of metric space $(S,d)$. The set $E$ is \emph{disconnected}  if one of the 
	following two equivalent conditions holds:
	\begin{enumerate}[label=\textbf{(\alph*)}]
		\item There are open subsets $U_1$ and $U_2$ of $S$ such that 
		\begin{align}
			(E\cap U_1)\cap(E\cap U_2)=\emptyset& \text{ and }E=(E\cap U_1)\cup(E\cap U_2)\\
			E\cap U_1\neq\emptyset& \text{ and }E\cap U_2\neq\emptyset.
		\end{align}

		\item There are nonempty disjoint subsets $A$ and $B$ of $E$ such that such that 
		$E=A\cup B$ and neither set intersects the closure of the other set, i.e.,
		\begin{align}
			A^-\cap B \neq \emptyset& \text{ and }A\cap B^- \neq \emptyset.
		\end{align}
	\end{enumerate}
	We say that $E$ is \emph{connected} if it is not disconnected.
}




\end{document}
