\documentclass{book}
\input{preamble.tex}
\newcommand*{\Var}{\ensuremath{\mathrm{Var}}}
\newcommand*{\Cov}{\ensuremath{\mathrm{Cov}}}
\newcommand*{\Corr}{\ensuremath{\mathrm{Corr}}}
\newcommand*{\Bias}{\ensuremath{\mathrm{Bias}}}
\newcommand*{\MSE}{\ensuremath{\mathrm{MSE}}}

\newcommand*{\range}{\ensuremath{\mathrm{range}}\,}
\newcommand*{\spann}{\ensuremath{\mathrm{span}}\,}
\newcommand*{\nul}{\ensuremath{\mathrm{null}}\,}
\newcommand*{\dom}{\ensuremath{\mathrm{dom}}\,}

\newcommand*{\pdv}[3][]{\frac{\partial^{#1}}{\partial#3^{#1}}#2}
\renewcommand*{\implies}{\ensuremath{\Longrightarrow}}
\renewcommand*{\impliedby}{\ensuremath{\Longleftarrow}}
\renewcommand*{\l}{\left}
\renewcommand*{\r}{\right}
\renewcommand{\leq}{\leqslant}
\renewcommand{\geq}{\geqslant}
\newcommand*{\tb}{\textbf}
\renewcommand*{\t}{\text}

\newcommand*{\Z}{\ensuremath{\mathbb{Z}}}
\newcommand*{\Q}{\ensuremath{\mathbb{Q}}}
\newcommand*{\R}{\ensuremath{\mathbb{R}}}
\newcommand*{\F}{\ensuremath{\mathbb{F}}}
\newcommand*{\C}{\ensuremath{\mathbb{C}}}
\newcommand*{\N}{\ensuremath{\mathbb{N}}}
\newcommand*{\E}{\ensuremath{\mathds{E}}}
\renewcommand*{\P}{\ensuremath{\mathds{P}}}
\newcommand*{\p}{\ensuremath{\mathcal{P}}}
\renewcommand*{\L}{\mathcal{L}}


% deliminators
\DeclarePairedDelimiter{\abs}{\lvert}{\rvert}
\DeclarePairedDelimiter{\norm}{\|}{\|}
\DeclarePairedDelimiter{\inner}{\langle}{\rangle}
\DeclarePairedDelimiter{\ceil}{\lceil}{\rceil}
\DeclarePairedDelimiter{\floor}{\lfloor}{\rfloor}
\DeclarePairedDelimiter{\round}{\lfloor}{\rceil}

\let\oldleq\leq % save them in case they're every wanted
\let\oldgeq\geq

\newcommand*{\img}[3][]{
    \begin{figure}[htb!]
         \centering
         \includegraphics[scale=#1]{#2}
         \caption{#3}
    \end{figure}
}
\newcommand*{\imgs}[5][]{
    \begin{figure}[htb]
        \qquad
        \begin{minipage}{.4\textwidth}
            \centering
            \includegraphics[scale=#1]{#2}
            \caption{#3}
        \end{minipage}    
        \qquad
        \begin{minipage}{.4\textwidth}
            \centering
            \includegraphics[scale=#1]{#4}
            \caption{#5}
        \end{minipage}
    \end{figure} 
}


\title{\Huge{MATH 105 Notes}\\\emph{Professor: Khalilah Beal-Uribe}}
\author{\huge{Neo Lee}}
\date{\huge{Spring 2024}}

\begin{document}

\maketitle
\let\cleardoublepage\clearpage
\pdfbookmark[section]{\contentsname}{toc}
\tableofcontents

\chapter{First chapter}
\section{Lecture 1}
\dfn{Norm}{
	Given a vector space $V$ over a subfield $\F$ of $\C$, a norm of $V$ is a real-valued function
	$p:V\to\R$ satisfying the following properties:
	\begin{enumerate}
		\item \tb{Triangle inequality:} $p(v+w)\leq p(v)+p(w)$,
		\item \tb{Absolute homogeneity:} $p(\alpha v) = |\alpha|p(v)$,
		\item \tb{Positive definiteness:} $p(v) \geq 0$ and $p(v) = 0$ iff $v = 0$.
	\end{enumerate}
	\nt{
		Usually, we denote the norm of $v$ by $\norm{v}$, and for clarity of the underlying vector 
		space, we may write $\norm{v}_V$.
	}
}
\mprop{Normed space is a metric space}{
	Let $V$ be a normed space. Then the function $d:V\times V\to\R$ defined by
	$d(v,w) = p(v-w) = \norm{v-w}$ is a metric on $V$.
}
\dfn{Isomorphism in vector spaces}{
	A function $f:V\to W$ between two vector spaces $V$ and $W$ over the same field $\F$ is called an
	isomorphism if it is bijective and linear. If such an isomorphism exists, we say that the two vector
	spaces are isomorphic.	
}
\dfn{Homeomorphism}{
	A function $f:X\to Y$ between two topological spaces $X$ and $Y$ is called a homeomorphism if it
	satisfies the following properties:
	\begin{enumerate}
		\item $f$ is bijective,
		\item $f$ is continuous,
		\item $f^{-1}$ is continuous.
	\end{enumerate}
	If such a homeomorphism exists, we say that the two topological spaces are homeomorphic.
}
\nt{
	In general, isomorphism does not imply homeomorphism. However, in certain cases, they are
	equivalent, which will be discussed in details later.
}
\dfn{Operator norm}{
	Let $T:V\to W$ be a linear oepration between normed spaces. Denote $\norm{\cdot}_V$ and
	$\norm{\cdot}_W$ be the norms in $V$ and $W$ respectively. The operator norm of $A$ is defined 
	by 
	\begin{align*}
		\norm{T} & = \sup \l\{\frac{\norm{Tv}_W}{\norm{v}_V}: v\neq 0, v\in V \r\} \\
		& = \inf \l\{c\geq 0: \norm{Tv}_W\leq c\norm{v}_V, \forall v\in V \r\}
	\end{align*}
	\nt{
		We say that $T$ is bounded if $\norm{T} < \infty$.
	}
}

\section{Lecture 2}
\thm{Multiplication of matrices are composition of linear maps}{
	$T_A\circ T_b=T_{AB}.$
}
\thm{Bounded operator is equivalent to continuous}{
	Let $T:V\to W$ be a linear transformation from one normed space to another. The following are
	equivalent:
	\begin{enumerate}
		\item $\norm{T} < \infty$,
		\item $T$ is uniformly continuous,
		\item $T$ is continuous,
		\item $T$ is continuous at $0$.
	\end{enumerate}
	\pf{Proof}{
		We show that $(1)\implies (2)\implies (3)\implies (4)\implies (1)$.
		\begin{itemize}
			\item $(1)\implies (2)$: Let $M=\norm{T}<\infty$, and let $\delta = \frac{\epsilon}{M}$.
			Then for any $x,y\in V$ such that $\norm{x-y}<\delta$, we have 
			\begin{align*}
				\norm{Tx-Ty} & = \norm{T(x-y)} \\
				& \leq M\norm{x-y} \\
				& < M\delta \\
				& = \epsilon.
			\end{align*}
			Hence, $T$ is uniformly continuous.

			\item $(2)\implies (3)$: Trivial. Uniformly continuous automatically implies continuous.
			\item $(3)\implies (4)$: Trivial. $T$ is continuous over the whole domain implies that
			it is continuous at any point in the domain, including $0$.

			\item $(4)\implies (1)$: Let $\epsilon=1$, then there exists $\delta>0$ such that 
			$\norm{x}<\delta$ implies $\norm{Tx}<1$. Then for any $v\neq 0$, define
			$v'=\frac{\delta}{2\norm{v}}$, then $\norm{v'}<\delta$ and hence
			$\norm{Tv'}<1$. Then we have 
			\begin{align*}
				\norm{Tv'} & < 1\\
				\l\|T\l(\frac{\delta}{2\norm{v}}v\r)\r\| & < 1 \\
				\frac{\delta}{2\norm{v}}\norm{Tv} & < 1 \\
				\norm{Tv} & < \frac{2}{\delta}\norm{v}.
			\end{align*}
			Then, from our \emph{definition 1.1.4} of operator norm, we have
			$\norm{T}<\frac{2}{\delta}$ and hence $\norm{T}<\infty$.
		\end{itemize}
	}
}
\thm{Linear map from finite-dimensional Euclidean space to normed space is continuous}{
	Let $T:\R^n\to W$, where $T$ is linear and $W$ is a normed space. Then 
	\begin{enumerate}
		\item $T$ is continuous,
		\item if $T$ is an isomorphism, then $T$ is a homeomorphism.
	\end{enumerate}
}
\cor{Linear maps from finite-dimensional normed space to normed space are continuous}{
	All linear maps from finite-dimensional normed space to another normed space are continuous, and all
	isomorphisms from finite-dimensional space to normed space are homeomorphisms.

	In particular, if a finite-dimensional vector spaces is equipped with two norms, then the
	identity map between them is a homeomorphism. For example, $T:\M\to\L$ is a homeomorphism.
	\pf{Proof}{
		Let $V$ be a $n$-dimensional normed space and $W$ be another normed space, and $T:V\to W$. Then,
		there exists an isoemorphism $S:V\to\R^n$. \emph{Theorem 1.2.2} gaurentees that $S$ and $S^{-1}$
		are homeomorphisms. Then, $T\circ S:\R^n\to W$ is also a continuous linear map guaranteed by
		\emph{Theorem 1.2.2}. Then, $$T = (T\circ S)\circ S^{-1}$$ is also a continuous linear because
		it is a composition of continuous linear maps. Hence, $T$ is continuous.

		Now, if $T:V\to W$ is an isomorphism where $V$ is a finite-dimensional normed space. Then, 
		$W$ is also a finite-dimensional normed space. Then, $T$ is continuous by the above argument.
		Then, $T^{-1}:W\to V$ is a linear map from a finite-dimensional normed space, hence also 
		continuous. Therefore, $T$ is a homeomorphism.

		Finally, let $V$ be a finite-dimenisonal vector space equipped with two norms
		$\norm{\cdot}_1$ and $\norm{\cdot}_2$. Then, the identity map $I:V\to V$ is an isomorphism 
		between the two finite-dimensional normed spaces. Then, $I$ is a homeomorphism by the above 
		argument.
	}
}


\chapter{Starting a new chapter}
\section{Demo of commands}
\dfn{Some defintion}{
	yap
}
\qs{Some question}{yap}
\sol{
	\pf{Some proof}{yap}
}
\nt{Some note}
\thm{Some theorem}{
	yap
}
\wc{Some wrong concept}{
	yap
}
\mlemma{Some lemma}{
	yap
}
\mprop{Some proposition}{
	yap
}
\ex{Some example}{
	yap
}
\clm{Some claim}{
	yap
}
\cor{Some corollary}{
	yap
}
\thmcon{Some unlabeled theorem}

This is a new paragraph


\begin{algorithm}
	\caption{Some algorithm}
	\KwIn{input}
	\KwOut{output}
	\SetAlgoLined
	\SetNoFillComment
	\tcc{This is a comment}
	This is first line \tcp*{This is also a comment}
	\uIf{$x > 5$} {
		do nothing
	} \uElseIf {$x < 5$} {
		do nothing
	} \Else {
		do nothing
	}
	\While{$x == 5$}{
		still do nothing
	}
	\ForEach{$x = 1:5$}{
		do nothing
	}
	\Return{return nothing}
\end{algorithm}


\end{document}
