\documentclass{book}
%--------------------------
% This amazing template is from 
% https://github.com/SeniorMars/dotfiles/tree/master/latex_template
%--------------------------

%%%%%%%%%%%%%%%%%%%%%%%%%%%%%%%%%
% PACKAGE IMPORTS
%%%%%%%%%%%%%%%%%%%%%%%%%%%%%%%%%


\usepackage[tmargin=2cm,rmargin=1in,lmargin=1in,margin=0.85in,bmargin=2cm,footskip=.2in]{geometry}
\usepackage{amsmath,amsfonts,amsthm,amssymb,mathtools}
\usepackage[varbb]{newpxmath}
\usepackage{xfrac}
\usepackage{indentfirst}
\usepackage[T1]{fontenc}
\usepackage[makeroom]{cancel}
\usepackage{bookmark}
\usepackage{enumitem}
\usepackage{hyperref,theoremref}
\hypersetup{
	pdftitle={Assignment},
	colorlinks=true, linkcolor=doc!90,
	bookmarksnumbered=true,
	bookmarksopen=true
}
\usepackage[most,many,breakable]{tcolorbox}
\usepackage{xcolor}
\usepackage{varwidth}
\usepackage{varwidth}
\usepackage{etoolbox}
\usepackage{authblk}
\usepackage{nameref}
\usepackage{multicol,array}
\usepackage{tikz-cd}
\usepackage[ruled,vlined,linesnumbered]{algorithm2e}
\usepackage{comment} % enables the use of multi-line comments (\ifx \fi) 
\usepackage{import}
\usepackage{xifthen}
\usepackage{pdfpages}
\usepackage{transparent}
\usepackage{setspace}
\setstretch{1.15}
\fontfamily{cmr}\selectfont
\DeclareSymbolFont{letters}     {OML}{cmm} {m}{it}


\newcommand\mycommfont[1]{\footnotesize\ttfamily\textcolor{blue}{#1}}
\SetCommentSty{mycommfont}
\newcommand{\incfig}[1]{%
    \def\svgwidth{\columnwidth}
    \import{./figures/}{#1.pdf_tex}
}

\usepackage{tikzsymbols}
\renewcommand\qedsymbol{$\Laughey$}


%%%%%%%%%%%%%%%%%%%%%%%%%%%%%%
% SELF MADE COLORS
%%%%%%%%%%%%%%%%%%%%%%%%%%%%%%


\definecolor{myg}{RGB}{56, 140, 70}
\definecolor{myb}{RGB}{45, 111, 177}
\definecolor{myr}{RGB}{199, 68, 64}
\definecolor{mytheorembg}{HTML}{F2F2F9}
\definecolor{mytheoremfr}{HTML}{00007B}
\definecolor{mylemmabg}{HTML}{FFFAF8}
\definecolor{mylemmafr}{HTML}{983b0f}
\definecolor{mypropbg}{HTML}{f2fbfc}
\definecolor{mypropfr}{HTML}{191971}
\definecolor{myexamplebg}{HTML}{F2FBF8}
\definecolor{myexamplefr}{HTML}{88D6D1}
\definecolor{myexampleti}{HTML}{2A7F7F}
\definecolor{mydefinitbg}{HTML}{E5E5FF}
\definecolor{mydefinitfr}{HTML}{3F3FA3}
\definecolor{notesgreen}{RGB}{0,162,0}
\definecolor{myp}{RGB}{197, 92, 212}
\definecolor{mygr}{HTML}{2C3338}
\definecolor{myred}{RGB}{127,0,0}
\definecolor{myyellow}{RGB}{169,121,69}
\definecolor{myexercisebg}{HTML}{F2FBF8}
\definecolor{myexercisefg}{HTML}{88D6D1}


%%%%%%%%%%%%%%%%%%%%%%%%%%%%
% TCOLORBOX SETUPS
%%%%%%%%%%%%%%%%%%%%%%%%%%%%

\setlength{\parindent}{1cm}
%================================
% THEOREM BOX
%================================

\tcbuselibrary{theorems,skins,hooks}
\newtcbtheorem[number within=section]{Theorem}{Theorem}
{%
	enhanced,
	breakable,
	colback = mytheorembg,
	frame hidden,
	boxrule = 0sp,
	borderline west = {2pt}{0pt}{mytheoremfr},
	sharp corners,
	detach title,
	before upper = \tcbtitle\par\smallskip,
	coltitle = mytheoremfr,
	fonttitle = \bfseries\sffamily,
	description font = \mdseries,
	separator sign none,
	segmentation style={solid, mytheoremfr},
}
{th}

\tcbuselibrary{theorems,skins,hooks}
\newtcbtheorem[number within=section]{theorem}{Theorem}
{%
	enhanced,
	breakable,
	colback = mytheorembg,
	frame hidden,
	boxrule = 0sp,
	borderline west = {2pt}{0pt}{mytheoremfr},
	sharp corners,
	detach title,
	before upper = \tcbtitle\par\smallskip,
	coltitle = mytheoremfr,
	fonttitle = \bfseries\sffamily,
	description font = \mdseries,
	separator sign none,
	segmentation style={solid, mytheoremfr},
}
{th}


\tcbuselibrary{theorems,skins,hooks}
\newtcolorbox{Theoremcon}
{%
	enhanced
	,breakable
	,colback = mytheorembg
	,frame hidden
	,boxrule = 0sp
	,borderline west = {2pt}{0pt}{mytheoremfr}
	,sharp corners
	,description font = \mdseries
	,separator sign none
}

%================================
% Corollery
%================================
\tcbuselibrary{theorems,skins,hooks}
\newtcbtheorem[number within=section]{Corollary}{Corollary}
{%
	enhanced
	,breakable
	,colback = myp!10
	,frame hidden
	,boxrule = 0sp
	,borderline west = {2pt}{0pt}{myp!85!black}
	,sharp corners
	,detach title
	,before upper = \tcbtitle\par\smallskip
	,coltitle = myp!85!black
	,fonttitle = \bfseries\sffamily
	,description font = \mdseries
	,separator sign none
	,segmentation style={solid, myp!85!black}
}
{th}
\tcbuselibrary{theorems,skins,hooks}
\newtcbtheorem[number within=section]{corollary}{Corollary}
{%
	enhanced
	,breakable
	,colback = myp!10
	,frame hidden
	,boxrule = 0sp
	,borderline west = {2pt}{0pt}{myp!85!black}
	,sharp corners
	,detach title
	,before upper = \tcbtitle\par\smallskip
	,coltitle = myp!85!black
	,fonttitle = \bfseries\sffamily
	,description font = \mdseries
	,separator sign none
	,segmentation style={solid, myp!85!black}
}
{th}


%================================
% LEMMA
%================================

\tcbuselibrary{theorems,skins,hooks}
\newtcbtheorem[number within=section]{Lemma}{Lemma}
{%
	enhanced,
	breakable,
	colback = mylemmabg,
	frame hidden,
	boxrule = 0sp,
	borderline west = {2pt}{0pt}{mylemmafr},
	sharp corners,
	detach title,
	before upper = \tcbtitle\par\smallskip,
	coltitle = mylemmafr,
	fonttitle = \bfseries\sffamily,
	description font = \mdseries,
	separator sign none,
	segmentation style={solid, mylemmafr},
}
{th}

\tcbuselibrary{theorems,skins,hooks}
\newtcbtheorem[number within=section]{lemma}{Lemma}
{%
	enhanced,
	breakable,
	colback = mylemmabg,
	frame hidden,
	boxrule = 0sp,
	borderline west = {2pt}{0pt}{mylemmafr},
	sharp corners,
	detach title,
	before upper = \tcbtitle\par\smallskip,
	coltitle = mylemmafr,
	fonttitle = \bfseries\sffamily,
	description font = \mdseries,
	separator sign none,
	segmentation style={solid, mylemmafr},
}
{th}


%================================
% PROPOSITION
%================================

\tcbuselibrary{theorems,skins,hooks}
\newtcbtheorem[number within=section]{Prop}{Proposition}
{%
	enhanced,
	breakable,
	colback = mypropbg,
	frame hidden,
	boxrule = 0sp,
	borderline west = {2pt}{0pt}{mypropfr},
	sharp corners,
	detach title,
	before upper = \tcbtitle\par\smallskip,
	coltitle = mypropfr,
	fonttitle = \bfseries\sffamily,
	description font = \mdseries,
	separator sign none,
	segmentation style={solid, mypropfr},
}
{th}

\tcbuselibrary{theorems,skins,hooks}
\newtcbtheorem[number within=section]{prop}{Proposition}
{%
	enhanced,
	breakable,
	colback = mypropbg,
	frame hidden,
	boxrule = 0sp,
	borderline west = {2pt}{0pt}{mypropfr},
	sharp corners,
	detach title,
	before upper = \tcbtitle\par\smallskip,
	coltitle = mypropfr,
	fonttitle = \bfseries\sffamily,
	description font = \mdseries,
	separator sign none,
	segmentation style={solid, mypropfr},
}
{th}


%================================
% CLAIM
%================================

\tcbuselibrary{theorems,skins,hooks}
\newtcbtheorem[number within=section]{claim}{Claim}
{%
	enhanced
	,breakable
	,colback = myg!10
	,frame hidden
	,boxrule = 0sp
	,borderline west = {2pt}{0pt}{myg}
	,sharp corners
	,detach title
	,before upper = \tcbtitle\par\smallskip
	,coltitle = myg!85!black
	,fonttitle = \bfseries\sffamily
	,description font = \mdseries
	,separator sign none
	,segmentation style={solid, myg!85!black}
}
{th}



%================================
% Exercise
%================================

\tcbuselibrary{theorems,skins,hooks}
\newtcbtheorem[number within=section]{Exercise}{Exercise}
{%
	enhanced,
	breakable,
	colback = myexercisebg,
	frame hidden,
	boxrule = 0sp,
	borderline west = {2pt}{0pt}{myexercisefg},
	sharp corners,
	detach title,
	before upper = \tcbtitle\par\smallskip,
	coltitle = myexercisefg,
	fonttitle = \bfseries\sffamily,
	description font = \mdseries,
	separator sign none,
	segmentation style={solid, myexercisefg},
}
{th}

\tcbuselibrary{theorems,skins,hooks}
\newtcbtheorem[number within=section]{exercise}{Exercise}
{%
	enhanced,
	breakable,
	colback = myexercisebg,
	frame hidden,
	boxrule = 0sp,
	borderline west = {2pt}{0pt}{myexercisefg},
	sharp corners,
	detach title,
	before upper = \tcbtitle\par\smallskip,
	coltitle = myexercisefg,
	fonttitle = \bfseries\sffamily,
	description font = \mdseries,
	separator sign none,
	segmentation style={solid, myexercisefg},
}
{th}

%================================
% EXAMPLE BOX
%================================

\newtcbtheorem[number within=section]{Example}{Example}
{%
	colback = myexamplebg
	,breakable
	,colframe = myexamplefr
	,coltitle = myexampleti
	,boxrule = 1pt
	,sharp corners
	,detach title
	,before upper=\tcbtitle\par\smallskip
	,fonttitle = \bfseries
	,description font = \mdseries
	,separator sign none
	,description delimiters parenthesis
}
{ex}

\newtcbtheorem[number within=section]{example}{Example}
{%
	colback = myexamplebg
	,breakable
	,colframe = myexamplefr
	,coltitle = myexampleti
	,boxrule = 1pt
	,sharp corners
	,detach title
	,before upper=\tcbtitle\par\smallskip
	,fonttitle = \bfseries
	,description font = \mdseries
	,separator sign none
	,description delimiters parenthesis
}
{ex}

%================================
% DEFINITION BOX
%================================

\newtcbtheorem[number within=section]{Definition}{Definition}{enhanced,
	before skip=2mm,after skip=2mm, colback=red!5,colframe=red!80!black,boxrule=0.5mm,
	attach boxed title to top left={xshift=1cm,yshift*=1mm-\tcboxedtitleheight}, varwidth boxed title*=-3cm,
	boxed title style={frame code={
					\path[fill=tcbcolback]
					([yshift=-1mm,xshift=-1mm]frame.north west)
					arc[start angle=0,end angle=180,radius=1mm]
					([yshift=-1mm,xshift=1mm]frame.north east)
					arc[start angle=180,end angle=0,radius=1mm];
					\path[left color=tcbcolback!60!black,right color=tcbcolback!60!black,
						middle color=tcbcolback!80!black]
					([xshift=-2mm]frame.north west) -- ([xshift=2mm]frame.north east)
					[rounded corners=1mm]-- ([xshift=1mm,yshift=-1mm]frame.north east)
					-- (frame.south east) -- (frame.south west)
					-- ([xshift=-1mm,yshift=-1mm]frame.north west)
					[sharp corners]-- cycle;
				},interior engine=empty,
		},
	fonttitle=\bfseries,
	title={#2},#1}{def}
\newtcbtheorem[number within=section]{definition}{Definition}{enhanced,
	before skip=2mm,after skip=2mm, colback=red!5,colframe=red!80!black,boxrule=0.5mm,
	attach boxed title to top left={xshift=1cm,yshift*=1mm-\tcboxedtitleheight}, varwidth boxed title*=-3cm,
	boxed title style={frame code={
					\path[fill=tcbcolback]
					([yshift=-1mm,xshift=-1mm]frame.north west)
					arc[start angle=0,end angle=180,radius=1mm]
					([yshift=-1mm,xshift=1mm]frame.north east)
					arc[start angle=180,end angle=0,radius=1mm];
					\path[left color=tcbcolback!60!black,right color=tcbcolback!60!black,
						middle color=tcbcolback!80!black]
					([xshift=-2mm]frame.north west) -- ([xshift=2mm]frame.north east)
					[rounded corners=1mm]-- ([xshift=1mm,yshift=-1mm]frame.north east)
					-- (frame.south east) -- (frame.south west)
					-- ([xshift=-1mm,yshift=-1mm]frame.north west)
					[sharp corners]-- cycle;
				},interior engine=empty,
		},
	fonttitle=\bfseries,
	title={#2},#1}{def}



%================================
% Solution BOX
%================================

\makeatletter
\newtcbtheorem{question}{Question}{enhanced,
	breakable,
	colback=white,
	colframe=myb!80!black,
	attach boxed title to top left={yshift*=-\tcboxedtitleheight},
	fonttitle=\bfseries,
	title={#2},
	boxed title size=title,
	boxed title style={%
			sharp corners,
			rounded corners=northwest,
			colback=tcbcolframe,
			boxrule=0pt,
		},
	underlay boxed title={%
			\path[fill=tcbcolframe] (title.south west)--(title.south east)
			to[out=0, in=180] ([xshift=5mm]title.east)--
			(title.center-|frame.east)
			[rounded corners=\kvtcb@arc] |-
			(frame.north) -| cycle;
		},
	#1
}{def}
\makeatother

%================================
% SOLUTION BOX
%================================

\makeatletter
\newtcolorbox{solution}{enhanced,
	breakable,
	colback=white,
	colframe=myg!80!black,
	attach boxed title to top left={yshift*=-\tcboxedtitleheight},
	title=Solution,
	boxed title size=title,
	boxed title style={%
			sharp corners,
			rounded corners=northwest,
			colback=tcbcolframe,
			boxrule=0pt,
		},
	underlay boxed title={%
			\path[fill=tcbcolframe] (title.south west)--(title.south east)
			to[out=0, in=180] ([xshift=5mm]title.east)--
			(title.center-|frame.east)
			[rounded corners=\kvtcb@arc] |-
			(frame.north) -| cycle;
		},
}
\makeatother

%================================
% Question BOX
%================================

\makeatletter
\newtcbtheorem{qstion}{Question}{enhanced,
	breakable,
	colback=white,
	colframe=mygr,
	attach boxed title to top left={yshift*=-\tcboxedtitleheight},
	fonttitle=\bfseries,
	title={#2},
	boxed title size=title,
	boxed title style={%
			sharp corners,
			rounded corners=northwest,
			colback=tcbcolframe,
			boxrule=0pt,
		},
	underlay boxed title={%
			\path[fill=tcbcolframe] (title.south west)--(title.south east)
			to[out=0, in=180] ([xshift=5mm]title.east)--
			(title.center-|frame.east)
			[rounded corners=\kvtcb@arc] |-
			(frame.north) -| cycle;
		},
	#1
}{def}
\makeatother

\newtcbtheorem[number within=section]{wconc}{Wrong Concept}{
	breakable,
	enhanced,
	colback=white,
	colframe=myr,
	arc=0pt,
	outer arc=0pt,
	fonttitle=\bfseries\sffamily\large,
	colbacktitle=myr,
	attach boxed title to top left={},
	boxed title style={
			enhanced,
			skin=enhancedfirst jigsaw,
			arc=3pt,
			bottom=0pt,
			interior style={fill=myr}
		},
	#1
}{def}



%================================
% NOTE BOX
%================================

\usetikzlibrary{arrows,calc,shadows.blur}
\tcbuselibrary{skins}
\newtcolorbox{note}[1][]{%
	enhanced jigsaw,
	colback=gray!20!white,%
	colframe=gray!80!black,
	size=small,
	boxrule=1pt,
	title=\textbf{Note:},
	halign title=flush center,
	coltitle=black,
	breakable,
	drop shadow=black!50!white,
	attach boxed title to top left={xshift=1cm,yshift=-\tcboxedtitleheight/2,yshifttext=-\tcboxedtitleheight/2},
	minipage boxed title=1.5cm,
	boxed title style={%
			colback=white,
			size=fbox,
			boxrule=1pt,
			boxsep=2pt,
			underlay={%
					\coordinate (dotA) at ($(interior.west) + (-0.5pt,0)$);
					\coordinate (dotB) at ($(interior.east) + (0.5pt,0)$);
					\begin{scope}
						\clip (interior.north west) rectangle ([xshift=3ex]interior.east);
						\filldraw [white, blur shadow={shadow opacity=60, shadow yshift=-.75ex}, rounded corners=2pt] (interior.north west) rectangle (interior.south east);
					\end{scope}
					\begin{scope}[gray!80!black]
						\fill (dotA) circle (2pt);
						\fill (dotB) circle (2pt);
					\end{scope}
				},
		},
	#1,
}


%%%%%%%%%%%%%%%%%%%%%%%%%%%%%%%%%%%%%%%%%%%
% TABLE OF CONTENTS
%%%%%%%%%%%%%%%%%%%%%%%%%%%%%%%%%%%%%%%%%%%

\usepackage{tikz}
\definecolor{doc}{RGB}{0,60,110}
\usepackage{titletoc}
\contentsmargin{0cm}
\titlecontents{chapter}[3.7pc]
{\addvspace{30pt}%
	\begin{tikzpicture}[remember picture, overlay]%
		\draw[fill=doc!60,draw=doc!60] (-7,-.1) rectangle (-0.5,.5);%
		\pgftext[left,x=-3.5cm,y=0.2cm]{\color{white}\Large\sc\bfseries Chapter\ \thecontentslabel};%
	\end{tikzpicture}\color{doc!60}\large\sc\bfseries}%
{}
{}
{\;\titlerule\;\large\sc\bfseries Page \thecontentspage
	\begin{tikzpicture}[remember picture, overlay]
		\draw[fill=doc!60,draw=doc!60] (2pt,0) rectangle (4,0.1pt);
	\end{tikzpicture}}%
\titlecontents{section}[3.7pc]
{\addvspace{2pt}}
{\contentslabel[\thecontentslabel]{2pc}}
{}
{\hfill\small \thecontentspage}
[]
\titlecontents{subsection}[3.7pc]
{\addvspace{-1pt}\small\qquad}
{\contentslabel[\thecontentslabel]{2pc}}
{}
{\ --- \small\thecontentspage}
[]

\makeatletter
\renewcommand{\tableofcontents}{%
	\chapter*{%
	  \vspace*{-20\p@}%
	  \begin{tikzpicture}[remember picture, overlay]%
		  \pgftext[right,x=15cm,y=0.2cm]{\color{doc!60}\Huge\sc\bfseries \contentsname};%
		  \draw[fill=doc!60,draw=doc!60] (13,-.75) rectangle (20,1);%
		  \clip (13,-.75) rectangle (20,1);
		  \pgftext[right,x=15cm,y=0.2cm]{\color{white}\Huge\sc\bfseries \contentsname};%
	  \end{tikzpicture}}%
	\@starttoc{toc}}
\makeatother


%%%%%%%%%%%%%%%%%%%%%%%%%%%%%%
% SELF MADE COMMANDS
%%%%%%%%%%%%%%%%%%%%%%%%%%%%%%

\newcommand{\thm}[2]{\begin{Theorem}{#1}{}\setlength{\parindent}{0.5cm}#2\end{Theorem}}
\newcommand{\cor}[2]{\begin{Corollary}{#1}{}\setlength{\parindent}{0.5cm}#2\end{Corollary}}
\newcommand{\mlemma}[2]{\begin{Lemma}{#1}{}\setlength{\parindent}{0.5cm}#2\end{Lemma}}
\newcommand{\mprop}[2]{\begin{Prop}{#1}{}\setlength{\parindent}{0.5cm}#2\end{Prop}}
\newcommand{\clm}[2]{\begin{claim}{#1}{}\setlength{\parindent}{0.5cm}#2\end{claim}}
\newcommand{\wc}[2]{\begin{wconc}{#1}{}\setlength{\parindent}{1cm}#2\end{wconc}}
\newcommand{\thmcon}[1]{\begin{Theoremcon}{\setlength{\parindent}{0.5cm}#1}\end{Theoremcon}}
\newcommand{\ex}[2]{\begin{Example}{#1}{}\setlength{\parindent}{0.5cm}#2\end{Example}}
\newcommand{\dfn}[2]{\begin{Definition}[colbacktitle=red!75!black]{#1}{}\setlength{\parindent}{0.5cm}#2\end{Definition}}
\newcommand{\dfns}[2]{\begin{definition}[colbacktitle=red!75!black]{#1}{}\setlength{\parindent}{0.5cm}#2\end{definition}}
\newcommand{\qs}[2]{\begin{question}{#1}{}\setlength{\parindent}{0.5cm}#2\end{question}}
\newcommand{\sol}[1]{\begin{solution}{\setlength{\parindent}{0.5cm}#1}{}\end{solution}}
\newcommand{\pf}[2]{\begin{myproof}[#1]\setlength{\parindent}{0.5cm}#2\end{myproof}}
\newcommand{\nt}[1]{\begin{note}\setlength{\parindent}{0.5cm}#1\end{note}}

\newcommand*\circled[1]{\tikz[baseline=(char.base)]{
		\node[shape=circle,draw,inner sep=1pt] (char) {#1};}}
\newcommand\getcurrentref[1]{%
	\ifnumequal{\value{#1}}{0}
	{??}
	{\the\value{#1}}%
}
\newcommand{\getCurrentSectionNumber}{\getcurrentref{section}}
\newenvironment{myproof}[1][\proofname]{%
	\proof[\bfseries #1: ]%
}{\endproof}

\newcommand{\mclm}[2]{\begin{myclaim}[#1]#2\end{myclaim}}
\newenvironment{myclaim}[1][\claimname]{\proof[\bfseries #1: ]}{}

\newcounter{mylabelcounter}

\makeatletter
\newcommand{\setword}[2]{%
	\phantomsection
	#1\def\@currentlabel{\unexpanded{#1}}\label{#2}%
}
\makeatother




\tikzset{
	symbol/.style={
			draw=none,
			every to/.append style={
					edge node={node [sloped, allow upside down, auto=false]{$#1$}}}
		}
}




% % redefine matrix env to allow for alignment, use r as default
% \renewcommand*\env@matrix[1][r]{\hskip -\arraycolsep
%     \let\@ifnextchar\new@ifnextchar
%     \array{*\c@MaxMatrixCols #1}}


%\usepackage{framed}
%\usepackage{titletoc}
%\usepackage{etoolbox}
%\usepackage{lmodern}


%\patchcmd{\tableofcontents}{\contentsname}{\sffamily\contentsname}{}{}

%\renewenvironment{leftbar}
%{\def\FrameCommand{\hspace{6em}%
%		{\color{myyellow}\vrule width 2pt depth 6pt}\hspace{1em}}%
%	\MakeFramed{\parshape 1 0cm \dimexpr\textwidth-6em\relax\FrameRestore}\vskip2pt%
%}
%{\endMakeFramed}

%\titlecontents{chapter}
%[0em]{\vspace*{2\baselineskip}}
%{\parbox{4.5em}{%
%		\hfill\Huge\sffamily\bfseries\color{myred}\thecontentspage}%
%	\vspace*{-2.3\baselineskip}\leftbar\textsc{\small\chaptername~\thecontentslabel}\\\sffamily}
%{}{\endleftbar}
%\titlecontents{section}
%[8.4em]
%{\sffamily\contentslabel{3em}}{}{}
%{\hspace{0.5em}\nobreak\itshape\color{myred}\contentspage}
%\titlecontents{subsection}
%[8.4em]
%{\sffamily\contentslabel{3em}}{}{}  
%{\hspace{0.5em}\nobreak\itshape\color{myred}\contentspage}




\usepackage{amsfonts, amsmath, amssymb, amsthm, dsfont, mathtools}
\usepackage{enumitem}
\usepackage{graphicx}
\usepackage{setspace}
\usepackage{indentfirst}
\usepackage[margin=1in]{geometry}
\graphicspath{{./images/}}
\setstretch{1.15}

\newtheorem{thm}{Theorem}
\newtheorem{proposition}[thm]{Proposition}
\newtheorem{corollary}[thm]{Corollary}
\newtheorem{lemma}[thm]{Lemma}

\newcommand*{\Var}{\ensuremath{\mathrm{Var}}}
\newcommand*{\Cov}{\ensuremath{\mathrm{Cov}}}
\newcommand*{\Corr}{\ensuremath{\mathrm{Corr}}}
\newcommand*{\Bias}{\ensuremath{\mathrm{Bias}}}
\newcommand*{\MSE}{\ensuremath{\mathrm{MSE}}}

\newcommand*{\range}{\ensuremath{\mathrm{range}}\,}
\newcommand*{\spann}{\ensuremath{\mathrm{span}}\,}
\newcommand*{\nul}{\ensuremath{\mathrm{null}}\,}
\newcommand*{\dom}{\ensuremath{\mathrm{dom}}\,}

\newcommand*{\pdv}[3][]{\frac{\partial^{#1}}{\partial#3^{#1}}#2}
\renewcommand*{\implies}{\ensuremath{\Longrightarrow}}
\renewcommand*{\impliedby}{\ensuremath{\Longleftarrow}}
\renewcommand*{\l}{\left}
\renewcommand*{\r}{\right}
\renewcommand{\leq}{\leqslant}
\renewcommand{\geq}{\geqslant}

\newcommand*{\Z}{\ensuremath{\mathbb{Z}}}
\newcommand*{\Q}{\ensuremath{\mathbb{Q}}}
\newcommand*{\R}{\ensuremath{\mathbb{R}}}
\newcommand*{\F}{\ensuremath{\mathbb{F}}}
\newcommand*{\C}{\ensuremath{\mathbb{C}}}
\newcommand*{\N}{\ensuremath{\mathbb{N}}}
\newcommand*{\E}{\ensuremath{\mathds{E}}}
\renewcommand*{\P}{\ensuremath{\mathds{P}}}
\newcommand*{\p}{\ensuremath{\mathcal{P}}}
\renewcommand*{\L}{\mathcal{L}}


% deliminators
\DeclarePairedDelimiter{\abs}{\lvert}{\rvert}
\DeclarePairedDelimiter{\norm}{\|}{\|}
\DeclarePairedDelimiter{\inner}{\langle}{\rangle}
\DeclarePairedDelimiter{\ceil}{\lceil}{\rceil}
\DeclarePairedDelimiter{\floor}{\lfloor}{\rfloor}


\let\oldleq\leq % save them in case they're every wanted
\let\oldgeq\geq

\newcommand*{\img}[3][]{
    \begin{figure}[htb!]
         \centering
         \includegraphics[scale=#1]{#2}
         \caption{#3}
    \end{figure}
}
\newcommand*{\imgs}[5][]{
    \begin{figure}[htb]
        \qquad
        \begin{minipage}{.4\textwidth}
            \centering
            \includegraphics[scale=#1]{#2}
            \caption{#3}
        \end{minipage}    
        \qquad
        \begin{minipage}{.4\textwidth}
            \centering
            \includegraphics[scale=#1]{#4}
            \caption{#5}
        \end{minipage}
    \end{figure} 
}


\title{\Huge{MATH 105 Notes}\\\emph{Professor: Khalilah Beal-Uribe}\\\emph{Book: Real Mathematical
Analysis\thanks{An introductory but holistic, intuitive, and easy to read book.} by Pugh}}
\author{\huge{Neo Lee}}

\date{\huge{Spring 2024}}

\begin{document}

\maketitle
\let\cleardoublepage\clearpage
\pdfbookmark[section]{\contentsname}{toc}
\tableofcontents

\chapter{First chapter}
\section{Lecture 1}
\dfn{Norm}{
	Given a vector space $V$ over a subfield $\F$ of $\C$, a norm of $V$ is a real-valued function
	$p:V\to\R$ satisfying the following properties:
	\begin{enumerate}
		\item \tb{Triangle inequality:} $p(v+w)\leq p(v)+p(w)$,
		\item \tb{Absolute homogeneity:} $p(\alpha v) = |\alpha|p(v)$,
		\item \tb{Positive definiteness:} $p(v) \geq 0$ and $p(v) = 0$ iff $v = 0$.
	\end{enumerate}
	\nt{
		Usually, we denote the norm of $v$ by $\norm{v}$, and for clarity of the underlying vector 
		space, we may write $\norm{v}_V$.
	}
}
\mprop{Normed space is a metric space}{
	Let $V$ be a normed space. Then the function $d:V\times V\to\R$ defined by
	$d(v,w) = p(v-w) = \norm{v-w}$ is a metric on $V$.
}
\dfn{Isomorphism in vector spaces}{
	A function $f:V\to W$ between two vector spaces $V$ and $W$ over the same field $\F$ is called an
	isomorphism if it is bijective and linear. If such an isomorphism exists, we say that the two vector
	spaces are isomorphic.	
}
\dfn{Homeomorphism}{
	A function $f:X\to Y$ between two topological spaces $X$ and $Y$ is called a homeomorphism if it
	satisfies the following properties:
	\begin{enumerate}
		\item $f$ is bijective,
		\item $f$ is continuous,
		\item $f^{-1}$ is continuous.
	\end{enumerate}
	If such a homeomorphism exists, we say that the two topological spaces are homeomorphic.
}
\nt{
	In general, isomorphism does not imply homeomorphism. However, in certain cases, they are
	equivalent, which will be discussed in details later.
}
\dfn{Operator norm}{
	Let $T:V\to W$ be a linear oepration between normed spaces. Denote $\norm{\cdot}_V$ and
	$\norm{\cdot}_W$ be the norms in $V$ and $W$ respectively. The operator norm of $A$ is defined 
	by 
	\begin{align*}
		\norm{T} & = \sup \l\{\frac{\norm{Tv}_W}{\norm{v}_V}: v\neq 0, v\in V \r\} \\
		& = \inf \l\{c\geq 0: \norm{Tv}_W\leq c\norm{v}_V, \forall v\in V \r\}
	\end{align*}
	\nt{
		We say that $T$ is bounded if $\norm{T} < \infty$.
	}
}

\section{Lecture 2}
\thm{Multiplication of matrices are composition of linear maps}{
	$T_A\circ T_b=T_{AB}.$
}
\thm{Bounded operator is equivalent to continuous}{
	Let $T:V\to W$ be a linear transformation from one normed space to another. The following are
	equivalent:
	\begin{enumerate}
		\item $\norm{T} < \infty$,
		\item $T$ is uniformly continuous,
		\item $T$ is continuous,
		\item $T$ is continuous at $0$.
	\end{enumerate}
	\pf{Proof}{
		We show that $(1)\implies (2)\implies (3)\implies (4)\implies (1)$.
		\begin{itemize}
			\item $(1)\implies (2)$: Let $M=\norm{T}<\infty$, and let $\delta = \frac{\epsilon}{M}$.
			Then for any $x,y\in V$ such that $\norm{x-y}<\delta$, we have 
			\begin{align*}
				\norm{Tx-Ty} & = \norm{T(x-y)} \\
				& \leq M\norm{x-y} \\
				& < M\delta \\
				& = \epsilon.
			\end{align*}
			Hence, $T$ is uniformly continuous.

			\item $(2)\implies (3)$: Trivial. Uniformly continuous automatically implies continuous.
			\item $(3)\implies (4)$: Trivial. $T$ is continuous over the whole domain implies that
			it is continuous at any point in the domain, including $0$.

			\item $(4)\implies (1)$: Let $\epsilon=1$, then there exists $\delta>0$ such that 
			$\norm{x}<\delta$ implies $\norm{Tx}<1$. Then for any $v\neq 0$, define
			$v'=\frac{\delta}{2\norm{v}}$, then $\norm{v'}<\delta$ and hence
			$\norm{Tv'}<1$. Then we have 
			\begin{align*}
				\norm{Tv'} & < 1\\
				\l\|T\l(\frac{\delta}{2\norm{v}}v\r)\r\| & < 1 \\
				\frac{\delta}{2\norm{v}}\norm{Tv} & < 1 \\
				\norm{Tv} & < \frac{2}{\delta}\norm{v}.
			\end{align*}
			Then, from our \emph{definition 1.1.4} of operator norm, we have
			$\norm{T}<\frac{2}{\delta}$ and hence $\norm{T}<\infty$.
		\end{itemize}
	}
}
\thm{Linear map from finite-dimensional Euclidean space to normed space is continuous}{
	Let $T:\R^n\to W$, where $T$ is linear and $W$ is a normed space. Then 
	\begin{enumerate}
		\item $T$ is continuous,
		\item if $T$ is an isomorphism, then $T$ is a homeomorphism.
	\end{enumerate}
}
\cor{Linear maps from finite-dimensional normed space to normed space are continuous}{
	All linear maps from finite-dimensional normed space to another normed space are continuous, and all
	isomorphisms from finite-dimensional space to normed space are homeomorphisms.

	In particular, if a finite-dimensional vector spaces is equipped with two norms, then the
	identity map between them is a homeomorphism. For example, $T:\M\to\L$ is a homeomorphism.
	\pf{Proof}{
		Let $V$ be a $n$-dimensional normed space and $W$ be another normed space, and $T:V\to W$. Then,
		there exists an isoemorphism $S:V\to\R^n$. \emph{Theorem 1.2.2} gaurentees that $S$ and $S^{-1}$
		are homeomorphisms. Then, $T\circ S:\R^n\to W$ is also a continuous linear map guaranteed by
		\emph{Theorem 1.2.2}. Then, $$T = (T\circ S)\circ S^{-1}$$ is also a continuous linear because
		it is a composition of continuous linear maps. Hence, $T$ is continuous.

		Now, if $T:V\to W$ is an isomorphism where $V$ is a finite-dimensional normed space. Then, 
		$W$ is also a finite-dimensional normed space. Then, $T$ is continuous by the above argument.
		Then, $T^{-1}:W\to V$ is a linear map from a finite-dimensional normed space, hence also 
		continuous. Therefore, $T$ is a homeomorphism.

		Finally, let $V$ be a finite-dimenisonal vector space equipped with two norms
		$\norm{\cdot}_1$ and $\norm{\cdot}_2$. Then, the identity map $I:V\to V$ is an isomorphism 
		between the two finite-dimensional normed spaces. Then, $I$ is a homeomorphism by the above 
		argument.
	}
}

\section{Lecture 3}
Our goal is to generalize on-variable differentiation to multi-variable differentiation. In more
precise terms, we want to:
\nt{
	Obtain a natural derivative of $F:U\to\R^m$ at a point $p\in$ open set $U\subseteq\R^n$ by
	generalizing the derivative of $f:U\to \R$ at a point $p\in U\subseteq\R$.

	The key is to understand that $f$ is differentiable at $p$ if and only if $f$ is "approximately
	linear" at $p$.
}
	Consider an example in 2-dimensional Euclidean space to motivate our new definition of
	derivatives in multi-dimensional spaces. 
\ex{Derivative in $\R^2$}{
	Is $f:\R^2\to\R^2$ defined by $$f(x_1,x_2):=(x_1^2,x_2^2)$$ differentiable at the point
	$(1,2)\in\R^2$? 

	\sol{
		Let's first try to use the definition of derivative in $\R$ to see if it works. Let
		$p=(1,2)$, then we have 
		$$f'(p) = \lim_{h\to 0}\frac{f(p+h)-f(p)}{h} = \lim_{h\to
		0}\frac{f(\inner{1,2}+h)-f(1,2)}{h}$$ where $h$ is a vector. But this does not make sense 
		because we have not defined what it meant by division of a vector by a scalar. Hence, we
		need a new definition of derivative in $\R^2$, or more generally in multi-dimensional
		spaces. 
	}
}
\dfn{Multi-variable derivative (aka total derivative or Frechet derivative)}{
		Let $f:U\to\R^m$ be given where $U$ is an open subset of $\R^n$. The function $f$ is
		differentiable at $p\in U$ with derivative $(Df)_p=T$ if $T:\R^n\to\R^m$ is a linear
		transformation and 
		$$f(p+v)=f(p)+T(v)+R(v)\implies \lim_{\norm{v}\to 0}\frac{R(v)}{\norm{v}}=0.$$
		\nt{
			The form is coming from the definition of derivative in $\R$ by rearranging the terms in 
			$$f'(x)=\lim_{h\to0}\frac{f(x+h)-f(x)}{h}$$ to $$f(x+h)=f(x)+f'(x)h+R(h)\implies
			\lim_{h\to0}\frac{R(h)}{\norm{h}}=0.$$ We say that they Taylor remainder $R$ is
			\emph{sublinear} because it tends to 0 faster than $\norm{v}$.
		}
		\nt{
			Our definition of differentiability is coordinate free, which means we can study
			differentiation on spaces other than $\R^n$, e.g. differential manifolds, which is the
			natural next topic to study after $\R^n$.
		}
}
\ex{Back to \emph{example 1.3.1}}{
	Under our new definition, we can try to determine the differentiability of $f$ at $p=(1,2)$.
	Write \begin{align*}
		f(p+v) & = f(1+v_1,2+v_2) \\
		& = \inner{(1+v_1)^2,(2+v_2)^2} \\
		& = \inner{1+2v_1+v_1^2,4+4v_2+v_2^2} \\
		& = \inner{1,4} + \inner{2v_1,4v_2} + \inner{v_1^2,v_2^2} \\
		& = f(p) + \inner{2v_1,4v_2} + \inner{v_1^2,v_2^2}.
	\end{align*}
	Then, we can define $T:\R^2\to\R^2$ by $T(v_1,v_2)=\inner{2v_1,4v_2}$ and $R:\R^2\to\R^2$ by
	$R(v_1,v_2)=\inner{v_1^2,v_2^2}$. Now we just have to show that
	$\lim_{\norm{v}\to0}\frac{R(v)}{\norm{v}}=0$. We can check whether the norm of
	$\frac{R(v)}{\norm{v}}$ go to 0 when $\norm{v}\to0$. It doesn't matter which norm we choose (the
	Euclidean norm, sum norm, or max norm, etc.), because they are equivalent in finite-dimensional
	spaces. Let's choose the Euclidean norm for simplicity. Then, we have
	\begin{align*}
		\l\|\frac{R(v)}{\norm{v}}\r\| = \frac{1}{\norm{v}}\norm{\inner{v_1^2, v_2^2}} 
		& = \sqrt{\frac{v_1^4 + v_2^4}{v_1^2+v_2^2}} \\
		& = \sqrt{\frac{v_1^4}{v_1^2+v_2^2} + \frac{v_2^4}{v_1^2+v_2^2}} \\
		& \leq \sqrt{\frac{v_1^4}{v_1^2} + \frac{v_2^4}{v_2^2}} = \sqrt{v_1^2 + v_2^2} = \norm{v},
	\end{align*}
	which obviously goes to 0 when $\norm{v}\to0$. Hence, $f$ is differentiable at $p=(1,2)$ with 
	derivative $T(v_1,v_2)=\inner{2v_1,4v_2}$.
	\nt{
		Note: this $T$ is only true for $p=(1,2)$. For other points, we may have different derivatives. 
	}
}
\thm{Derivative is unique}{
	If $f$ is differentiable at $p$, then it uniquely determins $(Df)_p$ according to the limit
	formula, valid for all $u\in\R^n$,
	$$(Df)_p(u)=\lim_{t\to0}\frac{f(p+tu)-f(p)}{t}.$$
	\pf{Proof}{
		Let $T$ be a linear transformation that satisfies \emph{definition 1.3.1}. Now fix any
		$u\in\R^n$ and take $v=tu$. Then
		$$\frac{f(p+tu)-f(p)}{t}=\frac{T(tu)+R(tu)}{t}=T(u)+\frac{R(tu)}{t\norm{u}}\norm{u} = T(u) +
		\frac{R(tu)}{\norm{tu}}\norm{u}.$$ The last term converges to 0 as $t\to 0$ since
		$\norm{tu}\to 0$. Limits, when they exist, are unique, so $T(u)$ is uniquely determined.
	}
}
\thm{Differentiability implies continuity}{
	\pf{Proof}{
		Differentiability at $p$ implies that $$|f(p+v)-f(p)|=|(Df)_p(v)+R(v)|\leq |(Df)_p(v)|+|R(v)|
		= \norm{(Df)_p}|v| + |R(v)|,$$ which tends to 0 as $v\to 0$ since $\norm{(Df)_p}$ is finite
		in a finite-dimensional space and $R$ is sublinear. Hence, $f$ is continuous at $p$.
	}
}
\cor{Total derivative existsnce implies partial derivative existence}{}
\ex{Some important concepts}{
	\begin{itemize}
		\item All partial derivatives exist at a point does not imply total derivative exists at
		that point.
		\item All directional derivatives exist at a point does not imply total derivative exists at
		that point.
		\item Partial derivatives exist and are continuous at a point implies total derivative
		exists at that point.
	\end{itemize}
}


\chapter{Starting a new chapter}
\section{Demo of commands}
\dfn{Some defintion}{
	yap
}
\qs{Some question}{yap}
\sol{
	\pf{Some proof}{yap}
}
\nt{Some note}
\thm{Some theorem}{
	yap
}
\wc{Some wrong concept}{
	yap
}
\mlemma{Some lemma}{
	yap
}
\mprop{Some proposition}{
	yap
}
\ex{Some example}{
	yap
}
\clm{Some claim}{
	yap
}
\cor{Some corollary}{
	yap
}
\thmcon{Some unlabeled theorem}

This is a new paragraph


\begin{algorithm}
	\caption{Some algorithm}
	\KwIn{input}
	\KwOut{output}
	\SetAlgoLined
	\SetNoFillComment
	\tcc{This is a comment}
	This is first line \tcp*{This is also a comment}
	\uIf{$x > 5$} {
		do nothing
	} \uElseIf {$x < 5$} {
		do nothing
	} \Else {
		do nothing
	}
	\While{$x == 5$}{
		still do nothing
	}
	\ForEach{$x = 1:5$}{
		do nothing
	}
	\Return{return nothing}
\end{algorithm}


\end{document}
