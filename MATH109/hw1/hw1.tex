\documentclass{article}
\usepackage{amsfonts, amsmath, amssymb, amsthm} % Math notations imported
\usepackage{enumitem}
\usepackage[margin=1in]{geometry}

\newtheorem{thm}{Theorem}
\newtheorem{prop}[thm]{Proposition}
\newtheorem{cor}[thm]{Corollary}

% title information
\title{Math 109 HW1}
\author{Neo Lee}
\date{01/22/2023}

% main content
\begin{document} 

% placing title information; comment out if using fancyhdr
\maketitle 

\begin{enumerate}[label={(\arabic*)}]
% Q1
\item 
$$((P \wedge (Q \rightarrow P)) \vee (Q \wedge (P \rightarrow Q))) \rightarrow (P \wedge Q)$$
\begin{displaymath}
    \begin{array}{|c c|c|c|c}
    % |c c|c| means that there are three columns in the table and
    % a vertical bar ’|’ will be printed on the left and right borders,
    % and between the second and the third columns.
    % The letter ’c’ means the value will be centered within the column,
    % letter ’l’, left-aligned, and ’r’, right-aligned.
    P & Q & P \wedge (Q \rightarrow P) & Q \wedge (P \rightarrow Q) & ((P \wedge (Q \rightarrow P)) \vee (Q \wedge (P \rightarrow Q))) \rightarrow (P \wedge Q)\\ % Use & to separate the columns
    \hline  % Put a horizontal line between the table header and the rest.
    T & T & T & T & T\\
    T & F & T & F & F\\
    F & T & F & T & F\\
    F & F & F & F & T\\
    \end{array}
\end{displaymath}

% Q2
\item
\begin{itemize}
    \item 
    \begin{prop}
        $(P \rightarrow Q) \vee R \equiv P \rightarrow (Q \vee R)$
    \end{prop}
    \begin{proof}
        \begin{align}
            (P \rightarrow Q) \vee R & \equiv (\neg P \vee Q) \vee R \\
            & \equiv \neg P \vee (Q \vee R) \\
            & \equiv P \rightarrow (Q \vee R)
        \end{align}

        % \begin{displaymath}
        %     \begin{array}{|c c c|c|c|}
        %     P & Q & R & (P \rightarrow Q) \vee R & P \rightarrow (Q \vee R)\\ % Use & to separate the columns
        %     \hline  
        %     T & T & T & T & T\\
        %     T & T & F & T & T\\
        %     T & F & T & T & T\\
        %     T & F & F & F & F\\
        %     F & T & T & T & T\\
        %     F & T & F & T & T\\
        %     F & F & T & T & T\\
        %     F & F & F & T & T\\
        %     \end{array}
        % \end{displaymath}

    \end{proof}

    \item 
    \begin{prop}
        $(P \vee Q) \rightarrow R \not\equiv P \vee (Q \rightarrow R)$
    \end{prop}
    \begin{proof}
        \begin{displaymath}
            \begin{array}{|c c c|c|c|}
                P & Q & R & (P \vee Q) \rightarrow R & P \vee (Q \rightarrow R) \\
                \hline
                T & F & F & F & T
            \end{array}
        \end{displaymath}
    \end{proof}
    
    \item 
    \begin{prop}
        $(P \rightarrow Q) \wedge R \not\equiv P \rightarrow (Q \wedge R)$
    \end{prop}
    \begin{proof}
        \begin{displaymath}
            \begin{array}{|c c c|c|c|}
                P & Q & R & (P \rightarrow Q) \wedge R & P \rightarrow (Q \wedge R) \\
                \hline
                F & T & F & F & T
            \end{array}
        \end{displaymath}
    \end{proof}
    
    \item 
    \begin{prop}
        $(P \wedge Q) \rightarrow R \not\equiv P \wedge (Q \rightarrow R)$
    \end{prop}
    \begin{proof}
        \begin{displaymath}
            \begin{array}{|c c c|c|c|}
                P & Q & R & (P \wedge Q) \rightarrow R & P \wedge (Q \rightarrow R) \\
                \hline
                F & T & T & T & F
            \end{array}
        \end{displaymath}
    \end{proof}
\end{itemize}


% Q3
\item 
\begin{align}
    ((P \rightarrow Q) \wedge (Q \rightarrow R)) \rightarrow (R \rightarrow P) & \equiv ((\neg P \vee Q) \wedge (\neg Q \vee R)) \rightarrow (\neg R \vee P) \\
    & \equiv \neg((\neg P \vee Q) \wedge (\neg Q \vee R)) \vee (\neg R \vee P) \\ 
    & \equiv \neg (\neg P \vee Q) \vee \neg (\neg Q \vee R) \vee (\neg R \vee P)
\end{align}

    
\end{enumerate}
\end{document}