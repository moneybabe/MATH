\documentclass{article}
\usepackage{amsfonts, amsmath, amssymb, amsthm} % Math notations imported
\usepackage{enumitem}
\usepackage[margin=1in]{geometry}

\newtheorem{thm}{Theorem}
\newtheorem{prop}[thm]{Proposition}
\newtheorem{cor}[thm]{Corollary}

% title information
\title{Math 109 HW3}
\author{Neo Lee}
\date{02/14/2023}

% main content
\begin{document} 

% placing title information; comment out if using fancyhdr
\maketitle 

\textbf{Problem 6.5}
\begin{enumerate}[label={(\roman*)}]
    \item 
    \begin{prop}
        $A \subseteq B \Leftrightarrow A \cup B = B$.
    \end{prop}
    \begin{proof}
        $(\Rightarrow; A \cup B \subseteq B)$ $\forall x \in A \cup B, x \in B$ because $A \subseteq B$.

        $(\Rightarrow; B \subseteq A \cup B)$ By definition, $\forall y \in B, y \in B \cup S$ for any arbitrary set $S$. Therefore, $B \subseteq A \cup B$.
        
        Since $A \cup B \subseteq B$ and $B \subseteq A \cup B$, $A \cup B = B$, and $(\Rightarrow)$ is proved.

        $(\Leftarrow)$ By definition, $\forall z \in A, z \in A \cup S$ for any arbitrary set $S$, which means $A \subseteq A \cup S$.
        Hence, $A \subseteq A \cup B$, which is equivalent to $A \subseteq B$.
    \end{proof}

    \item 
    \begin{prop}
        $A \subseteq B \Leftrightarrow A \cap B = A$.
    \end{prop}
    \begin{proof}
        $(\Rightarrow; A \cap B \subseteq A)$ By definition, $\forall x \in A \cap B, x \in A$, thus $A \cap B \subseteq A$.

        $(\Rightarrow; A \subseteq A \cap B)$ $\forall y \in A, x \in A \cap B$ because $A \subseteq B$.

        Since $A \cap B \subseteq A$ and $A \subseteq A \cap B$, $A \cap B = A$, and $(\Rightarrow)$ is proved.

        $(\Leftarrow)$ By definition, $(B \cap S) \subseteq B$ for any arbitrary set $S$. 
        Hence, $A = A \cap B \subseteq B$.
    \end{proof}
\end{enumerate}
\bigbreak

\textbf{Problem 6.6}
\begin{prop}
    If $A \cap B \subseteq C$ and $x \in B$, then $x \not\in A-C$.
\end{prop}
\begin{proof}
    Assume to the contrary that if $A \cap B \subseteq C$ and $x \in B$, then $x \in A-C$. 
    It means that $x \in A$ and $x \not \in C$. Since $A \cap B \subseteq C$, $x \not \in C \Rightarrow x \not \in A \cap B$.
    We know $x \in A$ and $x \not \in A \cap B$, therefore, $x \in A \cap B^c$. 
    It means $x \in B^c \Rightarrow x \not \in B$, which contradicts that $x \in B$.
\end{proof}
\bigbreak

\textbf{Problem 6.7}
\begin{prop}
    For subsets of a universal set $U$, $A \subseteq B$ if and only if $B^c \subseteq A^c$.
\end{prop}
\begin{proof}
    $A \subseteq B$ means that for an arbitrary $x$, if $x \in A$, then $x \in B$. 
    Logically, it is equivalent to its contrapositive, which states for an arbitrary $x$, if $x \not \in B$, then $x \not \in A$.
    $x \not \in B$ can be written as $x \in B^c$, and $x \not \in A$ can be written as $x \in A^c$.
    Therefore, the entire statement can be rewritten as for an arbitrary $x$, if $x \in B^c$, then $x \in A^c$, which is the definition of $B^c \subseteq A^c$.
    Hence, $A \subseteq B \Leftrightarrow B^c \subseteq A^c$.
\end{proof}
\bigbreak

\textbf{Problem 7.1}
\begin{enumerate}[label={(\roman*)}]
    \item $\mathbb{Z^+}$. Let $n =m$, $n, m \in \mathbb{Z^+}$ and $m \le n$.
    \item $\{1\}$. It is apparent that $\forall n \in \mathbb{Z^+}, 1 \le n$. For $m \neq 1$, $n = 1$ is a counterexample to $\forall n \in \mathbb{Z^+}, m \le n$.
    \item $\mathbb{Z^+}$. Let $n =m$, $n, m \in \mathbb{Z^+}$ and $m \le n$.
    \item $\emptyset$. Let $m=n+1$, $\forall m \in \mathbb{Z^+}, m \not\le n$.
\end{enumerate}
\bigbreak

\textbf{Problem 7.2}
\begin{enumerate}[label={(\roman*)}]
    \item 
    \begin{prop}
        Disproving $\forall m, n \in \mathbb{Z^+}, m \le n$ means proving $\exists m, n \in \mathbb{Z^+}, m >n$.
    \end{prop}
    \begin{proof}
        Let $m=3$ and $n=2$, $m>n$.
    \end{proof}

    \item 
    \begin{prop}
        $\exists m,n \in \mathbb{Z^+}, m \le n$.
    \end{prop}
    \begin{proof}
        Let $m=2$ and $n=3$, $m \le n$.
    \end{proof}

    \item 
    \begin{prop}
        $\forall m \in \mathbb{Z^+}, \exists n \in \mathbb{Z^+}, m \le n$.
    \end{prop}
    \begin{proof}
        Let $n=m$. $\forall m \in \mathbb{Z^+}, m=n \Rightarrow m \le n$.
    \end{proof}

    \item 
    \begin{prop}
        $\exists m \in \mathbb{Z^+}, \forall n \in \mathbb{Z^+},  m \le n$.
    \end{prop}
    \begin{proof}
        Let $m = 1$. $\forall n \in \mathbb{Z^+}, m \le n $.
    \end{proof}

    \item 
    \begin{prop}
        $\forall n \in \mathbb{Z^+}, \exists m \in \mathbb{Z^+}, m \le n$.
    \end{prop}
    \begin{proof}
        Let $m = 1$. $\forall n \in \mathbb{Z^+}, m \le n$.
    \end{proof}

    \item 
    \begin{prop}
        Disproving $\exists n \in \mathbb{Z^+}, \forall m \in \mathbb{Z^+}, m \le n$ means proving $\forall n \in \mathbb{Z^+}, \exists m \in \mathbb{Z^+}, m > n$.
    \end{prop}
    \begin{proof}
        Let $m = n+1$. $\forall n \in \mathbb{Z^+}, m > n$.
    \end{proof}
\end{enumerate}
\bigbreak

\textbf{Problem 7.4}
\begin{enumerate}[label={(\roman*)}]
    \item 
    \begin{prop}
        $\forall x \in \mathbb{R}, \exists y \in \mathbb{R}, x + y = 0$.
    \end{prop}
    \begin{proof}
        Let $y = -x$. $\forall x\in \mathbb{R}, x + y = x - x = 0$.
    \end{proof}

    \item 
    \begin{prop}
        Disproving $\exists y \in \mathbb{R}, \forall x \in \mathbb{R}, x + y=0$ mean proving $\forall y \in \mathbb{R}, \exists x \in \mathbb{R}, x + y \neq 0$.
    \end{prop}
    \begin{proof}
        Let $x = -y + 1$. $\forall y \in \mathbb{R}, y + x = y - y + 1 = 1 \neq 0$.
    \end{proof}

    \item 
    \begin{prop}
        $\forall x \in \mathbb{R}, \exists y \in \mathbb{R}, xy=0$.
    \end{prop}
    \begin{proof}
        Let $y = 0$. $\forall x \in \mathbb{R}, xy = x \cdot 0 = 0$.
    \end{proof}

    \item 
    \begin{prop}
        $\exists y \in \mathbb{R}, \forall x \in \mathbb{R}, xy = 0$.
    \end{prop}
    \begin{proof}
        Let $y = 0$. $\forall x \in \mathbb{R}, xy = x \cdot 0 = 0$.
    \end{proof}

    \item 
    \begin{prop}
        Disproving $\forall x \in \mathbb{R}, \exists y \in \mathbb{R}, xy = 1$ means proving $\exists x \in \mathbb{R}, \forall y \in \mathbb{R}, xy \neq 1$.
    \end{prop}
    \begin{proof}
        Let $x = 0$. $\forall y \in \mathbb{R}, xy = 0 \cdot y = 0 \neq 1$.
    \end{proof}

    \item 
    \begin{prop}
        Disproving $\exists y \in \mathbb{R}, \forall x \in \mathbb{R}, xy =1$ means proving $\forall y \in \mathbb{R}, \exists x \in \mathbb{R}, xy \neq 1$.
    \end{prop}
    \begin{proof}
        Let $x = 0$. $\forall y \in \mathbb{R}, xy = 0 \cdot y = 0 \neq 1$.
    \end{proof}

    \item 
    \begin{prop}
        $\forall n \in \mathbb{Z^+}$, ($n$ is even or $n$ is odd).
    \end{prop}
    \begin{proof}
        $\forall n \in \mathbb{Z^+}$, $n$ is either even or $n$ is not even. 
        By definition, if $n$ is not even, then $n$ is odd, which logically means $n$ ie even or $n$ is odd.
    \end{proof}

    \item 
    \begin{prop}
        Disproving $(\forall n \in \mathbb{Z^+}, n \text{ is even})$ or $(\forall n \in \mathbb{Z^+}, n \text{ is odd})$ means proving $(\exists n \in \mathbb{Z^+}, n \text{ is odd})$ and $(\exists n \in \mathbb{Z^+}, n \text{ is even})$.
    \end{prop}
    \begin{proof}
        For the first half of the statement, let $n = 1$, then $n$ is odd, which proves $(\forall n \in \mathbb{Z^+}, n \text{ is odd})$. 
        For the second half of the statement, let $n = 2$, then $n$ is even, which proves $(\exists n \in \mathbb{Z^+}, n \text{ is even})$.
    \end{proof}
\end{enumerate}
\bigbreak

\textbf{Problem 7.7}
\begin{prop}
    For sets $A, B, C, D$, $(A \times B) \cup (C \times D) \subseteq (A \cup C) \times (B \cup D)$.
\end{prop}
\begin{proof}
    Let $(x, y) \in (A \times B) \cup (C \times D)$. It means $(x,y) \in (A \times B)$ or $(x,y) \in (C \times D)$. 
    If $(x,y) \in (A \times B)$, then indeed $x \in (A \cup C)$ and $y \in (B \cup D)$.
    If $(x,y) \in (C \times D)$, then again $x \in (A \cup C)$ and $y \in (B \cup D)$.
    Hence, $\forall (x,y) \in (A \times B) \cup (C \times D), (x,y) \in (A \cup C) \times (B \cup D)$.

    For the counterexample, let $A=\{1\},B=\{2\},C=\{3\},D=\{4\}$. 
    $(A \times B) \cup (C \times D) = \{(1,2),(3,4)\}$ while $(A \cup C) \times (B \cup D) = \{(1,2),(1,4),(3,2),(3,4)\}$.
\end{proof}
\bigbreak

\textbf{Page 115 Problem 4}
\begin{prop}
    $A \cap B = A \cap C$ and $A \cup B = A \cup C$ if and only if $B = C$.
\end{prop}
\begin{proof}
    $(\Rightarrow)$ 
    \begin{align}
        B & = B \cap (A \cup B) \;\;\;\;\; (\because B \subseteq A \cup B)\\
        & = B \cap (A \cup C) \;\;\;\;\; (\because A \cup B = A \cup C) \\
        & = (B \cap A) \cup (B \cap C) \\
        & = (A \cap B) \cup (B \cap C) \\
        & = (A \cap C) \cup (B \cap C) \;\;\;\;\; (\because A \cap B = A \cap C)\\
        & = (A \cup B) \cap C \\
        & = (A \cup C) \cap C \;\;\;\;\; (\because A \cup B = A \cup C) \\
        & = C \;\;\;\;\; (\because C \subseteq A \cup C)
    \end{align}
    Hence, $B = C$.

    $(\Leftarrow)$ This is apparent because we only need to substitute $B$ with $C$, then we will get $A \cap B = A \cap C$ and $A \cup B = A \cup C$.
\end{proof}
\bigbreak

\textbf{Page 117 Problem 13}
\begin{enumerate}[label={(\roman*)}]
    \item 
    \begin{prop}
        $A \times (B \cup C) = (A \times B) \cup (A \times C)$.
    \end{prop}
    \begin{proof}
        \begin{align}
            (x,y) \in A \times (B \cup C) & \Leftrightarrow x \in A \text{ and } y \in (B \cup C) \\
            & \Leftrightarrow x \in A \text{ and } (y \in B \text{ or } y \in C) \\
            & \Leftrightarrow (x \in A \text{ and } y \in B) \text{ or } (x \in A \text{ and } y \in C) \\
            & \Leftrightarrow (x,y) \in (A \times B) \cup (A \times C).
        \end{align}
    \end{proof}

    \item 
    \begin{prop}
        $(A \times B) \cap (C \times D) = (A \cap C) \times (B \cap D)$.
    \end{prop}
    \begin{proof}
        \begin{align}
            (x,y) \in (A\times B) \cap (C\times D) & \Leftrightarrow (x,y) \in (A\times B) \text{ and } (x,y) \in (C\times D) \\
            & \Leftrightarrow x \in A \text{ and } y \in B \text{ and } x\in C \text{ and } y\in D \\
            & \Leftrightarrow (x \in A \text{ and } x\in C) \text{ and } (y \in B \text{ and } y\in D) \\
            & \Leftrightarrow (x,y) \in (A \cap C) \times (B \cap D)
        \end{align}
    \end{proof}
\end{enumerate}
\bigbreak

\end{document}