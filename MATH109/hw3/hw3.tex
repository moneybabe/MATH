\documentclass{article}
\usepackage{amsfonts, amsmath, amssymb, amsthm} % Math notations imported
\usepackage{enumitem}
\usepackage[margin=1in]{geometry}

\newtheorem{thm}{Theorem}
\newtheorem{prop}[thm]{Proposition}
\newtheorem{cor}[thm]{Corollary}

% title information
\title{Math 109 HW3}
\author{Neo Lee}
\date{02/14/2023}

% main content
\begin{document} 

% placing title information; comment out if using fancyhdr
\maketitle 

\textbf{Problem 6.5}
\begin{enumerate}[label={(\roman*)}]
    \item 
    \begin{prop}
        $A \subseteq B \Leftrightarrow A \cup B = B$.
    \end{prop}
    \begin{proof}
        $(\Rightarrow; A \cup B \subseteq B)$ $\forall x \in A \cup B, x \in B$ because $A \subseteq B$.

        $(\Rightarrow; B \subseteq A \cup B)$ By definition, $\forall y \in B, y \in B \cup S$ for any arbitrary set $S$. Therefore, $B \subseteq A \cup B$.
        
        Since $A \cup B \subseteq B$ and $B \subseteq A \cup B$, $A \cup B = B$, and $(\Rightarrow)$ is proved.

        $(\Leftarrow)$ By definition, $\forall z \in A, z \in A \cup S$ for any arbitrary set $S$, which means $A \subseteq A \cup S$.
        Hence, $A \subseteq A \cup B$, which is equivalent to $A \subseteq B$.
    \end{proof}

    \item 
    \begin{prop}
        $A \subseteq B \Leftrightarrow A \cap B = A$.
    \end{prop}
    \begin{proof}
        $(\Rightarrow; A \cap B \subseteq A)$ By definition, $\forall x \in A \cap B, x \in A$, thus $A \cap B \subseteq A$.

        $(\Rightarrow; A \subseteq A \cap B)$ $\forall y \in A, x \in A \cap B$ because $A \subseteq B$.

        Since $A \cap B \subseteq A$ and $A \subseteq A \cap B$, $A \cap B = A$, and $(\Rightarrow)$ is proved.

        $(\Leftarrow)$ By definition, $(B \cap S) \subseteq B$ for any arbitrary set $S$. 
        Hence, $A = A \cap B \subseteq B$.
    \end{proof}
\end{enumerate}
\bigbreak

\textbf{Problem 6.6}
\begin{prop}
    If $A \cap B \subseteq C$ and $x \in B$, then $x \not\in A-C$.
\end{prop}
\begin{proof}
    Assume to the contrary that if $A \cap B \subseteq C$ and $x \in B$, then $x \in A-C$. 
    It means that $x \in A$ and $x \not \in C$. Since $A \cap B \subseteq C$, $x \not \in C \Rightarrow x \not \in A \cap B$.
    We know $x \in A$ and $x \not \in A \cap B$, therefore, $x \in A \cap B^c$. 
    It means $x \in B^c \Rightarrow x \not \in B$, which contradicts that $x \in B$.
\end{proof}
\bigbreak

\textbf{Problem 6.7}
\begin{prop}
    For subsets of a universal set $U$, $A \subseteq B$ if and only if $B^c \subseteq A^c$.
\end{prop}
\begin{proof}
    $A \subseteq B$ means that for an arbitrary $x$, if $x \in A$, then $x \in B$. 
    Logically, it is equivalent to its contrapositive, which states for an arbitrary $x$, if $x \not \in B$, then $x \not \in A$.
    $x \not \in B$ can be written as $x \in B^c$, and $x \not \in A$ can be written as $x \in A^c$.
    Therefore, the entire statement can be rewritten as for an arbitrary $x$, if $x \in B^c$, then $x \in A^c$, which is the definition of $B^c \subseteq A^c$.
\end{proof}
\bigbreak

\textbf{Problem 7.1}
\begin{enumerate}[label={(\roman*)}]
    \item $m = \mathbb{Z^+}$
    \item $m = \{1\}$
    \item $m = \mathbb{Z^+}$
    \item $n = \emptyset$
\end{enumerate}
\bigbreak
\textbf{Problem 7.2}
\bigbreak
\textbf{Problem 7.4}
\bigbreak
\textbf{Problem 7.7}
\bigbreak
\textbf{Page 115 Problem 4}
\bigbreak
\textbf{Page 117 Problem 13}
\bigbreak

\end{document}