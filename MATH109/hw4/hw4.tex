\documentclass{article}
\usepackage{amsfonts, amsmath, amssymb, amsthm} % Math notations imported
\usepackage{enumitem}
\usepackage[margin=1in]{geometry}

\newtheorem{thm}{Theorem}
\newtheorem{prop}[thm]{Proposition}
\newtheorem{cor}[thm]{Corollary}

% title information
\title{Math 109 HW4}
\author{Neo Lee}
\date{03/01/2023}

% main content
\begin{document} 

% placing title information; comment out if using fancyhdr
\maketitle 
\textbf{Problem 8.1}
\begin{prop}
    $g(x,y)=\begin{cases}
        x & if x \ge y \\
        y & if x \le y
    \end{cases}$ is well defined for $g:\mathbb{R}^2\rightarrow\mathbb{R}$.
\end{prop}
\begin{proof}
    For all $(x,y) \in \mathbb{R}^2$, it is exclusively that $x>y$, $x<y$, or $x=y$. 
    If $x>y$, $g(x,y)$ is uniquely defined as $x \in \mathbb{R}$.
    If $x<y$, $g(x,y)$ is uniquely defined as $y \in \mathbb{R}$.
    If $x=y$, $g(x,y)$ is uniquely defined as $x=y \in \mathbb{R}$.
\end{proof}
\begin{prop}
    Let $f(x,y)=\frac{x+y}{2}+\frac{|x-y|}{2}$ for $f:\mathbb{R}^2\rightarrow\mathbb{R}$, $f=g$.
\end{prop}
\begin{proof}
    If $x>y$, $f(x,y)=\frac{x+y}{2}+\frac{x-y}{2}=x$.
    If $x<y$, $f(x,y)=\frac{x+y}{2}+\frac{y-x}{2}=y$.
    If $x=y$, $f(x,y)=\frac{x+x}{2}+\frac{x-x}{2}=x=y$.
    Hence, $f(x,y)=g(x,y)$ for all $(x,y)\in \mathbb{R}^2$.
\end{proof}
\bigbreak

\textbf{Problem 8.2}
\begin{enumerate}[label={(\roman*)}]
    \item $f\circ f=f(f(x))=f(x^3)=x^{3^3}=x^9$ for $\mathbb{R}\rightarrow\mathbb{R}$.
    \item $f\circ g=f(g(x))=f(1-x)=(1-x)^3$ for $\mathbb{R}\rightarrow\mathbb{R}$.
    \item $g\circ f=g(f(x))=g(x^3)=1-x^3$ for $\mathbb{R}\rightarrow\mathbb{R}$.
    \item $g\circ g=g(g(x))=g(1-x)=1-(1-x)=x$ for $\mathbb{R}\rightarrow\mathbb{R}$.
\end{enumerate}

$fg(x)=gf(x)\Leftrightarrow(1-x^3)=1-x^3\Leftrightarrow 1-3x+3x^2-x^3=1-x^3\Leftrightarrow x(x-1)=0\Leftrightarrow x=0$ or $x=1$. 
Hence, $\{x\in\mathbb{R}|fg(x)=gf(x)\}=\{0,1\}$.
\bigbreak

\textbf{Problem 8.3}
\begin{enumerate}[label={(\roman*)}]
    \item $f_1(x)=x$ for $\mathbb{R}\rightarrow\mathbb{R}$.
    \item $f_2(x)=|x|$  for $\mathbb{R}\rightarrow\mathbb{R}$.
    \item $f_3(x)=\begin{cases}
        x & if x\not\in\mathbb{Z} \\
        0.1 & if x\in\mathbb{Z}
    \end{cases}$ for $\mathbb{R}\rightarrow\mathbb{R}$.
    \item $f_4(x)=\lfloor x\rfloor$ for $\mathbb{R}\rightarrow\mathbb{R}$.
\end{enumerate}
\bigbreak

\textbf{Problem 8.5}
$(i)$ and $(iv)$ are graphs of a function $f:X\rightarrow Y$.
\begin{center}
    \begin{tabular}{ |c|c c| } 
     \hline
     $x$ & $f_i(x)$ & $f_{iv}(x)$  \\ 
     \hline
     $a$ & $z$ & $y$ \\ 
     $b$ & $y$ & $z$ \\ 
     $c$ & $z$ & $w$ \\ 
     $d$ & $x$ & $x$ \\ 
     \hline
    \end{tabular}
\end{center}

For $(ii)$, $\{c\}\times Y$ contains no elements, which means not every element in $X$ is mapped to $Y$. 
For $(iii)$, $\{b\}\times Y$ contains more than one element, which mean $f(x)$ is not uniquely defined in $Y$ for $x=b$.
\bigbreak

\textbf{Problem 9.1}
\begin{enumerate}[label={(\roman*)}]
    \item Bijective. It is surjective because for every image $y$, there is a pre-image $x=\frac{y-5}{2}$. It is injective because if $y=f(x_1)=f(x_2)$, $x_1=x_2=\frac{y-5}{2}$.
    \item Neither injective nor surjective. Let $f(x)=1$, $x=-2$ or $x=0$, thus it's not injective. Since there does not exists $x$ for $f(x)=-1$, it's not surjective.
    \item Neither injective nor surjective. Let $f(x)=0$, $x=0$ or $x=2$, thus it's not injective. Since there does not exists $x$ for $f(x)=-2$, it's not surjective.
    \item Bijective. It is surjective because for every image $y\neq 0$, there is a pre-image $x=\frac{1}{y}$; for $y=0$, $x=0$. It is injective becasue if $0\neq y =f(x_1)=f(x_2)$, $x_1=x_2=\frac{1}{y}$; if $y=0$, $x=0$.
\end{enumerate}
\bigbreak

\textbf{Problem 9.2}
\begin{enumerate}[label={(\roman*)}]
    \item Injective only. It is injective because if $y=f(x_1)=f(x_2)$, $x_1=x_2=\frac{y-5}{2}$. It is not surjective because there does not exist $x$ for $f(x) = 1$.
    \item Injective only. It is injective because if $y=f(x_1)=f(x_2)$, $x_1=x_2=-1+\frac{\sqrt{y}}{2}$. It is not surjective because there does not exist $x$ for $f(x) = 0.1$.
    \item Not a function. There does not exist $f(x) \in \mathbb{R}^+$ for $x=0.1$.
    \item Bijective. It is surjective because for every image $y$, there is a pre-image $x=\frac{1}{y}$. It is injective becasue if $y =f(x_1)=f(x_2)$, $x_1=x_2=\frac{1}{y}$.
\end{enumerate}
\bigbreak

\textbf{Problem 9.3}
\begin{enumerate}[label={(\roman*)}]
    \item $f^{-1}(y)=\frac{y-2}{3}$.
    \item $f^{-1}(y)=\sqrt[3]{y-1}$.
\end{enumerate}
\bigbreak

\textbf{Problem 9.4}
\begin{prop}
    $g\circ f$ is injective if $g$ and $f$ are both injective.
\end{prop}
\begin{proof}
    \begin{align}
        z = g(f(x_1)) = g(f(x_1)) & \Rightarrow f(x_1) = f(x_2) \;\;\;\;\; \because g \text{ is injective} \\
        & \Rightarrow x_1 = x_2 \;\;\;\;\; \because f \text{ is injective}
    \end{align}
    Hence, $g\circ f(x_1)=g\circ f(x_2)\Rightarrow x_1 = x_2$.
\end{proof}
\bigbreak

\textbf{Problem 9.6}
\begin{prop}
    Let $f:X\rightarrow Y$ be a function with graph $G_f \subseteq X\times Y$. $f$ is surjective if and only if $\forall y \in Y, (X\times \{y\}\cap G_f)\neq \emptyset$.
\end{prop}
\begin{proof}
    $(\Rightarrow)$ Since $f$ is surjective, $\forall y \in Y$, $\exists x$ such that $f(x)=y$. 
    Let $x_0 \in X$ such that $f(x_0)=y_0$ for arbitrary $y_0\in Y$.
    Then $(x_0,y_0) \in (X \times \{y_0\}\cap G_f)$. Hence, for all $y\in Y$, $(X \times \{y\}\cap G_f)\neq \emptyset$.
    
    $(\Leftarrow)$ Since $\forall y \in Y, (X\times \{y\}\cap G_f)\neq \emptyset$, we can take an arbitrary $y_1 \in Y$ and there must exist $(x_1,y_1) \in X\times \{y_1\}$.
    At the same time $(x_1,y_1)\in G_f$, so we know that $f(x_1)=y_1$. 
    Hence, it satisfies that definition of surjection that $\forall y \in Y, \exists x$ such that $f(x)=y$.
\end{proof}
\bigbreak

\textbf{Problem 14}
$f\circ f = x \mapsto x^4$. $f\circ g = x \mapsto x^4 -2x^2 + 1$. $g\circ f=x \mapsto x^4-1$. $g\circ g = x \mapsto x^4-2x^2$. \\
$\{x \in \mathbb{R}|fg(x)=gf(x)\} \Leftrightarrow x^4-2x^2+1=x^4-2x^2 \Leftrightarrow \emptyset$.
\bigbreak

\textbf{Problem 15}
\begin{enumerate}[label={(\roman*)}]
    \item 
    We can easily see that $\chi_A(x)\chi_B(x) \equiv \chi_{A\cap B}(x)$ by drawing a truth table.
    \begin{displaymath}
        \begin{array}{|c|c|c|c|}
        x\in A & x\in B & \chi_A(x)\chi_B(x) & \chi_{A\cap B}(x)\\ 
        \hline  
        T & T & 1 & 1 \\
        T & F & 0 & 0 \\
        F & T & 0 & 0 \\
        F & F & 0 & 0 \\
        \end{array}
    \end{displaymath}

    \item 
    Let $C = A \cup B$.
    \begin{displaymath}
        \begin{array}{|c|c|c|c|}
        x\in A & x\in B & \chi_A(x)+\chi_B(x)-\chi_A(x)\chi_B(x) & \chi_C(x)\\ 
        \hline  
        T & T & 1 & 1 \\
        T & F & 1 & 1 \\
        F & T & 1 & 1 \\
        F & F & 0 & 0 \\
        \end{array}
    \end{displaymath}
\end{enumerate}
\bigbreak

\textbf{Problem 16}
\begin{enumerate}[label={(\roman*)}]
    \item Bijective. Surjective: $\forall y=f_1(x), \exists x = y +1\in \mathbb{R}$. Injective: $y_0=f_1(x_1)=f_1(x_2)\Rightarrow x_1=x_2=y_0+1$. $f_1^{-1}(y)=y+1$.
    \item Bijective. Surjective: $\forall y=f_2(x), \exists x = \sqrt[3]{y}\in \mathbb{R}$. Injective: $y_0=f_2(x_1)=f_2(x_2)\Rightarrow x_1=x_2=\sqrt[3]{y}$. $f_2^{-1}(y)=\sqrt[3]{y}$.
    \item Surjective. Surjective: $\lim\limits_{x\rightarrow \infty}f_3(x)=\infty$ and $\lim\limits_{x\rightarrow -\infty}f_3(x)=-\infty$. Since $f_3(x)$ is a polynomial, it is a continuous function. By intermediate value theorem, $\forall y  \in (-\infty,\infty)\equiv \mathbb{R}$, $\exists x$ such that $y=f_3(x)$. Not injective: let $f_3(x)=0$, $x=-1$ or $x=0$ or $x=1$. 
    \item Bijective. Surjective: $\forall y=f_4(x), \exists x$ such that $x^3-3x^2+3x-1=y \Leftrightarrow (x-1)^3=y \Leftrightarrow x=\sqrt[3]{y}+1$. Injective: let $y=f_4(x_1)=f_4(x_2)\Rightarrow x_1=x_2=\sqrt[3]{y}+1$. $f_4^{-1}(y)=\sqrt[3]{y}+1$. 
    \item Bijective. Surjective: $\forall y = f_5(x), \exists x = ln(y)$. Injective: $y_0=f_5(x_1)=f_5(x_2)\Rightarrow x_1=x_2=ln(y_0)$. $f_5^{-1}(y)=ln(y)$. 
    \item Surjective. Surjective: $\forall y = f_6(x) > 0, \exists x=\sqrt{y}$; $\forall y=f_6(x)<0,\exists x=\sqrt{-y};y=0,x=0$. Not injective: let $f_6(x)=4$, $x=2$ or $x=-2$.
\end{enumerate}

\end{document}