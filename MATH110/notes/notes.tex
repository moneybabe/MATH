\documentclass{book}
\input{preamble.tex}
\newcommand*{\Var}{\ensuremath{\mathrm{Var}}}
\newcommand*{\Cov}{\ensuremath{\mathrm{Cov}}}
\newcommand*{\Corr}{\ensuremath{\mathrm{Corr}}}
\newcommand*{\Bias}{\ensuremath{\mathrm{Bias}}}
\newcommand*{\MSE}{\ensuremath{\mathrm{MSE}}}

\newcommand*{\range}{\ensuremath{\mathrm{range}}\,}
\newcommand*{\spann}{\ensuremath{\mathrm{span}}\,}
\newcommand*{\nul}{\ensuremath{\mathrm{null}}\,}
\newcommand*{\dom}{\ensuremath{\mathrm{dom}}\,}

\newcommand*{\pdv}[3][]{\frac{\partial^{#1}}{\partial#3^{#1}}#2}
\renewcommand*{\implies}{\ensuremath{\Longrightarrow}}
\renewcommand*{\impliedby}{\ensuremath{\Longleftarrow}}
\renewcommand*{\l}{\left}
\renewcommand*{\r}{\right}
\renewcommand{\leq}{\leqslant}
\renewcommand{\geq}{\geqslant}
\newcommand*{\tb}{\textbf}
\renewcommand*{\t}{\text}

\newcommand*{\Z}{\ensuremath{\mathbb{Z}}}
\newcommand*{\Q}{\ensuremath{\mathbb{Q}}}
\newcommand*{\R}{\ensuremath{\mathbb{R}}}
\newcommand*{\F}{\ensuremath{\mathbb{F}}}
\newcommand*{\C}{\ensuremath{\mathbb{C}}}
\newcommand*{\N}{\ensuremath{\mathbb{N}}}
\newcommand*{\E}{\ensuremath{\mathds{E}}}
\renewcommand*{\P}{\ensuremath{\mathds{P}}}
\newcommand*{\p}{\ensuremath{\mathcal{P}}}
\renewcommand*{\L}{\mathcal{L}}


% deliminators
\DeclarePairedDelimiter{\abs}{\lvert}{\rvert}
\DeclarePairedDelimiter{\norm}{\|}{\|}
\DeclarePairedDelimiter{\inner}{\langle}{\rangle}
\DeclarePairedDelimiter{\ceil}{\lceil}{\rceil}
\DeclarePairedDelimiter{\floor}{\lfloor}{\rfloor}
\DeclarePairedDelimiter{\round}{\lfloor}{\rceil}

\let\oldleq\leq % save them in case they're every wanted
\let\oldgeq\geq

\newcommand*{\img}[3][]{
    \begin{figure}[htb!]
         \centering
         \includegraphics[scale=#1]{#2}
         \caption{#3}
    \end{figure}
}
\newcommand*{\imgs}[5][]{
    \begin{figure}[htb]
        \qquad
        \begin{minipage}{.4\textwidth}
            \centering
            \includegraphics[scale=#1]{#2}
            \caption{#3}
        \end{minipage}    
        \qquad
        \begin{minipage}{.4\textwidth}
            \centering
            \includegraphics[scale=#1]{#4}
            \caption{#5}
        \end{minipage}
    \end{figure} 
}


\title{\Huge{MATH 110 Notes}\\\emph{Book: Linear Algebra Done Right}}
\author{\huge{Neo Lee}}
\date{Fall 2023}

\begin{document}

\maketitle
\let\cleardoublepage\clearpage
\pdfbookmark[section]{\contentsname}{toc}
\tableofcontents

\chapter{Isometry}
\section{November 30}
\dfn{Isometry and Unitary}{
    $T\in\L(V,w)$ is called an isometry if $\norm{T\vec{v}}=\norm{\vec{v}}$ for 
    all $\vec{v}\in V$.
    \nt{
        In case $T\in\L(V)$ is an \emph{invertible} isometry, it is called \emph{unitary}.
    }
}
\nt{
    If $T\in\L(v)$ is unitary, then any eigen-pair $(\lambda,\vec{v})$ of $T$ satifies 
    $$\norm{T\vec{v}}=\norm{\lambda\vec{v}}=\abs{\lambda}\norm{\vec{v}}=\norm{\vec{v}}.$$
    So $\abs{\lambda}=1$.
}
\mprop{Isometry only has eigenvalues of modulus 1.}{
    If $T$ is an isometry, then 
    $$\norm{T\vec{e_j}}=\norm{\vec{e_j}}=1,$$
    but also $$\norm{T\vec{e_j}}=\norm{s_j\cdot \vec{f_j}}=s_j=1.$$
}
\mprop{Isometry is injective.}{
    \pf{Proof}{
        Recall $\nul T^*T=\nul T$. So if $\vec{v}\in\nul T\implies T\vec{v}=\vec{0}$, then
        $$\norm{\vec{v}}=\norm{T\vec{v}}=\norm{\vec{0}}=0\implies \vec{v}=\vec{0}.$$

        This also shows isometries do not have zero singular values.
    }
}
\dfn{Jordan Normal Form}{
    \nt{
        The goal of Jordan Normal Form is to get a matrix representation of $T$ that is as
        simple (diagonal) as possible. This way, we get sparse matrices that are easy to
        compute with or understand.
    }

    Setting: Let $V$ be a finite-dimensional vector space over $\C$ and $T\in\L(V)$.

    First, represent $V=U\oplus W$ where $U,W$ are invariant subspaces of $T$, and 
    $T$ is nilpotent on $U$ and invertible on $W$. Here nilpotent on $U$ means 
    $T^k\vec{v}=\vec{0}$ for all $\vec{v}\in U$ and some $k\in\N$. We already get a 
    block diagonal matrix representation of $T$ when writing $\vec{v}=\vec{u}+\vec{w}$ for 
    $\vec{v}\in V$, $\vec{u}\in U$, and $\vec{w}\in W$.

    $\{0\}\subseteq\nul T\subseteq\nul T^2\subseteq\cdots$ is a chain of subspaces of $V$. 
    Since $V$ is finite-dimensional, there exists $k\in\N$ such that $\nul T^k=\nul T^{k+1}$. 
    In other words, $\nul T^k$ will stabilize at some point.

    Take $U=\nul T^k$ and $W=\range T^k$ where $\nul T^k$ stabilize. Notice, by the rank-nullity 
    theorem, $\dim U+\dim W=\dim V$. If $\vec{x}\in U\cap W$, i.e., $\vec{x}=T^k\vec{v}$ 
    for some $\vec{v}\in V$, then $T^k\vec{x}=T^{2k}\vec{v}=\vec{0}$, so $\vec{v}\in \nul T^{2k}=
    \nul T^k$. Thus, $\vec{x}=T^k\vec{v}=\vec{0}$.

    Within the subspace $U$, $T^k=0$. Witout loss of generality, $U$ is our whole space. 
    Then there exists $\vec{u_0}\in U$ such that $T^{k-1}\vec{u_0}\neq 0$ but 
    $T^k\vec{u_0}=0$. Then, there exists $v_0\in V$ such that $\inner{T^{k-1}\vec{v_0},v_0}\neq 0$.
    Consider the matrix 
    $$\inner{T^{j-1}\vec{u_0}, T^{*^{k-i}}\vec{v_0}}\text{ for }i, j=1,2,\dots, k,$$
    which is an invertible matrix. 
    
    This implies that $\spann(\vec{u_0}, T\vec{u_0},\dots, T^{k-1}\vec{u_0})\oplus 
    \spann(\vec{v_0}, T^*\vec{v_0},\dots, T^{*^{k-1}}\vec{v_0})^\perp=U$. Indeed, if $\vec{v}$ 
    is in the intersection of the two subspaces, this creates a linear system precisely with the 
    matrix above, which is invertible. Thus, $\vec{v}=\vec{0}$.

    Then, represent $U$ as a direct sum of generalized eigenspaces of $T$.

}

\end{document}
