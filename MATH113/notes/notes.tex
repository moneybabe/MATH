\documentclass{book}
\input{preamble.tex}
\newcommand*{\Var}{\ensuremath{\mathrm{Var}}}
\newcommand*{\Cov}{\ensuremath{\mathrm{Cov}}}
\newcommand*{\Corr}{\ensuremath{\mathrm{Corr}}}
\newcommand*{\Bias}{\ensuremath{\mathrm{Bias}}}
\newcommand*{\MSE}{\ensuremath{\mathrm{MSE}}}

\newcommand*{\range}{\ensuremath{\mathrm{range}}\,}
\newcommand*{\spann}{\ensuremath{\mathrm{span}}\,}
\newcommand*{\nul}{\ensuremath{\mathrm{null}}\,}
\newcommand*{\dom}{\ensuremath{\mathrm{dom}}\,}

\newcommand*{\pdv}[3][]{\frac{\partial^{#1}}{\partial#3^{#1}}#2}
\renewcommand*{\implies}{\ensuremath{\Longrightarrow}}
\renewcommand*{\impliedby}{\ensuremath{\Longleftarrow}}
\renewcommand*{\l}{\left}
\renewcommand*{\r}{\right}
\renewcommand{\leq}{\leqslant}
\renewcommand{\geq}{\geqslant}
\newcommand*{\tb}{\textbf}
\renewcommand*{\t}{\text}

\newcommand*{\Z}{\ensuremath{\mathbb{Z}}}
\newcommand*{\Q}{\ensuremath{\mathbb{Q}}}
\newcommand*{\R}{\ensuremath{\mathbb{R}}}
\newcommand*{\F}{\ensuremath{\mathbb{F}}}
\newcommand*{\C}{\ensuremath{\mathbb{C}}}
\newcommand*{\N}{\ensuremath{\mathbb{N}}}
\newcommand*{\E}{\ensuremath{\mathds{E}}}
\renewcommand*{\P}{\ensuremath{\mathds{P}}}
\newcommand*{\p}{\ensuremath{\mathcal{P}}}
\renewcommand*{\L}{\mathcal{L}}


% deliminators
\DeclarePairedDelimiter{\abs}{\lvert}{\rvert}
\DeclarePairedDelimiter{\norm}{\|}{\|}
\DeclarePairedDelimiter{\inner}{\langle}{\rangle}
\DeclarePairedDelimiter{\ceil}{\lceil}{\rceil}
\DeclarePairedDelimiter{\floor}{\lfloor}{\rfloor}
\DeclarePairedDelimiter{\round}{\lfloor}{\rceil}

\let\oldleq\leq % save them in case they're every wanted
\let\oldgeq\geq

\newcommand*{\img}[3][]{
    \begin{figure}[htb!]
         \centering
         \includegraphics[scale=#1]{#2}
         \caption{#3}
    \end{figure}
}
\newcommand*{\imgs}[5][]{
    \begin{figure}[htb]
        \qquad
        \begin{minipage}{.4\textwidth}
            \centering
            \includegraphics[scale=#1]{#2}
            \caption{#3}
        \end{minipage}    
        \qquad
        \begin{minipage}{.4\textwidth}
            \centering
            \includegraphics[scale=#1]{#4}
            \caption{#5}
        \end{minipage}
    \end{figure} 
}


\title{\Huge{MATH 113 Notes}\\\emph{Professor: Forte Shinko}}
\author{\huge{Neo Lee}}
\date{\huge{Spring 2024}}

\begin{document}

\maketitle
\let\cleardoublepage\clearpage
\pdfbookmark[section]{\contentsname}{toc}
\tableofcontents

\chapter{Motivation}
\section{Abstraction}
We have two very similar theorems, and turns out they can be generalized into one single property,
hence the abstraction. This is the idea of abstraction, and it is the core of modern mathematics. 

\thm{Prime factorization}{
	Every integer $n > 1$ can be uniquely factorized as a product of primes.
}
\thm{Fundamental theorem of algebra in \R}{
	Every polynomial $p(x)$ of degree $n > 0$ with real coefficients can be factorized into a
	product of linear and quadratic polynomials with real coefficients.
}

It turns out that these two theorems are very similar, and we can describe them with one single 
property, namely the unique factorization domain (UFD). 
\cor{Generalization as UFD}{
	\begin{enumerate}
		\item $\l(\Z, +,\cdot\r)$ is a UFD.
		\item $\l(\R[x], +,\cdot\r)$ is a UFD.
	\end{enumerate}
}

With abstraction, we can also prove theorems in a more general setting and sometimes apply
generalized theorems to specific cases to get an easier proof.
\thm{Fermat's little theorem}{
	If $p$ is a prime and $a \in \Z$, then $a^p \equiv a \pmod{p}$.
	\pf{Proof}{
		Apply Lagrange's theorem to the group $\l( \Z /p \Z , \cdot\r)$.
	}
	\nt{
		This proof using abstracted theorem is much easier than the original proof in traditional 
		number theory.
	}
}

\chapter{Group Theory}
\section{Lecture 1}
\dfn{Binary operation}{
	A binary operation on a set $S$ is a funtion from $S \times S$ to $S$. 
}
\ex{Addition on \Z}{
	We can define addition on $\Z$ as a binary operation, since for any $a, b \in \Z$, $a + b \in
	\Z$.
}
\wc{Non-examples of binary operation}{
	\begin{enumerate}
		\item 
		Subtraction on $\N$ is not a binary operation. Consider the case $a = 1, b = 2$, then $a - b 
		= -1 \notin \N$.
		
		\item
		Division on $\R$ is not a binary operation. Division by zero is not defined on $\R$. Hence, 
		division is not a binary operation from $\R \times \R$ to $\R$.
	\end{enumerate}
}

\section{Lecture 2}
Some examples of binary operations:
\ex{The midpoint operation on $\R^2$}{
	$$T(\vec{a},\vec{b})=\frac{1}{2}(\vec{a}+\vec{b}).$$
}
\ex{The set of functions from $\R$ to $\R$}{
	Some examples are $x^2, \sin x$. The set of functions from $\R$ to $\R$ has a binary operation of
	composition, taking $(f,g)$  to $f\circ g$.
	\nt{
		For convenience, we typically write composition by $fg = f\circ g$.
	}
}
\ex{Cross product}{
	Cross product is a binary operation on $\R^3$.
	\wc{Dot product is not a binary operation}{
		Dot product is not a binary operation on $\R^3$, since the result of dot product is a scalar, 
		which is not in $\R^3$. In particular \emph{dot prodcut}: $\R^3 \times \R^3 \to \R$.
	}
}
\dfn{Matrix of $n\times n$}{
	Denote $\M_n(\F)$ to be the set of $n\times n$ matrices over $\F$.
}
\ex{$\M_n(\F)$ has two common binary operations}{
	Namely addition and multiplication on matrices.
}
\ex{Power set of \R}{
	The power set of $\R$ has union and intersection as binary operations.
	\nt{
		$\{1,2,3\}\cup \{\pi, e\} = \{1,2,3,\pi,e\}$ is in the power set of \R.
	}
}
\dfn{Monus operation}{
	Consider the monus operation defined by $(x,y)\mapsto\max(x-y, 0)$.
}
\dfn{Multiplication table (aka Cayley table)}{
	Try to do the Cayley table on the set $\{0,1,2,3\}$ with the monus operation.
	\nt{
		You can also define a binary opeartion by simply listing out the Cayley table. Hence, a binary
		operation on $\{0,1,2,3\}$ is just any way to fill the $4\times 4$ grid. So binary operation is
		actually not that unique.
	}
}
\nt{
	In this class, we typically talk about sets equipped with binary operations. For example, we talk
	abou the set $\Z$ equipped with addition $(\Z, +)$ or even with multiplication as well $(\Z, +,
	\cdot)$.
}
\dfn{Monoid}{
	A monoid is a set $M$ equipped with a binary operation $\cdot$ such that
	\begin{enumerate}
		\item \tb{Associativity:} $\cdot$ is associative, i.e. $(a\cdot b)\cdot c = a\cdot (b\cdot
		c)$ for all $a,b,c\in M$.

		\item \tb{Existence of identity:} There exists an identity element $e\in M$ such that
		$e\cdot a = a\cdot e = a$ for all $a\in M$.
	\end{enumerate}
}
\dfn{Group}{
	A group is a set $G$ equipped with a binary operation $\cdot$ such that
	\begin{enumerate}
		\item \tb{Associativity:} $\cdot$ is associative, i.e. $(a\cdot b)\cdot c = a\cdot (b\cdot
		c)$ for all $a,b,c\in G$.

		\item \tb{Existence of identity:} There exists an identity element $e\in G$ such that
		$e\cdot a = a\cdot e = a$ for all $a\in G$.
		
		\item \tb{Existence of inverse:} For every $a\in G$, there exists an inverse $a^{-1}\in G$
		such that $a\cdot a^{-1} = a^{-1}\cdot a = e$.
	\end{enumerate}
	\nt{
		\begin{itemize}
			\item 
			We typically denote a group by $(G, \cdot)$, but sometimes we omit the binary operation
			$\cdot$ if it is clear from the context.

			\item
			Group is an extension of monoid, since every group is a monoid, but not every monoid is
			a group.
		\end{itemize}
	}
}







\chapter{Starting a new chapter}
\section{Demo of commands}
\dfn{Some defintion}{
	yap
}
\qs{Some question}{yap}
\sol{
	\pf{Some proof}{yap}
}
\nt{Some note}
\thm{Some theorem}{
	yap
}
\wc{Some wrong concept}{
	yap
}
\mlemma{Some lemma}{
	yap
}
\mprop{Some proposition}{
	yap
}
\ex{Some example}{
	yap
}
\clm{Some claim}{
	yap
}
\cor{Some corollary}{
	yap
}
\thmcon{Some unlabeled theorem}

This is a new paragraph


\begin{algorithm}
	\caption{Some algorithm}
	\KwIn{input}
	\KwOut{output}
	\SetAlgoLined
	\SetNoFillComment
	\tcc{This is a comment}
	This is first line \tcp*{This is also a comment}
	\uIf{$x > 5$} {
		do nothing
	} \uElseIf {$x < 5$} {
		do nothing
	} \Else {
		do nothing
	}
	\While{$x == 5$}{
		still do nothing
	}
	\ForEach{$x = 1:5$}{
		do nothing
	}
	\Return{return nothing}
\end{algorithm}


\end{document}
