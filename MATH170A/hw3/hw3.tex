\documentclass{article}
\usepackage{amsfonts, amsmath, amssymb, amsthm} % Math notations imported
\usepackage{enumitem}
\usepackage[margin=1in]{geometry}

\newtheorem{thm}{Theorem}
\newtheorem{prop}[thm]{Proposition}
\newtheorem{cor}[thm]{Corollary}

% title information
\title{Math 170A HW3}
\author{Neo Lee}
\date{04/21/2023}

% main content
\begin{document} 

% placing title information; comment out if using fancyhdr
\maketitle 

\textbf{Problem 4.}
$B = X^{^\top}AX = X^{^\top}R^{^\top}RX = (RX)^{^\top}(RX)$. 
Let $M = RX$, then $B = (RX)^{^\top}(RX) \Leftrightarrow B = M^{^\top}M$. 
Since $R, X$ are both invertible with determinant $\neq 0$, $M$ is also invertible with determinant $\neq 0$.

$B^{^\top} = (M^{^\top}M)^{^\top} = M^{^\top}M = B$. So $B$ is symmetric.

Then let $\vec{x}\neq \vec{0}$. $\vec{x}^{^\top}B\vec{x} = \vec{x}^{^\top}M^{^\top}M\vec{x} = (M\vec{x})^{^\top}M\vec{x} = M\vec{x} \cdot M\vec{x}$. 
Let $y = M\vec{x}$. 
Since $M$ is invertible and $\vec{x} \neq \vec{0}$, $\vec{y}\neq\vec{0}$.
Hence, $M\vec{x}\cdot M\vec{x} = \vec{y}\cdot \vec{y} > 0$.

Therefore, $B$ is positive definite.

\end{document}