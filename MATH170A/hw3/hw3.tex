\documentclass{article}
\usepackage{amsfonts, amsmath, amssymb, amsthm} % Math notations imported
\usepackage{enumitem}
\usepackage[margin=1in]{geometry}
\usepackage{graphicx}
\graphicspath{{./images/}}

\newtheorem{thm}{Theorem}
\newtheorem{prop}[thm]{Proposition}
\newtheorem{cor}[thm]{Corollary}

% title information
\title{Math 170A HW3}
\author{Neo Lee}
\date{04/21/2023}

% main content
\begin{document} 

% placing title information; comment out if using fancyhdr
\maketitle 
\textbf{Problem 1.}
\begin{figure}[h]
    \centering
    \includegraphics*[scale=0.5]{Gauss_solve.png}
    \caption{\emph{Gauss\_solve.m}}
\end{figure}
\bigbreak

\textbf{Problem 2.}
\begin{align}
    & \begin{bmatrix}
        2 & 2 & -4 \\
        1 & 1 & 5 \\
        1 & 3 & 6 
    \end{bmatrix} \nonumber 
    \Rightarrow (\emph{row 2 - $0.5\times$ row 1, row 3 - $0.5\times$ row 1}) \\ \Rightarrow
    & \begin{bmatrix}
        2 & 2 & -4 \\
        0 & 0 & 7 \\
        0 & 2 & 8 
    \end{bmatrix} \nonumber 
    \Rightarrow (\emph{row 2 and row 3 swap}) \\ \Rightarrow
    & \begin{bmatrix}
        2 & 2 & -4 \\
        0 & 2 & 8 \\
        0 & 0 & 7 
    \end{bmatrix} \nonumber \\ \Rightarrow
    & P = \begin{bmatrix}
        1 & 0 & 0 \\
        0 & 0 & 1 \\
        0 & 1 & 0
    \end{bmatrix}; \quad
    L = \begin{bmatrix}
        1 & 0 & 0 \\
        0.5 & 1 & 0 \\
        0.5 & 0 & 1
    \end{bmatrix}; \quad
    U = \begin{bmatrix}
        2 & 2 & -4 \\
        0 & 2 & 8 \\
        0 & 0 & 7
    \end{bmatrix} \nonumber
\end{align}
\bigbreak

\textbf{Problem 3.}
\begin{align}
    \begin{bmatrix}
        1 & 0 & 2 \\
        0 & 1 & -1 \\
        2 & -1 & 9
    \end{bmatrix} & =
    \begin{bmatrix}
        r_{1,1} & 0 & 0 \\
        r_{1,2} & r_{2,2} & 0 \\
        r_{1,3} & r_{2,3} & r_{3,3}
    \end{bmatrix}
    \begin{bmatrix}
        r_{1,1} & r_{1,2} & r_{1,3} \\
        0 & r_{2,2} & r_{2,3} \\
        0 & 0 & r_{3,3}
    \end{bmatrix} \nonumber \\
    & \Rightarrow r_{1,1} = \sqrt{1} = 1 \nonumber \\
    & \Rightarrow r_{1,1} \times r_{1,2} = 0 \Rightarrow r_{1,2} = 0 \nonumber \\
    & \Rightarrow r_{1,1} \times r_{1,3} = 2 \Rightarrow r_{1,3} = 2 \nonumber \\
    & \Rightarrow r_{1,2}^2 + r_{2,2}^2 = 1 \Rightarrow r_{2,2} = 1 \nonumber \\
    & \Rightarrow r_{1,2} \times r_{1,3} + r_{2,2} \times r_{2,3} = -1 \Rightarrow r_{2,3} = -1 \nonumber \\
    & \Rightarrow r_{1,3}^2 + r_{2,3}^2 + r_{3,3}^2 = 9 \Rightarrow r_{3,3} = 2 \nonumber \\
    & \Rightarrow R = \begin{bmatrix}
        1 & 0 & 2 \\
        0 & 1 & -1 \\
        0 & 0 & 2
    \end{bmatrix} \nonumber 
\end{align}
Since all the diagonal entires are greater than 0, $R$ is positive definite.
\bigbreak

\textbf{Problem 4.}
$B = X^{^\top}AX = X^{^\top}R^{^\top}RX = (RX)^{^\top}(RX)$. 
Let $M = RX$, then $B = (RX)^{^\top}(RX) \Leftrightarrow B = M^{^\top}M$. 
Since $R, X$ are both invertible with determinant $\neq 0$, $M$ is also invertible with determinant $\neq 0$.

$B^{^\top} = (M^{^\top}M)^{^\top} = M^{^\top}M = B$. So $B$ is symmetric.

Then let $\vec{x}\neq \vec{0}$. $\vec{x}^{^\top}B\vec{x} = \vec{x}^{^\top}M^{^\top}M\vec{x} = (M\vec{x})^{^\top}M\vec{x} = M\vec{x} \cdot M\vec{x}$. 
Let $y = M\vec{x}$. 
Since $M$ is invertible and $\vec{x} \neq \vec{0}$, $\vec{y}\neq\vec{0}$.
Hence, $M\vec{x}\cdot M\vec{x} = \vec{y}\cdot \vec{y} > 0$.

Therefore, $B$ is positive definite.
\bigbreak

\textbf{Problem 5.}
\begin{enumerate}[label={\alph*)}]
    \item 
    \begin{align}
        A = \begin{bmatrix}
            1 & 1 & 0 & 0 & 0 \\
            2 & 3 & 3 & 0 & 0 \\
            0 & 1 & 2 & 1 & 0 \\
            0 & 0 & 2 & 3 & 1 \\
            0 & 0 & 0 & 5 & 2
        \end{bmatrix} & \Rightarrow 
        L = \begin{bmatrix}
            1 & 0 & 0 & 0 & 0 \\
            2 & 1 & 0 & 0 & 0 \\
            0 & 0 & 1 & 0 & 0 \\
            0 & 0 & 0 & 1 & 0 \\
            0 & 0 & 0 & 0 & 1
        \end{bmatrix}; \quad
        U = \begin{bmatrix}
            1 & 1 & 0 & 0 & 0 \\
            0 & 1 & 3 & 0 & 0 \\
            0 & 1 & 2 & 1 & 0 \\
            0 & 0 & 2 & 3 & 1 \\
            0 & 0 & 0 & 5 & 2
        \end{bmatrix} \nonumber \\ 
        & \Rightarrow
        L = \begin{bmatrix}
            1 & 0 & 0 & 0 & 0 \\
            2 & 1 & 0 & 0 & 0 \\
            0 & 1 & 1 & 0 & 0 \\
            0 & 0 & 0 & 1 & 0 \\
            0 & 0 & 0 & 0 & 1
        \end{bmatrix}; \quad
        U = \begin{bmatrix}
            1 & 1 & 0 & 0 & 0 \\
            0 & 1 & 3 & 0 & 0 \\
            0 & 0 & -1 & 1 & 0 \\
            0 & 0 & 2 & 3 & 1 \\
            0 & 0 & 0 & 5 & 2
        \end{bmatrix} \nonumber \\
        & \Rightarrow
        L = \begin{bmatrix}
            1 & 0 & 0 & 0 & 0 \\
            2 & 1 & 0 & 0 & 0 \\
            0 & 1 & 1 & 0 & 0 \\
            0 & 0 & -2 & 1 & 0 \\
            0 & 0 & 0 & 0 & 1
        \end{bmatrix}; \quad
        U = \begin{bmatrix}
            1 & 1 & 0 & 0 & 0 \\
            0 & 1 & 3 & 0 & 0 \\
            0 & 0 & -1 & 1 & 0 \\
            0 & 0 & 0 & 5 & 1 \\
            0 & 0 & 0 & 5 & 2
        \end{bmatrix} \nonumber \\
        & \Rightarrow
        L = \begin{bmatrix}
            1 & 0 & 0 & 0 & 0 \\
            2 & 1 & 0 & 0 & 0 \\
            0 & 1 & 1 & 0 & 0 \\
            0 & 0 & -2 & 1 & 0 \\
            0 & 0 & 0 & 1 & 1
        \end{bmatrix}; \quad
        U = \begin{bmatrix}
            1 & 1 & 0 & 0 & 0 \\
            0 & 1 & 3 & 0 & 0 \\
            0 & 0 & -1 & 1 & 0 \\
            0 & 0 & 0 & 5 & 1 \\
            0 & 0 & 0 & 0 & 1
        \end{bmatrix} \nonumber 
    \end{align}

    $L$ is a lower triangular matrix with bandwidth $s=1$ (only have non-zero entries on diagonal and subdiagonal), and $U$ is an upper triangular matrix with bandwidth $t=1$ (only have non-zero entries on diagonal and superdiagonal).

    \item 
    $L$ would be a lower triangular matrix with bandwidth $s$ and $U$ would be an upper triangular matrix with bandwidth $t$.
\end{enumerate}
\end{document}