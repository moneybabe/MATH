\documentclass{article}
\usepackage{amsfonts, amsmath, amssymb, amsthm} % Math notations imported
\usepackage{enumitem}
\usepackage[margin=1in]{geometry}

\newtheorem{thm}{Theorem}
\newtheorem{prop}[thm]{Proposition}
\newtheorem{cor}[thm]{Corollary}

% title information
\title{Math 180A HW6}
\author{Neo Lee}
\date{02/22/2023}

% main content
\begin{document} 

% placing title information; comment out if using fancyhdr
\maketitle 

\textbf{Problem 1.}
\begin{enumerate}
    \item (b)
    \item (a)
    \item 
    \begin{align}
        E[X^2] & = Var(X)+\left(E[X]\right)^2 \\
        & = \sigma^2+\mu^2 \\
        & = 4 + 9 \\
        & = 13
    \end{align}
\end{enumerate}
\bigbreak

\textbf{Problem 2.} (c)
\bigbreak

\textbf{Problem 3.} 
\begin{enumerate}[label={(\alph*)}]
    \item 
    \begin{align}
        P(X\ge 10 + 15|X\ge 10) & = P(X\ge 15) \\ 
        & = e^{-(\frac{1}{20}\times 15)} \\
        & \approx 0.4724
    \end{align}

    \item 
    \begin{align}
        P(X \ge 25|X\ge 10) & = \frac{P(X\ge 25)}{P(X\ge 10)} \\
        & = \frac{\int_{25}^{40}\frac{1}{40}dx}{\int_{10}^{40}\frac{1}{40}dx} \\
        & = \frac{40-25}{40-10} \\
        & = \frac{1}{2}
    \end{align}

    \item 
    Since exponential random variable is memoryless, the probability of waiting for at least 15 additional minutes is always the same given the time before has already passed.

    On the other hand, uniform random vairable is not memoryless. In fact, the probability is eventually ditributed throughout the interval $[0,40]$.
    Therefore, the probabily of waiting would be $\frac{P(X\ge n+15)}{P(X\ge n)}$ given that $n$ has passed.
\end{enumerate}
\pagebreak

\textbf{Problem 4.}
\begin{enumerate}[label={(\alph*)}]
    \item 
    \begin{align}
        E[Z^n] & = \frac{1}{\sqrt{2\pi}}\int_{-\infty}^{\infty}z^ne^{\frac{-z^2}{2}}dz \\
        & = \frac{1}{\sqrt{2\pi}}\left[\left[-z^{n-1}e^{\frac{-z^2}{2}}\right]_{-\infty}^{\infty}+\int_{-\infty}^{\infty}(n-1)z^{n-2}e^{\frac{-z^2}{2}}dz\right] \\
        & = \frac{n-1}{\sqrt{2\pi}}\int_{-\infty}^{\infty}z^{n-2}e^{\frac{-z^2}{2}}dz \\
        & = (n-1)E[Z^{n-2}]
    \end{align}
    
    Hence, $E[Z^3]=(3-1)E[Z]=0.$

    \item 
    \begin{align}
        E[X^3] & = E[\left(\sigma Z+\mu\right)^3] \\
        & = E[\sigma^3Z^3+3\sigma^2\mu Z^2+3\sigma\mu^2Z+\mu^3] \\
        & = \sigma^2E[Z^3]+3\sigma^2\mu E[Z^2]+3\sigma\mu^2E[Z]+E[\mu^3] \\
        & =3\sigma^2\mu+\mu^3
    \end{align}
\end{enumerate}
\bigbreak

\textbf{Problem 5.}
For $a \in [0,1]$,
\begin{align}
    P(X \in [0,1] \cap X \in [a,2]) & = P(X\in[a,1]) \\
    & = e^{-2a}-e^{-2}.
\end{align}

For the events to be independent,
\begin{align}
    P(X \in [0,1] \cap X \in [a,2]) & =P(X\in [0,1])P(X\in [a,2]) \\
    e^{-2a}-e^{-2} & = \left(1-e^{-2}\right)\times \left(e^{-2a}-e^{-4}\right) \\
    e^{-2a}-e^{-2} & = e^{-2a}-e^{-4}-e^{-2a-2}+e^{-6} \\
    e^{-2a-2} & =e^{-6}-e^{-4}+e^{-2} \\
    e^{-2a-2} & =e^{-2}\left(e^{-4}-e^{-2}+1\right) \\
    -2a-2 & = -2 + ln(e^{-4}-e^{-2}+1) \\
    a & = \frac{ln(e^{-4}-e^{-2}+1)}{-2} \\
    & \approx 0.0622
\end{align}
\bigbreak

\textbf{Problem 6.}
\begin{enumerate}[label={(\alph*)}]
    \item There can be two scenarios: 1) if the stove breaks within $r$ years, profit $ = \$C-600$; 2) if the stove lasts longer than $r$ years, profit = $\$200+C$.
    
    Let $g(x)$ be a function that calculates the profit, then 
    \begin{align}
        E[g(x)] & = (C-600) \times \left(1-P(X \ge r)\right)+(200+C)\times P(X\ge r) \\
        & = (C-600) \times (1 - e^{\frac{-r}{10}})+(200+C)\times e^{\frac{-r}{10}} \\
        & = C-600+600e^{\frac{-r}{10}}+200e^{\frac{-r}{10}} \\
        & = 800e^{\frac{-r}{10}}-600+C
    \end{align}

    \item 
    \begin{align}
        800e^{\frac{-5}{10}}-600+C & = 0 \\
        C & = 600 - 800e^{-0.5} \\
        & \approx 114.8
    \end{align}
\end{enumerate}
\bigbreak

\textbf{Problem 7.}
\begin{align}
    P((X,Y)\in [0,1])&=1 \\
    c\int_{0}^{1}\int_{0}^{1}xy+y^2dxdy & = 1 \\
    c\int_{0}^{1}\left[\frac{1}{2}x^2y+xy^2\right]_{x=0}^{x=1} & = 1 \\
    c\int_{0}^{1}\frac{1}{2}y+y^2& = 1 \\
    c\left[\frac{1}{4}y^2+\frac{1}{3}y^3\right]_0^1 & = 1 \\
    \frac{7}{12}c & = 1 \\
    c & = \frac{12}{7}
\end{align}
\end{document}