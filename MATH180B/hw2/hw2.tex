\documentclass{article}
\usepackage{amsfonts, amsmath, amssymb, amsthm} % Math notations imported
\usepackage{enumitem}
\usepackage[margin=1in]{geometry}
\usepackage{blkarray}

\newtheorem{thm}{Theorem}
\newtheorem{prop}[thm]{Proposition}
\newtheorem{cor}[thm]{Corollary}

% title information
\title{Math 180B HW2}
\author{Neo Lee}
\date{04/21/2023}

% main content
\begin{document} 

% placing title information; comment out if using fancyhdr
\maketitle 
\textbf{PK Exercise 3.1.2}
\begin{align}
    P(X_2 = 1, X_3 = 1|X_1 = 0) &= P(X_3=1|X_2=1)P(X_1=1|X_1=0) \nonumber \\
    & = 0.6 \times 0.2 \nonumber \\
    & = 0.12. \nonumber 
\end{align}

Since we are considering only stationary Markov's Chain,
\begin{align}
    P(X_1 = 1, X_2 = 1|X_0 = 0) &= P(X_3=1|X_2=1)P(X_1=1|X_1=0) \nonumber \\
    &= 0.12. \nonumber
\end{align}
\bigbreak


\textbf{PK Problem 3.1.1}
\begin{align}
    P = 
    \begin{blockarray}{cccccc}
        & 1 & 2 & 3 & 4 & 5 \\
        \begin{block}{c||ccccc||}
          1 & 0.96 & 0.04 & 0 & 0 & 0 \\
          2 & 0 & 0.94 & 0.06 & 0 & 0 \\
          3 & 0 & 0 & 0.94 & 0.06 & 0 \\
          4 & 0 & 0 & 0 & 0.96 & 0.04 \\
          5 & 0 & 0 & 0 & 0 & 1 \\
        \end{block}
    \end{blockarray} \nonumber
\end{align}

Each entry of the matrix is simply determined by combinatorics after considering how the two people can be picked, which we will omit here for better readability.
For solely demonstration purposes, 
\begin{align}    
    P_{33} = \frac{{3\choose 2} + {2\choose 2} + 0.9\times {3\choose 1}{2\choose 1}}{{5\choose 2}} = 0.94; \quad P_{34} = \frac{0.1\times{3\choose 1}{2\choose 1}}{{5\choose 2}} = 0.06 = 1-0.94. \nonumber
\end{align}

Then, at the end of \emph{n-th} period, the transition matrix is simply $P^n$.
\bigbreak


\textbf{PK Problem 3.1.2}
\begin{enumerate}[label=(\alph*)]
    \item \begin{align}
        P(X_0=0, X_1=0, X_2=0) = P_{00}\times P_{00} = (1-\alpha)^2. \nonumber
    \end{align}

    \item \begin{align}
        P(X_0=0, X_2=0) & = P_{00}\times P_{00} + P_{01}\times P_{10} = (1-\alpha)^2 + \alpha^2. \nonumber
    \end{align}
\end{enumerate}
\bigbreak


\textbf{PK Exercise 3.2.6}
\begin{align}
    P^2 = 
    \begin{blockarray}{cccc}
        & 0 & 1 & 2  \\
        \begin{block}{c||ccc||}
          0 & 0.44 & 0.18 & 0.38  \\
          1 & 0.4 & 0.19 & 0.41  \\
          2 & 0.4 & 0.18 & 0.42  \\
        \end{block}
    \end{blockarray}~; \quad
    P^3 =
    \begin{blockarray}{cccc}
        & 0 & 1 & 2  \\
        \begin{block}{c||ccc||}
          0 & 0.412 & 0.182 & 0.406  \\
          1 & 0.42 & 0.181 & 0.399  \\
          2 & 0.42 & 0.182 & 0.398  \\
        \end{block}
    \end{blockarray}~. \nonumber
\end{align}

Then,
\begin{align}
    P(X_2=0) & = P(X_2=0|X_0=0)P(X_0=0) + P(X_2=0|X_0=1)P(X_0=1) \nonumber \\ 
    & = 0.44 \times 0.5 + 0.4 \times 0.5 \nonumber \\
    & = 0.42, \nonumber
\end{align} and

\begin{align}
    P(X_3=0) & = P(X_3=0|X_0=0)P(X_0=0) + P(X_3=0|X_0=1)P(X_0=1) \nonumber \\
    & = 0.412 \times 0.5 + 0.42 \times 0.5 \nonumber \\
    & = 0.416. \nonumber
\end{align}
\bigbreak


\textbf{PK Problem 3.2.2}
Let each transition be a \emph{Bernoulli} random variable with 1 being sending a wrong signal and 0 being sending a correct signal. 
Then,
\begin{align}
    I = \begin{cases}
        1, & \alpha \\
        0, & 1-\alpha.
    \end{cases} \nonumber
\end{align}

Then, the distribution of sending a signal is a \emph{Binomial(5, $1-\alpha$)}, and the probability of sending a correct signal is 
\begin{align}
    P & = {5\choose 0} (1-\alpha)^5 + {5\choose 2} \alpha^2(1-\alpha)^3 + {5\choose 4} \alpha^4(1-\alpha) \nonumber \\
    & = -16\alpha^5 + 40\alpha^4 - 40\alpha^3 + 20\alpha^2 - 5\alpha + 1. \nonumber
\end{align}
\bigbreak


\textbf{PK Problem 3.3.1}
\begin{align}
    P = 
    \begin{blockarray}{ccccc}
        & 0 & 1 & 2 & 3  \\
        \begin{block}{c||cccc||}
          0 & 1 & 0 & 0 & 0  \\
          1 & \frac{1}{15} & \frac{14}{15} & 0 & 0  \\
          2 & 0 & \frac{4}{15} & \frac{11}{15} & 0  \\
          3 & 0 & 0 & 0.6 & 0.4 \\
        \end{block}
    \end{blockarray} \nonumber
\end{align}

Again, each entry of the matrix is simply determined by combinatorics after considering how the two tags can be picked, which we will omit here for better readability.
For solely demonstration purposes,
\begin{align}
    P_{21} = \frac{{2\choose 1}{2\choose 1}}{{6\choose 2}} = \frac{4}{15};\quad P_{22} = \frac{3\times {2\choose 2}}{{6\choose 2}} + \frac{2\times {2\choose 1}{2\choose 1}}{{6\choose 2}}= \frac{11}{15} = 1-\frac{4}{15}. \nonumber
\end{align}
\end{document}