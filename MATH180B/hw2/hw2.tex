\documentclass{article}
\usepackage{amsfonts, amsmath, amssymb, amsthm} % Math notations imported
\usepackage{enumitem}
\usepackage[margin=1in]{geometry}
\usepackage{blkarray}

\newtheorem{thm}{Theorem}
\newtheorem{prop}[thm]{Proposition}
\newtheorem{cor}[thm]{Corollary}

% title information
\title{Math 180B HW2}
\author{Neo Lee}
\date{04/21/2023}

% main content
\begin{document} 

% placing title information; comment out if using fancyhdr
\maketitle 
\textbf{PK Exercise 3.1.2}
\begin{align}
    P(X_2 = 1, X_3 = 1|X_1 = 0) &= P(X_3=1|X_2=1)P(X_1=1|X_1=0) \nonumber \\
    & = 0.6 \times 0.2 \nonumber \\
    & = 0.12. \nonumber 
\end{align}

Since we are considering only stationary Markov's Chain,
\begin{align}
    P(X_1 = 1, X_2 = 1|X_0 = 0) &= P(X_3=1|X_2=1)P(X_1=1|X_1=0) \nonumber \\
    &= 0.12. \nonumber
\end{align}
\bigbreak


\textbf{PK Problem 3.1.1}
\begin{align}
    P = 
    \begin{blockarray}{cccccc}
        & 1 & 2 & 3 & 4 & 5 \\
        \begin{block}{c||ccccc||}
          1 & 0.96 & 0.04 & 0 & 0 & 0 \\
          2 & 0 & 0.94 & 0.06 & 0 & 0 \\
          3 & 0 & 0 & 0.94 & 0.06 & 0 \\
          4 & 0 & 0 & 0 & 0.96 & 0.04 \\
          5 & 0 & 0 & 0 & 0 & 1 \\
        \end{block}
    \end{blockarray} \nonumber
\end{align}

Then, at the end of \emph{n-th} period, the transition matrix is simply $P^n$.
\bigbreak


\textbf{PK Problem 3.1.2}
\begin{enumerate}[label=(\alph*)]
    \item \begin{align}
        P(X_0=0, X_1=0, X_2=0) = P_{00}\times P_{00}\times P_{00} = (1-\alpha)^3. \nonumber
    \end{align}

    \item \begin{align}
        P(X_0=0, X_1=0, X_2=0) & = P_{00}\times P_{00}\times P_{01} + P_{00}\times P_{01}\times P_{01} \nonumber \\
        & = (1-\alpha)^3 + \alpha(1-\alpha)^2. \nonumber
    \end{align}
\end{enumerate}
\bigbreak


\textbf{PK Exercise 3.2.6}

\bigbreak


\textbf{PK Problem 3.2.2}

\bigbreak


\textbf{PK Problem 3.3.1}

\end{document}