\documentclass{article}
\usepackage{amsfonts, amsmath, amssymb, amsthm} % Math notations imported
\usepackage{enumitem}
\usepackage[margin=1in]{geometry}
\usepackage{blkarray}

\newtheorem{thm}{Theorem}
\newtheorem{prop}[thm]{Proposition}
\newtheorem{cor}[thm]{Corollary}

% title information
\title{Math 180B HW5}
\author{Neo Lee}
\date{05/12/2023}

% main content
\begin{document} 

% placing title information; comment out if using fancyhdr
\maketitle 

\textbf{PK Exercise 4.1.4}
\begin{align*}
    0.3\pi_0 + 0.5\pi_1 + 0.5\pi_2 & = \pi_0 \\
    0.2\pi_0 + 0.1\pi_1 + 0.2\pi_2 & = \pi_1 \\
    0.5\pi_0 + 0.4\pi_1 + 0.3\pi_2 & = \pi_2 \\
    \pi_0 + \pi_1 + \pi_2 & = 1
\end{align*}

Then, solving the system of equations, we get $\pi_0 = \frac{5}{12}, \pi_1 = \frac{2}{11}, \pi_2 = \frac{53}{132}$.
Therefore, the long run cost per period is $2\times \frac{5}{12} + 5\times \frac{2}{11} + 3\times \frac{53}{132} \approx 2.95.$
\bigbreak

\textbf{PK Problem 4.1.3} \emph{assume $\alpha_i > 0$ for $i \in \{1,\dots, 6\}$}
\begin{align*}
    \pi_0\alpha_1 + \pi_1 & = \pi_0 \Rightarrow \pi_1 = (1-\alpha_1)\pi_0 \\
    \pi_0\alpha_2 + \pi_2 & = \pi_1 \Rightarrow \pi_2 = (1-\alpha_1-\alpha_2)\pi_0 \\
    \pi_0\alpha_3 + \pi_3 & = \pi_2 \Rightarrow \pi_3 = (1-\alpha_1-\alpha_2-\alpha_3)\pi_0 \\
    \pi_0\alpha_4 + \pi_4 & = \pi_3 \Rightarrow \pi_4 = (1-\alpha_1-\alpha_2-\alpha_3-\alpha_4)\pi_0 \\
    \pi_0\alpha_5 + \pi_5 & = \pi_4 \Rightarrow \pi_5 = (1-\alpha_1-\alpha_2-\alpha_3-\alpha_4-\alpha_5)\pi_0 \\
    \pi_0 + \pi_1 + \pi_2 + \pi_3 + \pi_4 + \pi_5 & = 1 \Rightarrow \pi_0 = \frac{1}{\sum_{i=0}^5 (1-\alpha_i)}
\end{align*}
\bigbreak

\textbf{PK Problem 4.1.5}
\begin{align*}
    P & = 
    \begin{blockarray}{ccccc}
        & A & B & C & D  \\
        \begin{block}{c||cccc||}
            A & 0 & \frac{1}{2} & 0 & \frac{1}{2} \\
            B & \frac{1}{3} & 0 & \frac{1}{3} & \frac{1}{3} \\
            C & 0 & 1 & 0 & 0 \\
            D & \frac{1}{2} & \frac{1}{2} & 0 & 0 \\
        \end{block}
    \end{blockarray}\; . \nonumber \\
    \frac{1}{3}\pi_1 + \frac{1}{2}\pi_3 & = \pi_0 \\
    \frac{1}{2}\pi_0 + \pi_2 + \frac{1}{2}\pi_3 & = \pi_1 \\
    \frac{1}{3}\pi_1 & = \pi_2 \\
    \pi_0 + \pi_1 + \pi_2 + \pi_3 & = 1
\end{align*} 

Solving the system of equations, we get $\pi_0 = \frac{1}{4}, \pi_1 = \frac{3}{8}, \pi_2 = \frac{1}{8}, \pi_3 = \frac{1}{4}$.
\bigbreak

\textbf{PK Problem 4.1.11 (a), (b)}
\begin{enumerate}[label={(\alph*)}]
    \item 
    \begin{align*}
        0.5\pi_0 + 0.2\pi_1 + 0.3\pi_2 +0.2\pi_3 & = \pi_1 \\
        0.2\pi_1 + 0.4\pi_2 +0.4\pi_3 & = \pi_2
    \end{align*}

    Pluggin in $\pi_1 = \frac{119}{379}$ and $\pi_2 = \frac{81}{379}$ and solving the system of equations, 
    we get $\underline{\pi_0 = \frac{117}{379}}, \pi_1 = \frac{119}{379}, \pi_2 = \frac{81}{379}, \underline{\pi_3 = \frac{62}{379}}$.

    \item 
    $\mu = \pi_2 + \pi_3 = \frac{81}{379} + \frac{62}{379} \approx 0.377.$
\end{enumerate}
\bigbreak

\textbf{PK Problem 4.2.4}
\begin{enumerate}[label={(\alph*)}]
    \item 
    \begin{align*}
        P & = 
        \begin{blockarray}{ccccc}
            & 0 & 1 & 2 & 3 \\
            \begin{block}{c||cccc||}
                0 & 0.1 & 0.3 & 0.2 & 0.4 \\
                1 & 1 & 0 & 0 & 0 \\
                2 & 0 & 1 & 0 & 0 \\
                3 & 0 & 0 & 1 & 0 \\
            \end{block}
        \end{blockarray}\; . \nonumber \\
    \end{align*}

    \item
    $P$ is regular because for $i = 0$, $P_{00} > 0$, and there is a path $k_1, \dots, k_r$ for which $P_{ik_1}\cdots P_{k_rj} > 0$ for every state pair $i,j$.
    \begin{align*}
        0.1\pi_0 + \pi_1 & = \pi_0 \\
        0.3\pi_0 + \pi_2 & = \pi_1 \\
        0.2\pi_0 + \pi_3 & = \pi_2 \\
        0.4\pi_0 & = \pi_3 \\
        \pi_0 + \pi_1 + \pi_2 + \pi_3 & = 1
    \end{align*}

    Then, solving the system of equations, we get $\underline{\pi_0 = \frac{10}{29}}, \pi_1 = \frac{9}{29}, \pi_2 = \frac{6}{29}, \pi_3 = \frac{4}{29}$.

    \item
    \begin{align*}
        \pi_0 = \frac{1}{E[\xi]} = \frac{1}{0.1 + 0.3\times 2 + 0.2\times 3 + 0.4\times 4} = \frac{10}{29}.
    \end{align*}
\end{enumerate}
\bigbreak

\textbf{PK Problem 4.2.6} 

On day $n$, if the computer is operating, which means at state 1, it has probability $q$ of remaining "up" at state 1 and probability $p$ of failing and going to state 0.


On the other hand, on day $n$, if the computer is down, which means at state 0, it has probability $\beta$ of being repaired within a day and goes to state 1, and probability $1-\beta = \alpha$ of remaining "down" at state 0.

\begin{align*}\
    \pi_0\beta + \pi_1q & = \pi_1 \\
    \pi_0 + \pi_1 & = 1.
\end{align*}

Then, by solving the system of equations, we get 
\begin{align*}
    (1-\pi_1)\beta + \pi_1q & = \pi_1 \\
    \beta + \pi_1(q-\beta) & = \pi_1 \\
    \beta & = \pi_1(1-q+\beta) \\
    \pi_1 & = \frac{\beta}{1-q+\beta} \\
    \pi_1 & = \frac{\beta}{p+\beta}.
\end{align*}
\bigbreak



\end{document}