\documentclass{article}
\usepackage{amsfonts, amsmath, amssymb, amsthm} % Math notations imported
\usepackage{enumitem}
\usepackage[margin=1in]{geometry}

\newtheorem{thm}{Theorem}
\newtheorem{prop}[thm]{Proposition}
\newtheorem{cor}[thm]{Corollary}

% title information
\title{Math 180B HW6}
\author{Neo Lee}
\date{05/19/2023}

% main content
\begin{document} 

% placing title information; comment out if using fancyhdr
\maketitle 

\textbf{PK Exercise 4.3.3 (b)}
\begin{proof}[Solution]
    \{0\}, \{1, 2\}, \{3, 4\}, \{5\} are communicating classes. By observing whether the class is closed or not, we can immediately tell that \{0\}, \{1, 2\}, \{5\} are recurrent and \{3, 4\} are transient.
\end{proof}
\bigbreak


\textbf{PK Exercise 4.3.4}
\begin{proof}[Solution]
    \{2, 3, 4, 5\} is a communicating class. By observing the period of \{5\} is 1, we can immediately tell that the period of \{2, 3, 4, 5\} is 1. The period of \{0\} is 1. The period of \{1\} is 0.
\end{proof}
\bigbreak


\textbf{PK Problem 4.3.3} \emph{(for n = 1,2,3,4)}
\begin{proof}[Solution]
    Using MATLAB and computing the matrix multiplication $P^n$, we have
    \begin{align*}
        P^{(0)} & = 1 \\
        P^{(1)} & = 0 \\
        P^{(2)} & = \frac{1}{4} \\
        P^{(3)} & = \frac{1}{8} \\
        P^{(4)} & = \frac{3}{8} \\
        P^{(5)} & = \frac{7}{32}.
    \end{align*}

    Then, using \emph{equation (4.16)}, 
    \begin{align*}
        P^{(1)} & = f^{(1)}P^{(0)} \Rightarrow f^{(1)} = 0 \\
        P^{(2)} & = f^{(1)}P^{(1)} + f^{(2)}P^{(0)} \Rightarrow f^{(2)} = \frac{1}{4} \\
        P^{(3)} & = f^{(1)}P^{(2)} + f^{(2)}P^{(1)} + f^{(3)}P^{(0)} \Rightarrow f^{(3)} = \frac{1}{8} \\
        P^{(4)} & = f^{(1)}P^{(3)} + f^{(2)}P^{(2)} + f^{(3)}P^{(1)} + f^{(4)}P^{(0)} \Rightarrow f^{(4)} = \frac{5}{16} \\
        P^{(5)} & = f^{(1)}P^{(4)} + f^{(2)}P^{(3)} + f^{(3)}P^{(2)} + f^{(4)}P^{(1)} + f^{(5)}P^{(0)} \Rightarrow f^{(5)} = \frac{5}{32}.
    \end{align*}
\end{proof}
\bigbreak

\newpage
\textbf{PK Exercise 4.4.2}
\begin{enumerate}[label=(\alph*)]
    \item 
    \begin{align*}
        0.1\pi_1 + 0.2\pi_2 + 0.3\pi_3 & = \pi_0 \\
        \pi_0 + 0.4\pi_1 + 0.2\pi_2 + 0.3\pi_3 & = \pi_1 \\
        0.2\pi_1 + 0.5\pi_2 + 0.4\pi_3 & = \pi_2 \\
        \pi_0 + \pi_1 + \pi_2 + \pi_3 & = 1.
    \end{align*}

    By solving the system of equations, we have $\underline{\pi_0 \approx 0.1449}, \pi_1 \approx 0.4140, \pi_2 \approx 0.2880, \pi_3 \approx 0.1530$.

    \item
    \begin{align*}
        v_1 & = 1 + 0.4v_1 + 0.2v_2 + 0.3v_3 \\
        v_2 & = 1 + 0.2v_1 + 0.5v_2 + 0.1v_3 \\
        v_3 & = 1 + 0.3v_1 + 0.4v_2.
    \end{align*}
    
    By solving the system of equations, we have $\underline{v_1 \approx 5.90}, v_2 \approx 5.34, v_3 \approx 4.91$.

    \item
    $$m_0 = 1 + 5.90 = 6.90.$$
    Indeed, $$\pi_0 = \frac{1}{m_0} = \frac{1}{6.90} \approx 0.1449.$$
    
\end{enumerate}
\bigbreak


\textbf{PK Problem 4.4.6}
\begin{proof}[Solution]
    $P^{(4)}_{00} > 0$ and $P^{(5)}_{00} > 0$. $gcd(4,5) = 1$. Hence, $d(0) = 1$.
\end{proof}
\bigbreak


\textbf{PK Problem 4.4.8}
\begin{proof}[Solution]
    \begin{align*}
        P = \begin{bmatrix}
            \frac{1}{2} & \frac{1}{2} & 0 & 0 & 0 & \cdots \\
            \frac{1}{3} & \frac{1}{3} & \frac{1}{3} & 0 & 0 & \cdots \\
            \frac{1}{4} & \frac{1}{4} & \frac{1}{4} & \frac{1}{4} & 0 & \cdots \\
            \vdots & \vdots & \vdots & \vdots & \vdots & \ddots
        \end{bmatrix}
    \end{align*}

    Then, we can set up the system of linear equations:
    \begin{align*}
        \pi_0 + \pi_1 + \pi_2 + \pi_3 + \cdots & = 1 \\
        \frac{1}{2}\pi_0 + \frac{1}{3}\pi_1 + \frac{1}{4}\pi_2 + \cdots & = \pi_0 \\
        \frac{1}{2}\pi_0 + \frac{1}{3}\pi_1 + \frac{1}{4}\pi_2 + \cdots & = \pi_1 \\
        \frac{1}{3}\pi_1 + \frac{1}{4}\pi_2 + \cdots & = \pi_2 \\
        \frac{1}{4}\pi_2 + \cdots & = \pi_3 \\
        \vdots &
    \end{align*}

    Now, notice that by subtracting $\pi_{n}$ from $\pi_{n-1}$, $$\pi_n = \pi_{n-1} - \frac{1}{n}\pi_{n-2}.$$
    Then, we can set up the recurrence relation:
    \begin{align*}
        \pi_1 & = \pi_0 \\
        \pi_2 & = \pi_1 - \frac{1}{2}\pi_0 = \frac{1}{2}\pi_0 \\
        \pi_3 & = \pi_2 - \frac{1}{3}\pi_1 = \frac{1}{2}\pi_0 - \frac{1}{3}\pi_0 = \frac{1}{6}\pi_0 \\
        \pi_4 & = \pi_3 - \frac{1}{4}\pi_2 = \frac{1}{6}\pi_0 - \frac{1}{8}\pi_0 = \frac{1}{24}\pi_0 \\
        \pi_5 & = \pi_4 - \frac{1}{5}\pi_3 = \frac{1}{24}\pi_0 - \frac{1}{30}\pi_0 = \frac{1}{120}\pi_0. \\
    \end{align*}
    By observing the pattern, we claim that $$\pi_n = \frac{1}{n!}\pi_0.$$ This can be proved rigorously by induction, but we will omit here.

    Now, we can solve for $\pi_0$:
    \begin{align*}
        \pi_0 + \pi_1 + \pi_2 + \pi_3 + \cdots & = 1 \\
        \pi_0 + \frac{1}{1!}\pi_0 + \frac{1}{2!}\pi_0 + \frac{1}{3!}\pi_0 + \cdots & = 1 \\
        \pi_0\left(1 + \frac{1}{1!} + \frac{1}{2!} + \frac{1}{3!} + \cdots\right) & = 1 \\
        \pi_0\left(e\right) & = 1 \\
        \pi_0 & = \frac{1}{e}.
    \end{align*}
    Then, we plug in $\pi_0 = \frac{1}{e}$ to get $\pi_n$
    \begin{align*}
        \pi_n & = \frac{1}{n!}\pi_0 \\
        \pi_n & = \frac{1}{n!\cdot e}.
    \end{align*}
\end{proof}
\bigbreak



\end{document}