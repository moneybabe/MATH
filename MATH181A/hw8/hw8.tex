\documentclass{article}
\usepackage{amsfonts, amsmath, amssymb, amsthm} % Math notations imported
\usepackage{enumitem}
\usepackage[margin=1in]{geometry}

\newtheorem{thm}{Theorem}
\newtheorem{prop}[thm]{Proposition}
\newtheorem{cor}[thm]{Corollary}

% title information
\title{Math 181A HW8}
\author{Neo Lee}
\date{05/23/2023}

% main content
\begin{document} 

% placing title information; comment out if using fancyhdr
\maketitle 

\textbf{Probelm 7.3.4} 
Use the fact that $\frac{(n-1)S^2}{\sigma^2}$ is a chi square random variable with $n - 1$ df to prove that
$Var(S^2) = \frac{2\sigma^4}{n-1}$.
\begin{proof}[Solution]
    \begin{align*}
        \frac{(n-1)S^2}{\sigma^2} & = \chi^2_{n-1} \\
        S^2 & = \frac{\sigma^2}{n-1}\chi^2_{n-1} \\
        Var(S^2) & = \frac{\sigma^4}{(n-1)^2}Var(\chi^2_{n-1}) \\
        Var(S^2) & = \frac{\sigma^4}{(n-1)^2}\cdot2(n-1) \\
        Var(S^2) & = \frac{2\sigma^4}{n-1}.
    \end{align*}
\end{proof}
\bigbreak


\textbf{Probelm 7.3.5}
Let $Y_1,Y_2,...,Y_n$ be a random sample from a normal distribution. 
Use the statement of \emph{Question 7.3.4} to prove that $S^2$ is consistent for $\sigma^2$.
\begin{proof}[Solution]
    Notice that $S^2 = \frac{1}{n-1}\sum_{k=1}^{n}(Y_k-\overline{Y})$ is an unbiased estimator of $\sigma^2$, which mean $E[S^2] = \sigma^2$.
    Then, by \emph{Chebyshev's Inequality}, we have
    \begin{align*}
        P(|S^2 - \sigma^2| \ge \epsilon) & \le \frac{Var(S^2)}{\epsilon^2} \\
        P(|S^2 - \sigma^2| \ge \epsilon) & \le \frac{2\sigma^4}{(n-1)\epsilon^2} \\
        \lim_{n\to\infty}P(|S^2 - \sigma^2| \ge \epsilon) & \le \lim_{n\to\infty}\frac{2\sigma^4}{(n-1)\epsilon^2} \\
        \lim_{n\to\infty}P(|S^2 - \sigma^2| \ge \epsilon) & \le 0 \\
        \lim_{n\to\infty}P(|S^2 - \sigma^2| \ge \epsilon) & = 0.
    \end{align*}
\end{proof}
\bigbreak


\textbf{Probelm 7.4.4}
A random sample of size $n = 9$ is drawn from a normal distribution with $\mu = 27.6$. 
Within what interval $(-a, +a)$ can we expect to find $\frac{\overline{Y}-27.6}{S/\sqrt{9}} 80\%$ the time? 90\% the time?
\begin{proof}[Solution]
    The sameple size is $n=9$, so the degree of freedom is $n-1=8$. 
    Then, we have $\frac{\overline{Y}-27.6}{S/\sqrt{9}} \sim t_8$ . Hence the critical value for 80\% and 90\% confidence are $t_{8, 0.01}= 1.397$ and $t_{8,0.05} = 1.86$ respectively.
    Therefore, the interval for 80\% confidence is $(-1.397, 1.397)$ and the interval for 90\% confidence is $(-1.86, 1.86)$.
\end{proof}


\bigbreak


\textbf{Probelm 7.4.7}
Cell phones emit radio frequency energy that is absorbed by the body when the phone is next to the ear may be harmful.
The table in the next column gives the absorption rate for a sample of twenty high-radiation cell phones.
(The Federal Communication Commission sets a maximum of 1.6 watts per kilogram for the absorption rate of such energy.) 
Construct a 90\% confidence interval for the true average cell phone absorption rate.
Sample: \{1.54, 1.41, 1.54, 1.40, 1.49, 1.40, 1.49, 1.39, 1.48, 1.39, 1.45, 1.39, 1.44, 1.38, 1.42, 1.38, 1.41, 1.37, 1.41, 1.33\}.
\begin{proof}[Solution]
    We will construct the confidence interval using the $t$ distribution with degree of freedom $n=19$ and $\alpha = 0.1$.
    First, let us find $\overline{X}$ and $s$.
    \begin{align*}
        \overline{X} & = \frac{1}{20}\sum_{k=1}^{n}X_k \\
        \overline{X} & \approx 1.4255 \\
        s^2 & = \frac{1}{n-1}\sum_{k=1}^{n}(X_k-\overline{X})^2 \\
        s^2 & \approx 0.003184 \\
        s & \approx 0.0564.
    \end{align*}
    
    The critical value is $t_{19, 0.05} \approx 1.729$.
    Then, the confidence interval is $1.4255 \pm 1.729\times\frac{0.0564}{\sqrt{20}} \approx (1.402, 1.449)$.
\end{proof}
\bigbreak


\textbf{Probelm 7.4.14}
Revenues reported last week from nine boutiques franchised by an international clothier averaged \$59,540 with a standard deviation of \$6860. 
Based on those figures, in what range might the company expect to find the average revenue of all of its boutiques?
\begin{proof}[Solution]
    We will construct the confidence interval using the $t$ distribution with degree of freedom $n=8$ and $\alpha = 0.05$.
    First, let's define our parameters $\overline{X} = 59540$ and $s = 6860$.
    
    The critical value is $t_{8, 0.025} \approx 2.306$.
    Then, the confidence interval is $59540 \pm 2.306\times\frac{6860}{\sqrt{9}} \approx (54267, 64813)$.
    Hence, with 95\% confidence, the company can expect to find the average revenue of all of its boutiques in the range of $(54267, 64813)$.
\end{proof}
\bigbreak



\end{document}