\documentclass{book}
%--------------------------
% This amazing template is from 
% https://github.com/SeniorMars/dotfiles/tree/master/latex_template
%--------------------------

%%%%%%%%%%%%%%%%%%%%%%%%%%%%%%%%%
% PACKAGE IMPORTS
%%%%%%%%%%%%%%%%%%%%%%%%%%%%%%%%%


\usepackage[tmargin=2cm,rmargin=1in,lmargin=1in,margin=0.85in,bmargin=2cm,footskip=.2in]{geometry}
\usepackage{amsmath,amsfonts,amsthm,amssymb,mathtools}
\usepackage[varbb]{newpxmath}
\usepackage{xfrac}
\usepackage{indentfirst}
\usepackage[T1]{fontenc}
\usepackage[makeroom]{cancel}
\usepackage{bookmark}
\usepackage{enumitem}
\usepackage{hyperref,theoremref}
\hypersetup{
	pdftitle={Assignment},
	colorlinks=true, linkcolor=doc!90,
	bookmarksnumbered=true,
	bookmarksopen=true
}
\usepackage[most,many,breakable]{tcolorbox}
\usepackage{xcolor}
\usepackage{varwidth}
\usepackage{varwidth}
\usepackage{etoolbox}
\usepackage{authblk}
\usepackage{nameref}
\usepackage{multicol,array}
\usepackage{tikz-cd}
\usepackage[ruled,vlined,linesnumbered]{algorithm2e}
\usepackage{comment} % enables the use of multi-line comments (\ifx \fi) 
\usepackage{import}
\usepackage{xifthen}
\usepackage{pdfpages}
\usepackage{transparent}
\usepackage{setspace}
\setstretch{1.15}
\fontfamily{cmr}\selectfont
\DeclareSymbolFont{letters}     {OML}{cmm} {m}{it}


\newcommand\mycommfont[1]{\footnotesize\ttfamily\textcolor{blue}{#1}}
\SetCommentSty{mycommfont}
\newcommand{\incfig}[1]{%
    \def\svgwidth{\columnwidth}
    \import{./figures/}{#1.pdf_tex}
}

\usepackage{tikzsymbols}
\renewcommand\qedsymbol{$\Laughey$}


%%%%%%%%%%%%%%%%%%%%%%%%%%%%%%
% SELF MADE COLORS
%%%%%%%%%%%%%%%%%%%%%%%%%%%%%%


\definecolor{myg}{RGB}{56, 140, 70}
\definecolor{myb}{RGB}{45, 111, 177}
\definecolor{myr}{RGB}{199, 68, 64}
\definecolor{mytheorembg}{HTML}{F2F2F9}
\definecolor{mytheoremfr}{HTML}{00007B}
\definecolor{mylemmabg}{HTML}{FFFAF8}
\definecolor{mylemmafr}{HTML}{983b0f}
\definecolor{mypropbg}{HTML}{f2fbfc}
\definecolor{mypropfr}{HTML}{191971}
\definecolor{myexamplebg}{HTML}{F2FBF8}
\definecolor{myexamplefr}{HTML}{88D6D1}
\definecolor{myexampleti}{HTML}{2A7F7F}
\definecolor{mydefinitbg}{HTML}{E5E5FF}
\definecolor{mydefinitfr}{HTML}{3F3FA3}
\definecolor{notesgreen}{RGB}{0,162,0}
\definecolor{myp}{RGB}{197, 92, 212}
\definecolor{mygr}{HTML}{2C3338}
\definecolor{myred}{RGB}{127,0,0}
\definecolor{myyellow}{RGB}{169,121,69}
\definecolor{myexercisebg}{HTML}{F2FBF8}
\definecolor{myexercisefg}{HTML}{88D6D1}


%%%%%%%%%%%%%%%%%%%%%%%%%%%%
% TCOLORBOX SETUPS
%%%%%%%%%%%%%%%%%%%%%%%%%%%%

\setlength{\parindent}{1cm}
%================================
% THEOREM BOX
%================================

\tcbuselibrary{theorems,skins,hooks}
\newtcbtheorem[number within=section]{Theorem}{Theorem}
{%
	enhanced,
	breakable,
	colback = mytheorembg,
	frame hidden,
	boxrule = 0sp,
	borderline west = {2pt}{0pt}{mytheoremfr},
	sharp corners,
	detach title,
	before upper = \tcbtitle\par\smallskip,
	coltitle = mytheoremfr,
	fonttitle = \bfseries\sffamily,
	description font = \mdseries,
	separator sign none,
	segmentation style={solid, mytheoremfr},
}
{th}

\tcbuselibrary{theorems,skins,hooks}
\newtcbtheorem[number within=section]{theorem}{Theorem}
{%
	enhanced,
	breakable,
	colback = mytheorembg,
	frame hidden,
	boxrule = 0sp,
	borderline west = {2pt}{0pt}{mytheoremfr},
	sharp corners,
	detach title,
	before upper = \tcbtitle\par\smallskip,
	coltitle = mytheoremfr,
	fonttitle = \bfseries\sffamily,
	description font = \mdseries,
	separator sign none,
	segmentation style={solid, mytheoremfr},
}
{th}


\tcbuselibrary{theorems,skins,hooks}
\newtcolorbox{Theoremcon}
{%
	enhanced
	,breakable
	,colback = mytheorembg
	,frame hidden
	,boxrule = 0sp
	,borderline west = {2pt}{0pt}{mytheoremfr}
	,sharp corners
	,description font = \mdseries
	,separator sign none
}

%================================
% Corollery
%================================
\tcbuselibrary{theorems,skins,hooks}
\newtcbtheorem[number within=section]{Corollary}{Corollary}
{%
	enhanced
	,breakable
	,colback = myp!10
	,frame hidden
	,boxrule = 0sp
	,borderline west = {2pt}{0pt}{myp!85!black}
	,sharp corners
	,detach title
	,before upper = \tcbtitle\par\smallskip
	,coltitle = myp!85!black
	,fonttitle = \bfseries\sffamily
	,description font = \mdseries
	,separator sign none
	,segmentation style={solid, myp!85!black}
}
{th}
\tcbuselibrary{theorems,skins,hooks}
\newtcbtheorem[number within=section]{corollary}{Corollary}
{%
	enhanced
	,breakable
	,colback = myp!10
	,frame hidden
	,boxrule = 0sp
	,borderline west = {2pt}{0pt}{myp!85!black}
	,sharp corners
	,detach title
	,before upper = \tcbtitle\par\smallskip
	,coltitle = myp!85!black
	,fonttitle = \bfseries\sffamily
	,description font = \mdseries
	,separator sign none
	,segmentation style={solid, myp!85!black}
}
{th}


%================================
% LEMMA
%================================

\tcbuselibrary{theorems,skins,hooks}
\newtcbtheorem[number within=section]{Lemma}{Lemma}
{%
	enhanced,
	breakable,
	colback = mylemmabg,
	frame hidden,
	boxrule = 0sp,
	borderline west = {2pt}{0pt}{mylemmafr},
	sharp corners,
	detach title,
	before upper = \tcbtitle\par\smallskip,
	coltitle = mylemmafr,
	fonttitle = \bfseries\sffamily,
	description font = \mdseries,
	separator sign none,
	segmentation style={solid, mylemmafr},
}
{th}

\tcbuselibrary{theorems,skins,hooks}
\newtcbtheorem[number within=section]{lemma}{Lemma}
{%
	enhanced,
	breakable,
	colback = mylemmabg,
	frame hidden,
	boxrule = 0sp,
	borderline west = {2pt}{0pt}{mylemmafr},
	sharp corners,
	detach title,
	before upper = \tcbtitle\par\smallskip,
	coltitle = mylemmafr,
	fonttitle = \bfseries\sffamily,
	description font = \mdseries,
	separator sign none,
	segmentation style={solid, mylemmafr},
}
{th}


%================================
% PROPOSITION
%================================

\tcbuselibrary{theorems,skins,hooks}
\newtcbtheorem[number within=section]{Prop}{Proposition}
{%
	enhanced,
	breakable,
	colback = mypropbg,
	frame hidden,
	boxrule = 0sp,
	borderline west = {2pt}{0pt}{mypropfr},
	sharp corners,
	detach title,
	before upper = \tcbtitle\par\smallskip,
	coltitle = mypropfr,
	fonttitle = \bfseries\sffamily,
	description font = \mdseries,
	separator sign none,
	segmentation style={solid, mypropfr},
}
{th}

\tcbuselibrary{theorems,skins,hooks}
\newtcbtheorem[number within=section]{prop}{Proposition}
{%
	enhanced,
	breakable,
	colback = mypropbg,
	frame hidden,
	boxrule = 0sp,
	borderline west = {2pt}{0pt}{mypropfr},
	sharp corners,
	detach title,
	before upper = \tcbtitle\par\smallskip,
	coltitle = mypropfr,
	fonttitle = \bfseries\sffamily,
	description font = \mdseries,
	separator sign none,
	segmentation style={solid, mypropfr},
}
{th}


%================================
% CLAIM
%================================

\tcbuselibrary{theorems,skins,hooks}
\newtcbtheorem[number within=section]{claim}{Claim}
{%
	enhanced
	,breakable
	,colback = myg!10
	,frame hidden
	,boxrule = 0sp
	,borderline west = {2pt}{0pt}{myg}
	,sharp corners
	,detach title
	,before upper = \tcbtitle\par\smallskip
	,coltitle = myg!85!black
	,fonttitle = \bfseries\sffamily
	,description font = \mdseries
	,separator sign none
	,segmentation style={solid, myg!85!black}
}
{th}



%================================
% Exercise
%================================

\tcbuselibrary{theorems,skins,hooks}
\newtcbtheorem[number within=section]{Exercise}{Exercise}
{%
	enhanced,
	breakable,
	colback = myexercisebg,
	frame hidden,
	boxrule = 0sp,
	borderline west = {2pt}{0pt}{myexercisefg},
	sharp corners,
	detach title,
	before upper = \tcbtitle\par\smallskip,
	coltitle = myexercisefg,
	fonttitle = \bfseries\sffamily,
	description font = \mdseries,
	separator sign none,
	segmentation style={solid, myexercisefg},
}
{th}

\tcbuselibrary{theorems,skins,hooks}
\newtcbtheorem[number within=section]{exercise}{Exercise}
{%
	enhanced,
	breakable,
	colback = myexercisebg,
	frame hidden,
	boxrule = 0sp,
	borderline west = {2pt}{0pt}{myexercisefg},
	sharp corners,
	detach title,
	before upper = \tcbtitle\par\smallskip,
	coltitle = myexercisefg,
	fonttitle = \bfseries\sffamily,
	description font = \mdseries,
	separator sign none,
	segmentation style={solid, myexercisefg},
}
{th}

%================================
% EXAMPLE BOX
%================================

\newtcbtheorem[number within=section]{Example}{Example}
{%
	colback = myexamplebg
	,breakable
	,colframe = myexamplefr
	,coltitle = myexampleti
	,boxrule = 1pt
	,sharp corners
	,detach title
	,before upper=\tcbtitle\par\smallskip
	,fonttitle = \bfseries
	,description font = \mdseries
	,separator sign none
	,description delimiters parenthesis
}
{ex}

\newtcbtheorem[number within=section]{example}{Example}
{%
	colback = myexamplebg
	,breakable
	,colframe = myexamplefr
	,coltitle = myexampleti
	,boxrule = 1pt
	,sharp corners
	,detach title
	,before upper=\tcbtitle\par\smallskip
	,fonttitle = \bfseries
	,description font = \mdseries
	,separator sign none
	,description delimiters parenthesis
}
{ex}

%================================
% DEFINITION BOX
%================================

\newtcbtheorem[number within=section]{Definition}{Definition}{enhanced,
	before skip=2mm,after skip=2mm, colback=red!5,colframe=red!80!black,boxrule=0.5mm,
	attach boxed title to top left={xshift=1cm,yshift*=1mm-\tcboxedtitleheight}, varwidth boxed title*=-3cm,
	boxed title style={frame code={
					\path[fill=tcbcolback]
					([yshift=-1mm,xshift=-1mm]frame.north west)
					arc[start angle=0,end angle=180,radius=1mm]
					([yshift=-1mm,xshift=1mm]frame.north east)
					arc[start angle=180,end angle=0,radius=1mm];
					\path[left color=tcbcolback!60!black,right color=tcbcolback!60!black,
						middle color=tcbcolback!80!black]
					([xshift=-2mm]frame.north west) -- ([xshift=2mm]frame.north east)
					[rounded corners=1mm]-- ([xshift=1mm,yshift=-1mm]frame.north east)
					-- (frame.south east) -- (frame.south west)
					-- ([xshift=-1mm,yshift=-1mm]frame.north west)
					[sharp corners]-- cycle;
				},interior engine=empty,
		},
	fonttitle=\bfseries,
	title={#2},#1}{def}
\newtcbtheorem[number within=section]{definition}{Definition}{enhanced,
	before skip=2mm,after skip=2mm, colback=red!5,colframe=red!80!black,boxrule=0.5mm,
	attach boxed title to top left={xshift=1cm,yshift*=1mm-\tcboxedtitleheight}, varwidth boxed title*=-3cm,
	boxed title style={frame code={
					\path[fill=tcbcolback]
					([yshift=-1mm,xshift=-1mm]frame.north west)
					arc[start angle=0,end angle=180,radius=1mm]
					([yshift=-1mm,xshift=1mm]frame.north east)
					arc[start angle=180,end angle=0,radius=1mm];
					\path[left color=tcbcolback!60!black,right color=tcbcolback!60!black,
						middle color=tcbcolback!80!black]
					([xshift=-2mm]frame.north west) -- ([xshift=2mm]frame.north east)
					[rounded corners=1mm]-- ([xshift=1mm,yshift=-1mm]frame.north east)
					-- (frame.south east) -- (frame.south west)
					-- ([xshift=-1mm,yshift=-1mm]frame.north west)
					[sharp corners]-- cycle;
				},interior engine=empty,
		},
	fonttitle=\bfseries,
	title={#2},#1}{def}



%================================
% Solution BOX
%================================

\makeatletter
\newtcbtheorem{question}{Question}{enhanced,
	breakable,
	colback=white,
	colframe=myb!80!black,
	attach boxed title to top left={yshift*=-\tcboxedtitleheight},
	fonttitle=\bfseries,
	title={#2},
	boxed title size=title,
	boxed title style={%
			sharp corners,
			rounded corners=northwest,
			colback=tcbcolframe,
			boxrule=0pt,
		},
	underlay boxed title={%
			\path[fill=tcbcolframe] (title.south west)--(title.south east)
			to[out=0, in=180] ([xshift=5mm]title.east)--
			(title.center-|frame.east)
			[rounded corners=\kvtcb@arc] |-
			(frame.north) -| cycle;
		},
	#1
}{def}
\makeatother

%================================
% SOLUTION BOX
%================================

\makeatletter
\newtcolorbox{solution}{enhanced,
	breakable,
	colback=white,
	colframe=myg!80!black,
	attach boxed title to top left={yshift*=-\tcboxedtitleheight},
	title=Solution,
	boxed title size=title,
	boxed title style={%
			sharp corners,
			rounded corners=northwest,
			colback=tcbcolframe,
			boxrule=0pt,
		},
	underlay boxed title={%
			\path[fill=tcbcolframe] (title.south west)--(title.south east)
			to[out=0, in=180] ([xshift=5mm]title.east)--
			(title.center-|frame.east)
			[rounded corners=\kvtcb@arc] |-
			(frame.north) -| cycle;
		},
}
\makeatother

%================================
% Question BOX
%================================

\makeatletter
\newtcbtheorem{qstion}{Question}{enhanced,
	breakable,
	colback=white,
	colframe=mygr,
	attach boxed title to top left={yshift*=-\tcboxedtitleheight},
	fonttitle=\bfseries,
	title={#2},
	boxed title size=title,
	boxed title style={%
			sharp corners,
			rounded corners=northwest,
			colback=tcbcolframe,
			boxrule=0pt,
		},
	underlay boxed title={%
			\path[fill=tcbcolframe] (title.south west)--(title.south east)
			to[out=0, in=180] ([xshift=5mm]title.east)--
			(title.center-|frame.east)
			[rounded corners=\kvtcb@arc] |-
			(frame.north) -| cycle;
		},
	#1
}{def}
\makeatother

\newtcbtheorem[number within=section]{wconc}{Wrong Concept}{
	breakable,
	enhanced,
	colback=white,
	colframe=myr,
	arc=0pt,
	outer arc=0pt,
	fonttitle=\bfseries\sffamily\large,
	colbacktitle=myr,
	attach boxed title to top left={},
	boxed title style={
			enhanced,
			skin=enhancedfirst jigsaw,
			arc=3pt,
			bottom=0pt,
			interior style={fill=myr}
		},
	#1
}{def}



%================================
% NOTE BOX
%================================

\usetikzlibrary{arrows,calc,shadows.blur}
\tcbuselibrary{skins}
\newtcolorbox{note}[1][]{%
	enhanced jigsaw,
	colback=gray!20!white,%
	colframe=gray!80!black,
	size=small,
	boxrule=1pt,
	title=\textbf{Note:},
	halign title=flush center,
	coltitle=black,
	breakable,
	drop shadow=black!50!white,
	attach boxed title to top left={xshift=1cm,yshift=-\tcboxedtitleheight/2,yshifttext=-\tcboxedtitleheight/2},
	minipage boxed title=1.5cm,
	boxed title style={%
			colback=white,
			size=fbox,
			boxrule=1pt,
			boxsep=2pt,
			underlay={%
					\coordinate (dotA) at ($(interior.west) + (-0.5pt,0)$);
					\coordinate (dotB) at ($(interior.east) + (0.5pt,0)$);
					\begin{scope}
						\clip (interior.north west) rectangle ([xshift=3ex]interior.east);
						\filldraw [white, blur shadow={shadow opacity=60, shadow yshift=-.75ex}, rounded corners=2pt] (interior.north west) rectangle (interior.south east);
					\end{scope}
					\begin{scope}[gray!80!black]
						\fill (dotA) circle (2pt);
						\fill (dotB) circle (2pt);
					\end{scope}
				},
		},
	#1,
}


%%%%%%%%%%%%%%%%%%%%%%%%%%%%%%%%%%%%%%%%%%%
% TABLE OF CONTENTS
%%%%%%%%%%%%%%%%%%%%%%%%%%%%%%%%%%%%%%%%%%%

\usepackage{tikz}
\definecolor{doc}{RGB}{0,60,110}
\usepackage{titletoc}
\contentsmargin{0cm}
\titlecontents{chapter}[3.7pc]
{\addvspace{30pt}%
	\begin{tikzpicture}[remember picture, overlay]%
		\draw[fill=doc!60,draw=doc!60] (-7,-.1) rectangle (-0.5,.5);%
		\pgftext[left,x=-3.5cm,y=0.2cm]{\color{white}\Large\sc\bfseries Chapter\ \thecontentslabel};%
	\end{tikzpicture}\color{doc!60}\large\sc\bfseries}%
{}
{}
{\;\titlerule\;\large\sc\bfseries Page \thecontentspage
	\begin{tikzpicture}[remember picture, overlay]
		\draw[fill=doc!60,draw=doc!60] (2pt,0) rectangle (4,0.1pt);
	\end{tikzpicture}}%
\titlecontents{section}[3.7pc]
{\addvspace{2pt}}
{\contentslabel[\thecontentslabel]{2pc}}
{}
{\hfill\small \thecontentspage}
[]
\titlecontents{subsection}[3.7pc]
{\addvspace{-1pt}\small\qquad}
{\contentslabel[\thecontentslabel]{2pc}}
{}
{\ --- \small\thecontentspage}
[]

\makeatletter
\renewcommand{\tableofcontents}{%
	\chapter*{%
	  \vspace*{-20\p@}%
	  \begin{tikzpicture}[remember picture, overlay]%
		  \pgftext[right,x=15cm,y=0.2cm]{\color{doc!60}\Huge\sc\bfseries \contentsname};%
		  \draw[fill=doc!60,draw=doc!60] (13,-.75) rectangle (20,1);%
		  \clip (13,-.75) rectangle (20,1);
		  \pgftext[right,x=15cm,y=0.2cm]{\color{white}\Huge\sc\bfseries \contentsname};%
	  \end{tikzpicture}}%
	\@starttoc{toc}}
\makeatother


%%%%%%%%%%%%%%%%%%%%%%%%%%%%%%
% SELF MADE COMMANDS
%%%%%%%%%%%%%%%%%%%%%%%%%%%%%%

\newcommand{\thm}[2]{\begin{Theorem}{#1}{}\setlength{\parindent}{0.5cm}#2\end{Theorem}}
\newcommand{\cor}[2]{\begin{Corollary}{#1}{}\setlength{\parindent}{0.5cm}#2\end{Corollary}}
\newcommand{\mlemma}[2]{\begin{Lemma}{#1}{}\setlength{\parindent}{0.5cm}#2\end{Lemma}}
\newcommand{\mprop}[2]{\begin{Prop}{#1}{}\setlength{\parindent}{0.5cm}#2\end{Prop}}
\newcommand{\clm}[2]{\begin{claim}{#1}{}\setlength{\parindent}{0.5cm}#2\end{claim}}
\newcommand{\wc}[2]{\begin{wconc}{#1}{}\setlength{\parindent}{1cm}#2\end{wconc}}
\newcommand{\thmcon}[1]{\begin{Theoremcon}{\setlength{\parindent}{0.5cm}#1}\end{Theoremcon}}
\newcommand{\ex}[2]{\begin{Example}{#1}{}\setlength{\parindent}{0.5cm}#2\end{Example}}
\newcommand{\dfn}[2]{\begin{Definition}[colbacktitle=red!75!black]{#1}{}\setlength{\parindent}{0.5cm}#2\end{Definition}}
\newcommand{\dfns}[2]{\begin{definition}[colbacktitle=red!75!black]{#1}{}\setlength{\parindent}{0.5cm}#2\end{definition}}
\newcommand{\qs}[2]{\begin{question}{#1}{}\setlength{\parindent}{0.5cm}#2\end{question}}
\newcommand{\sol}[1]{\begin{solution}{\setlength{\parindent}{0.5cm}#1}{}\end{solution}}
\newcommand{\pf}[2]{\begin{myproof}[#1]\setlength{\parindent}{0.5cm}#2\end{myproof}}
\newcommand{\nt}[1]{\begin{note}\setlength{\parindent}{0.5cm}#1\end{note}}

\newcommand*\circled[1]{\tikz[baseline=(char.base)]{
		\node[shape=circle,draw,inner sep=1pt] (char) {#1};}}
\newcommand\getcurrentref[1]{%
	\ifnumequal{\value{#1}}{0}
	{??}
	{\the\value{#1}}%
}
\newcommand{\getCurrentSectionNumber}{\getcurrentref{section}}
\newenvironment{myproof}[1][\proofname]{%
	\proof[\bfseries #1: ]%
}{\endproof}

\newcommand{\mclm}[2]{\begin{myclaim}[#1]#2\end{myclaim}}
\newenvironment{myclaim}[1][\claimname]{\proof[\bfseries #1: ]}{}

\newcounter{mylabelcounter}

\makeatletter
\newcommand{\setword}[2]{%
	\phantomsection
	#1\def\@currentlabel{\unexpanded{#1}}\label{#2}%
}
\makeatother




\tikzset{
	symbol/.style={
			draw=none,
			every to/.append style={
					edge node={node [sloped, allow upside down, auto=false]{$#1$}}}
		}
}




% % redefine matrix env to allow for alignment, use r as default
% \renewcommand*\env@matrix[1][r]{\hskip -\arraycolsep
%     \let\@ifnextchar\new@ifnextchar
%     \array{*\c@MaxMatrixCols #1}}


%\usepackage{framed}
%\usepackage{titletoc}
%\usepackage{etoolbox}
%\usepackage{lmodern}


%\patchcmd{\tableofcontents}{\contentsname}{\sffamily\contentsname}{}{}

%\renewenvironment{leftbar}
%{\def\FrameCommand{\hspace{6em}%
%		{\color{myyellow}\vrule width 2pt depth 6pt}\hspace{1em}}%
%	\MakeFramed{\parshape 1 0cm \dimexpr\textwidth-6em\relax\FrameRestore}\vskip2pt%
%}
%{\endMakeFramed}

%\titlecontents{chapter}
%[0em]{\vspace*{2\baselineskip}}
%{\parbox{4.5em}{%
%		\hfill\Huge\sffamily\bfseries\color{myred}\thecontentspage}%
%	\vspace*{-2.3\baselineskip}\leftbar\textsc{\small\chaptername~\thecontentslabel}\\\sffamily}
%{}{\endleftbar}
%\titlecontents{section}
%[8.4em]
%{\sffamily\contentslabel{3em}}{}{}
%{\hspace{0.5em}\nobreak\itshape\color{myred}\contentspage}
%\titlecontents{subsection}
%[8.4em]
%{\sffamily\contentslabel{3em}}{}{}  
%{\hspace{0.5em}\nobreak\itshape\color{myred}\contentspage}




\usepackage{amsfonts, amsmath, amssymb, amsthm, dsfont, mathtools}
\usepackage{enumitem}
\usepackage{graphicx}
\usepackage{setspace}
\usepackage{indentfirst}
\usepackage[margin=1in]{geometry}
\graphicspath{{./images/}}
\setstretch{1.15}

\newtheorem{thm}{Theorem}
\newtheorem{proposition}[thm]{Proposition}
\newtheorem{corollary}[thm]{Corollary}
\newtheorem{lemma}[thm]{Lemma}

\newcommand*{\Var}{\ensuremath{\mathrm{Var}}}
\newcommand*{\Cov}{\ensuremath{\mathrm{Cov}}}
\newcommand*{\Corr}{\ensuremath{\mathrm{Corr}}}
\newcommand*{\Bias}{\ensuremath{\mathrm{Bias}}}
\newcommand*{\MSE}{\ensuremath{\mathrm{MSE}}}

\newcommand*{\range}{\ensuremath{\mathrm{range}}\,}
\newcommand*{\spann}{\ensuremath{\mathrm{span}}\,}
\newcommand*{\nul}{\ensuremath{\mathrm{null}}\,}
\newcommand*{\dom}{\ensuremath{\mathrm{dom}}\,}

\newcommand*{\pdv}[3][]{\frac{\partial^{#1}}{\partial#3^{#1}}#2}
\renewcommand*{\implies}{\ensuremath{\Longrightarrow}}
\renewcommand*{\impliedby}{\ensuremath{\Longleftarrow}}
\renewcommand*{\l}{\left}
\renewcommand*{\r}{\right}
\renewcommand{\leq}{\leqslant}
\renewcommand{\geq}{\geqslant}

\newcommand*{\Z}{\ensuremath{\mathbb{Z}}}
\newcommand*{\Q}{\ensuremath{\mathbb{Q}}}
\newcommand*{\R}{\ensuremath{\mathbb{R}}}
\newcommand*{\F}{\ensuremath{\mathbb{F}}}
\newcommand*{\C}{\ensuremath{\mathbb{C}}}
\newcommand*{\N}{\ensuremath{\mathbb{N}}}
\newcommand*{\E}{\ensuremath{\mathds{E}}}
\renewcommand*{\P}{\ensuremath{\mathds{P}}}
\newcommand*{\p}{\ensuremath{\mathcal{P}}}
\renewcommand*{\L}{\mathcal{L}}


% deliminators
\DeclarePairedDelimiter{\abs}{\lvert}{\rvert}
\DeclarePairedDelimiter{\norm}{\|}{\|}
\DeclarePairedDelimiter{\inner}{\langle}{\rangle}
\DeclarePairedDelimiter{\ceil}{\lceil}{\rceil}
\DeclarePairedDelimiter{\floor}{\lfloor}{\rfloor}


\let\oldleq\leq % save them in case they're every wanted
\let\oldgeq\geq

\newcommand*{\img}[3][]{
    \begin{figure}[htb!]
         \centering
         \includegraphics[scale=#1]{#2}
         \caption{#3}
    \end{figure}
}
\newcommand*{\imgs}[5][]{
    \begin{figure}[htb]
        \qquad
        \begin{minipage}{.4\textwidth}
            \centering
            \includegraphics[scale=#1]{#2}
            \caption{#3}
        \end{minipage}    
        \qquad
        \begin{minipage}{.4\textwidth}
            \centering
            \includegraphics[scale=#1]{#4}
            \caption{#5}
        \end{minipage}
    \end{figure} 
}


\title{\Huge{MATH 185 Notes}\\\emph{Professor: Constantin Teleman}}
\author{\huge{Neo Lee}}
\date{\huge{Spring 2024}}

\begin{document}

\maketitle
\let\cleardoublepage\clearpage
\pdfbookmark[section]{\contentsname}{toc}
\tableofcontents

\chapter{First chapter}
\section{Lecture 1 (skipped)}
\section{Lecture 2}
\dfn{Integral powers of complex numbers from geomtery}{
	If $z=r\cdot e^{i\theta}$, so $\abs{z}=r$ and define $\arg{z}=\theta$, 
	then $z^n=r^n\cdot e^{in\theta}$.
	\nt{
		$\abs{z} = r$ can be shown by writing $e^{i\theta}=\cos\theta+i\sin\theta$ and taking absolute
		value. 
	}
	\nt{
		Because of our definition, integral powers of complex numbers can be easily visualized in polar
		coordinates. In particular, for $z=r\cdot e^{i\theta}$, $$\abs{z}=r^n\quad \t{and} \quad
		\arg{z^n} = n\theta.$$
	}
}
\ex{Map from $z\to z^2$}{
	Let $z=x+yi$, then $\re z$ is mapped from $x$ to $x^2$, while $\im z$ is mapped from $yi$ to
	$-y^2$.

	$$z=x+yi\mapsto x^2-y^2+2xyi.$$ Hence, $\re z^2 = x^2-y^2$ and $\im z = 2xy$. If we plot $z^2$
	on the real and imaginary axis, then it's a parabola.
}

\thm{De Moivre formula}{
	$$(\cos\theta+i\sin\theta)^n = \cos(n\theta)+i\sin(n\theta).$$
	\pf{Proof}{
		This falls immediately by taking $n^\t{}th$ power of $z=e^i\theta$ and following
		\emph{definition 1.1}.
	}
	\nt{
		By applying binomial expansion to the left hand side of De Moivre formula, we get the following
		consequences immediately:
		\begin{align*}
			\cos^n\theta - {n\choose2} \cos^{n-2}\theta\cdot\sin^2\theta +\cdots & = \cos(n\theta) \\
			i\l(n\cdot\cos^{n-1}\theta\cdot\sin\theta - {n\choose3}\cos^{n-3}\theta\cdot\sin^3\theta + \cdots \r) & = i\sin(n\theta).
		\end{align*}
		From the first equation, we can see $\cos^2\theta - \sin^2\theta = \cos(2\theta)$ is a
		special case of De Moivre formula. From the second equation, we can see
		$2\cos\theta\cdot\sin\theta = \sin(2\theta)$ is a special case as well.
	}
}

\dfn{$n^\text{th}$ roots of complex number}{
	The $n^\t{th}$ roots of complex number are the solutions $z$ to the equation $z^n=x$.
	\nt{
		We call the solutions $z$ to the equation $z^n=1$ the $n^\t{th}$ roots of unity.
	}
}

\mprop{There are $n^\t{th}$ distinct roots of unity}{
	There are $n^\t{th}$ distinct roots $z$ to the equation $z^n=1$, and they are 
	$$1,\cos\frac{2\pi}{n}+i\sin\frac{2\pi}{n}, \cos\frac{4\pi}{n}+i\sin\frac{4\pi}{n},\cdots,
	\cos\frac{(2n-2)\pi}{n} + i\sin\frac{(2n-2)\pi}{n}.$$
	Let $\xi_n = \cos\frac{2\pi}{n} + i\sin\frac{2\pi}{n}$, then the list is 
	$$1,\xi_n, \xi_n^2, \dots, \xi_n^{n-1}.$$
	\nt{
		The roots are essentially cutting the unit circle into $n$ parts, and the cuts lying on the unit
		circle are the roots.
	}
}

\cor{Linear factorization of $x^n-1$}{
	From the previous proposition, we have the linear factorization 
	$$x^n-1=(x-1)(x-\xi_n)(x-\xi_n^2)\cdots(x-\xi_n^{n-1}).$$
	\pf{Proof}{
		From Bezout's theorem, if $x=a$ is a solution to $P(x)=0$, then $(x-a)$ divides $P(x)$. Now
		from \emph{proposition 1.1}, we found all the roots to $x^n-1=0$, so we can linearly factor
		$x^n-1$. 
	}
}

\dfn{Primitive roots of unity}{
	An $n^\t{th}$ root of unity $\xi$ is called primitive if $\xi^n=1$ but $\xi^k\neq 1$ for all
	$0<k<n$. In other words $\xi$ is the $n^\t{th}$ primitive root of unity if $\xi$ raised to the
	power $n$ is the first time equating to 1.
}

\ex{$6^\t{th}$ roots of unity}{
	Consider the equation $z^6=1$ and let $\xi_n = \cos\frac{\pi}{3}+i\sin\frac{\pi}{3}$. From
	\emph{proposition 1.1}, we know all the roots are raising $\xi$ to the power until $5$. Then by
	drawing it geometrically as points on the unit circle, and raising power is equavalent to
	rotating along the circle, the $6^\t{th}$ primitive roots are $\xi$ and $\xi^5$.
}

\thm{$n^\t{th}$ roots of unity can be partition into divisors of $n$}{
	Let $\{d_1, \dots, d_k\}$ be the divisors of $n$, then the set of $n^\t{th}$ roots of unity
	can be partitioned into groups of primitive $d_k^\t{th}$ roots of unity.
}

\thm{Factorization of $x^n-1$ in integer coefficient terms}{
	Define the $d^\t{th}$ cyclotomic polynomial as $\Phi_d(x)$, which is an irreducible polynomial
	over rationals and with integer coefficients. Then, grouping the factors of
	$$x^n-1=(x-1)(x-\xi_n)\cdots(x-\xi_n^{n-1})$$ according to divisors of $n$ with primitive roots
	gives a factorization $$x^n-1=\prod_{d\vert n}\Phi_d(x).$$
}

\ex{Factorization of $x^3-1$ and $x^6-1$}{
	\begin{align*}
		x^3-1 & = \underbrace{(x-1)}_{\Phi_1(x)}\underbrace{(x^2+x+1)}_{\Phi_3(x)} \\
		x^6-1 & = \underbrace{(x-1)}_{\Phi_1(x)}\underbrace{(x+1)}_{\Phi_2(x)}
		\underbrace{(x^2+x+1)}_{\Phi_3(x)}\underbrace{(x^2-x+1)}_{\Phi_6(x)}.
	\end{align*}
}

\dfn{"Cyclotomic integers"}{
	Cyclotomic integers are $\Z[\xi_2,\xi_3,\dots] = \Z[\xi_n]$ for a fixed $n$. Equivalently, it's
	formed by adjoining all roots of units to $\Z$.
}

\section{Lecture 3 (skipped)}
\section{Lecture 4}
\subsection{Multi-variable real differentiation}
\nt{
	Naturally constructed functions will often have singularities, for example, the inversion 
	$$z\mapsto \frac{1}{z}.$$ As $z\to0$ in \C, $\abs{z}\to0$ in \R, so
	$\abs{z^{-1}}=\abs{z}^{-1}\to\infty$. This can be fixed by the extended complex plane $\C\cup
	\{\infty\}.$ 
}
\dfn{Real differentiation in $\R$}{
	Let $I\in\R$ be an open interval. A function $f:I\to\R$ is differentiable at a point $x_0\in I$
	if $$\lim_{x\to x_0}\frac{f(x)-f(x_0)}{x-x_0}$$ exists.
}
\ex{Motivating example of new defintion of $\R^2$ differentiation}{
	We say $f(x_0)+f'(x_0)(x-x_0)$ is the best linear appriximation to $f$ at point $x_0$, and 
	$$\frac{|f(x)-f(x_0)-f'(x_0)(x-x_0)|}{|x-x_0|}\to0\quad\t{as}\quad x\to x_0.$$
	\nt{
		With some knowledge of multivariable analysis, we can get this definition by rearranging the
		definition of multi-variable differentiation involving the Taylor remainder $R(x-x_0)$.
	}
}
\dfn{Real differentiation from $\R^2\to\R$}{
	Let $U\in\R^2$ be an open subset. A function $f:U\to\R$ is differentiable at $\inner{x_0,y_0}\in
	U$ if there exists a linear map $Df:\R^2\to\R$ such that 
	$$\frac{|f(\inner{x,y})-f(\inner{x_0,y_0})-Df(\inner{x-x_0,y-y_0})|}{\norm{\inner{x-x_0,y-y_0}}}\to
	0\quad\t{as}\quad\inner{x,y}\to\inner{x_0,y_0}.$$ 
	\nt{
		$Df$ is the derivative at $\inner{x_0,y_0}$, which gives the best linear approximation to $f$ 
		at $\inner{x_0,y_0}$.
	}
}
\dfn{Partial differentiation from $\R^2\to \R$}{
	Let $f:U\to\R$ where $U$ is an open subset of $\R^2$, then $$\pdv[]{f}{x}=\lim_{x\to
	x_0}\frac{f(\inner{x,y_0})-f(\inner{x_0,y_0})}{x-x_0}\quad\t{and}\quad\pdv[]{f}{y}=\lim_{x\to
	x_0}\frac{f(\inner{x_0,y})-f(\inner{x_0,y_0})}{y-y_0}.$$
	\nt{
	\begin{itemize}
		\item 
		This is reducible to the definition of real differentiation in $\R$ by fixing $x=x_0$ or
		$y=y_0$.

		\item
		$Df$ as the derivative at $\inner{x_0,y_0}$ in this case is given by the (matrix)
		multiplication by the vector $\inner{\pdv[]{f}{x}, \pdv[]{f}{y}}$.
	\end{itemize}	
	}
}
\dfn{Differentiation from $\R^2$ to $\R^2$}{
	Obvious adoptation of the previous definition, where $f$ is a vector valued function, say
	$\R^2\to\R^2$. If $f=\begin{bmatrix}
		f_1 \\ f_2
	\end{bmatrix}$, $Df$ is the Jacobian matric $\begin{bmatrix}
		\pdv[]{f_1}{x} & \pdv[]{f_1}{y} \\
		\pdv[]{f_2}{x} & \pdv[]{f_2}{y}
	\end{bmatrix}$.
}
\thm{Sufficient condition for total differentiability}{
	If all partial derivatives of $f$ exist and are continuous in a neighborhood of
	$\inner{x_0,y_0}$, then $f$ is totally differentiable at $\inner{x_0,y_0}$.
}
\nt{
	For limits and convergence questions in \C, we can consider it as limits and convergence in
	$\R^2$. However, note that we cannot treat differentiability in $\C$ directly as
	differentiability in $\R^2$, which we will later prove that there are extra conditions.
}
\nt{
	Addition and multiplication of complexnumbesr are continuous functions $\C\times\C\to\C$, and
	inversion map $z\to z^{-1}=\frac{z}{\abs{z}^2}$ is also continuous away from $z=0$.
}

\subsection{Complex differentiation}
\dfn{Complex differentiable}{
	$f$ is complex differentiable at $z_0$ if $$\lim_{z\to z_0}\frac{f(z)-f(z_0)}{z-z_0}$$ exists.
	\nt{ 
		The limit is the (complex) derivative of $f$ at $z_0, f'(z_0)$.
	}
}
\thm{Complex differentiability implies continuity}{
	If $f$ is complex differentiable at $x_0$, then $f$ is continuous at $z_0$.
	\pf{Proof}{
		\begin{align*}
			\lim_{z\to z_0}f(x)-f(x_0) & = \lim_{z\to z_0}\l(f(x)-f(x_0)\r)\frac{z-z_0}{z-z_0} \\
			& = \lim_{z\to z_0}\frac{f(z)-f(z_0)}{z-z_0} \cdot \lim_{z\to z_0}z-z_0 \\
			& = f'(z_0)\cdot 0 = 0.
		\end{align*}
	}
}
\ex{Derivative of monomial}{
	Let $f(z)=z^n$, then notice $$\frac{f(z)-f(z_0)}{z-z_0}=\frac{z^n-z_0^n}{z-z_0}=z^{n-1}+z^{n-2}z_0
	+\cdots+z_0^{n-1}.$$ Since addition and multiplication are continuous on $\C$ and $\lim_{z\to z_0}z =
	z_0$, the limit of the RHS is $nz_0^{n-1}$. Hence, $f'(z_0)=nz_0^{n-1}$.
}
\thm{Properties of complex differentiation}{
	If $f,g$ are complex differentiable at $z_0$, then
	\begin{enumerate}
		\item $f+g$ is complex differentiable at $z_0$ and $(f+g)'(z_0)=f'(z_0)+g'(z_0)$.
		\item $fg$ is complex differentiable at $z_0$ and $(fg)'(z_0)=f'(z_0)g(z_0)+f(z_0)g'(z_0)$.
		\item If $g(z_0)\neq 0$, then $\frac{f}{g}$ is complex differentiable at $z_0$ and
			$$\l(\frac{f}{g}\r)'(z_0)=\frac{f'(z_0)g(z_0)-f(z_0)g'(z_0)}{g(z_0)^2}.$$
	\end{enumerate}
	\pf{Proof}{
		The same proof as in $\R$ using absolute value of complex numbers and triangle inequality.
	}
}
\cor{Complex differentiable of rational functions}{
	Let 
	\[
		z = \frac{P(z)}{Q(z)} \quad\t{for some polynomials } P,Q \t{ and } Q \neq 0.
	\]
	We can omplex differentiate $z$ using the same algebraic manipulation as in $\R$. 
	\pf{Proof}{
		From \emph{example 1.4.2}, we know how to differentiate monomials just like in the real case,
		and from \emph{theorem 1.4.3}, we can differentiate polynomials just like in the real case by
		linear combination of monomials. Hence, we can differentiate $P(z)$ and $Q(z)$, and then apply
		\emph{theorem 1.4.3 (3)}.
	}
}
\ex{$f=\frac{1}{z}$}{
	$f'(z)=\frac{-1}{z^2}$ when $z\neq0$.
}
\dfn{Holomorphic}{
	$f:U\to\C$ is holomorphic if it is complex differentiable at every point in the open set
	$U\subseteq\C$.
}
\subsection{Complex vs real differentiability}
\mprop{Complex differentiabilit implies $\R^2$ differentiability}{
	Consider complex number $z=x+yi$ and $f(z)=u(x,y)+iv(x,y)$.
	If $f$ is complex differentiable at $z_0=x_0+y_0i$, then the map 
	\[
		\begin{bmatrix}
			u(x,y) \\ v(x,y)
		\end{bmatrix}: \R^2\to\R^2
	\]
	is real differentiable at $\inner{x_0,y_0}$.
	\pf{Proof}{
		If $f$ is differentiable at $z_0$, then 
		\[
			\lim_{z\to z_0}\frac{f(z)-f(z_0)-f'(z_0)(z-z_0)}{z-z_0} = 0,
		\]
		which is equivalent to 
		\[
			\lim_{z\to z_0}\frac{|f(z)-f(z_0)-f'(z_0)(z-z_0)|}{|z-z_0|} = 0.
		\]
		Now notice that in face $f'(z_0)$ is a linear homogeneous function from $\C\to\C$. Again if
		we consider $z_0=x_0+y_0i$, then $f'(z_0)$ is a linear function of $\inner{x_0,y_0}$ from
		$\R^2\to\R^2$. Hence, we have found the derivative $D\begin{bmatrix}
			u(x_0,y_0) \\ v(x_0,y_0)
		\end{bmatrix}$, which is $f'(z_0)$ when writing in vector $\R^2$ form.
		\nt{
			Since $f'(z_0)$ is a linear function from $\C\to\C$, it is simply scaling $z_0$ by a complex 
			number. if $f'(z_0) = (p + qi)z_0$, then $\re f'(z_0) = p\cdot\re z_0 - q\cdot\im z_0$ and
			$\im f'(z_0)=p\cdot\im z_0 + q\cdot\re z_0$. Writing it in matrix form and representing
			complex numbers as vectors in $\R^2$, we get the matrix 
			$
				\begin{bmatrix}
					p & -q \\
					q & p
				\end{bmatrix}.
			$
		}
	}	
}
\thm{Equivalence of complex and $\R^2$ differentiability}{
	\[
		(z=x+yi)\mapsto (f(z)=w=u(x,y)+v(x,y)i)
	\] is complex differentiable at $z_0 = x_0+y_0i$ if and only if the map 
	\[
		\begin{bmatrix}
			x \\ y
		\end{bmatrix}\mapsto 
		\begin{bmatrix}
			u(x,y) \\ v(x,y)
		\end{bmatrix}
	\] is real differentiable at $(x_0,y_0)$ AND satisfies the Cauchy-Reimann equations 
	\[
		u_x = v_y,\quad v_x = -u_y
	\] at $(x_0,y_0)$.
}
\ex{Exponential}{
	The complex exponential $f:z\mapsto e^z$ is holomorphic in \C.
	\pf{Proof}{
		\[
			f:z\mapsto e^z \iff f:(x+yi)\mapsto e^x(\cos y + i\sin y).	
		\]
		If we write it in $\R^2$, then 
		\[
			g:\begin{bmatrix}
				x \\ y
			\end{bmatrix}\mapsto 
			\begin{bmatrix}
				e^x\cos y \\ e^x\sin y
			\end{bmatrix}.
		\] We know the function is differentiable in $\R^2$ because the partial derivatives exist and 
		are continuous. In particular, the derivative is 
		\[
			Dg: \begin{bmatrix}
				x \\ y
			\end{bmatrix}\mapsto 
			\begin{bmatrix}
				e^x\cos y & -e^x\sin y \\
				e^x\sin y & e^x\cos y
			\end{bmatrix}.	
		\]
		Indeed, the Cauchy-Reimann equations are satisfied, and hence $f$ is real differentiable in
		$\R^2$, hence complex differentiable in \C, which is the definition of holomorphic.
	}
}
\ex{Other holomorphic / non-holomorphic examples}{
	\begin{itemize}
		\item $(x,y)\mapsto\l(x^2+y^2,2xy\r)$ \quad  non-holomorphic
		\item $(x,y)\mapsto\l(x^2-y^2,2xy\r)$
		\item $(x,y)\mapsto\l(\frac{1}{2}\log\l(x^2+y^2\r),\arctan\frac{y}{x}\r)$
	\end{itemize}
	\pf{Proof}{
		Conside as $U\to\R^2:U\subseteq\R^2$ map and check that the Jacobian matrix satisfies the 
		Cauchy-Reimann equations.
	}
}
\subsection{Some general properties of holomorphic functions}
\mprop{Set of holomorphic maps is an algebra}{
	The set of holomorphic maps $U\to\C$ is an algebra:
	\begin{itemize}
		\item If $f,g$ are holomorphic, then $f+g$ is holomorphic.
		\item If $k\in\C$, then $kf$ is holomorphic.
		\item If $f,g$ are holomorphic, then $fg$ is holomorphic.
		\item If $f,g$ are holomorphic and $g(z)\neq 0$ for all $z\in U$, then $\frac{f}{g}$ is
			holomorphic.
	\end{itemize}
}
\thm{Holomorphic function implies holomorphic derivative}{
	If $f:U\to\C$ is holomorphic and twice real differentiable, then $f'$ is also holomorphic.
	\nt{
		In fact, the assumption of twice real differentiable can be omitted because holomorphic maps are
		infinitely differentiable.
	}
	\pf{Proof}{
		Let $f(x+yi)=u(x,y)+v(x,y)i$ where $u,v:U\to\R$. Then the Jacobian matrix of $f$ is
		\[
			Df = \begin{bmatrix}
					u_x & u_y \\
					v_x & v_y
			\end{bmatrix} = 
			\begin{bmatrix}
				u_x & -v_x \\
				v_x & u_x
			\end{bmatrix}
		\]
		Or we can write $Df=(u_x-v_x) + (v_x+u_x)i$, then the Hessian matrix (derivative of $f'$) of
		$f$ is
		\[
			D^2f = \begin{bmatrix}
				u_{xx}-v_{xx} & u_{xy}-v_{xy} \\
				v_{xx}+u_{xx} & v_{xy}+u_{xy}
			\end{bmatrix} = 
			\begin{bmatrix}
				u_{xx}-v_{xx} & \overbrace{u_{yx} - v_{yx}}^{\t{Clairaut's}} \\
				u_{xx}+v_{xx} & \underbrace{v_{yx} + u_{yx}}_{\t{Clairaut's}}
			\end{bmatrix} = 
			\begin{bmatrix}
				u_{xx}-v_{xx} & \overbrace{-u_{xx} - v_{xx}}^{u_x = v_y,v_x = -u_y} \\
				u_{xx}+v_{xx} & \underbrace{u_{xx} - v_{xx}}_{u_x = v_y,v_x = -u_y}
			\end{bmatrix}.
		\]
		Hence, we have shown that for $D^2f=f''$ is also real differentiable satisfying the 
		Cauchy-Reimann equations. Hence, $f'$ is holomorphic.
	}
}






\chapter{Starting a new chapter}
\section{Demo of commands}
\dfn{Some defintion}{
	yap
}
\qs{Some question}{yap}
\sol{
	\pf{Some proof}{yap}
}
\nt{Some note}
\thm{Some theorem}{
	yap
}
\wc{Some wrong concept}{
	yap
}
\mlemma{Some lemma}{
	yap
}
\mprop{Some proposition}{
	yap
}
\ex{Some example}{
	yap
}
\clm{Some claim}{
	yap
}
\cor{Some corollary}{
	yap
}
\thmcon{Some unlabeled theorem}

This is a new paragraph


\begin{algorithm}
	\caption{Some algorithm}
	\KwIn{input}
	\KwOut{output}
	\SetAlgoLined
	\SetNoFillComment
	\tcc{This is a comment}
	This is first line \tcp*{This is also a comment}
	\uIf{$x > 5$} {
		do nothing
	} \uElseIf {$x < 5$} {
		do nothing
	} \Else {
		do nothing
	}
	\While{$x == 5$}{
		still do nothing
	}
	\ForEach{$x = 1:5$}{
		do nothing
	}
	\Return{return nothing}
\end{algorithm}


\end{document}
