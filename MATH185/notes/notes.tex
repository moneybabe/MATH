\documentclass{book}
\input{preamble.tex}
\newcommand*{\Var}{\ensuremath{\mathrm{Var}}}
\newcommand*{\Cov}{\ensuremath{\mathrm{Cov}}}
\newcommand*{\Corr}{\ensuremath{\mathrm{Corr}}}
\newcommand*{\Bias}{\ensuremath{\mathrm{Bias}}}
\newcommand*{\MSE}{\ensuremath{\mathrm{MSE}}}

\newcommand*{\range}{\ensuremath{\mathrm{range}}\,}
\newcommand*{\spann}{\ensuremath{\mathrm{span}}\,}
\newcommand*{\nul}{\ensuremath{\mathrm{null}}\,}
\newcommand*{\dom}{\ensuremath{\mathrm{dom}}\,}

\newcommand*{\pdv}[3][]{\frac{\partial^{#1}}{\partial#3^{#1}}#2}
\renewcommand*{\implies}{\ensuremath{\Longrightarrow}}
\renewcommand*{\impliedby}{\ensuremath{\Longleftarrow}}
\renewcommand*{\l}{\left}
\renewcommand*{\r}{\right}
\renewcommand{\leq}{\leqslant}
\renewcommand{\geq}{\geqslant}
\newcommand*{\tb}{\textbf}
\renewcommand*{\t}{\text}

\newcommand*{\Z}{\ensuremath{\mathbb{Z}}}
\newcommand*{\Q}{\ensuremath{\mathbb{Q}}}
\newcommand*{\R}{\ensuremath{\mathbb{R}}}
\newcommand*{\F}{\ensuremath{\mathbb{F}}}
\newcommand*{\C}{\ensuremath{\mathbb{C}}}
\newcommand*{\N}{\ensuremath{\mathbb{N}}}
\newcommand*{\E}{\ensuremath{\mathds{E}}}
\renewcommand*{\P}{\ensuremath{\mathds{P}}}
\newcommand*{\p}{\ensuremath{\mathcal{P}}}
\renewcommand*{\L}{\mathcal{L}}


% deliminators
\DeclarePairedDelimiter{\abs}{\lvert}{\rvert}
\DeclarePairedDelimiter{\norm}{\|}{\|}
\DeclarePairedDelimiter{\inner}{\langle}{\rangle}
\DeclarePairedDelimiter{\ceil}{\lceil}{\rceil}
\DeclarePairedDelimiter{\floor}{\lfloor}{\rfloor}
\DeclarePairedDelimiter{\round}{\lfloor}{\rceil}

\let\oldleq\leq % save them in case they're every wanted
\let\oldgeq\geq

\newcommand*{\img}[3][]{
    \begin{figure}[htb!]
         \centering
         \includegraphics[scale=#1]{#2}
         \caption{#3}
    \end{figure}
}
\newcommand*{\imgs}[5][]{
    \begin{figure}[htb]
        \qquad
        \begin{minipage}{.4\textwidth}
            \centering
            \includegraphics[scale=#1]{#2}
            \caption{#3}
        \end{minipage}    
        \qquad
        \begin{minipage}{.4\textwidth}
            \centering
            \includegraphics[scale=#1]{#4}
            \caption{#5}
        \end{minipage}
    \end{figure} 
}


\title{\Huge{MATH 185 Notes}\\\emph{Professor: Constantin Teleman}}
\author{\huge{Neo Lee}}
\date{\huge{Spring 2024}}

\begin{document}

\maketitle
\let\cleardoublepage\clearpage
\pdfbookmark[section]{\contentsname}{toc}
\tableofcontents

\chapter{First chapter}
\section{Lecture 1 (skipped)}
\section{Lecture 2}
\dfn{Integral powers of complex numbers from geomtery}{
	If $z=r\cdot e^{i\theta}$, so $\abs{z}=r$ and define $\arg{z}=\theta$, 
	then $z^n=r^n\cdot e^{in\theta}$.
	\nt{
		$\abs{z} = r$ can be shown by writing $e^{i\theta}=\cos\theta+i\sin\theta$ and taking absolute
		value. 
	}
	\nt{
		Because of our definition, integral powers of complex numbers can be easily visualized in polar
		coordinates. In particular, for $z=r\cdot e^{i\theta}$, $$\abs{z}=r^n\quad \t{and} \quad
		\arg{z^n} = n\theta.$$
	}
}
\ex{Map from $z\to z^2$}{
	Let $z=x+yi$, then $\re z$ is mapped from $x$ to $x^2$, while $\im z$ is mapped from $yi$ to
	$-y^2$.

	$$z=x+yi\mapsto x^2-y^2+2xyi.$$ Hence, $\re z^2 = x^2-y^2$ and $\im z = 2xy$. If we plot $z^2$
	on the real and imaginary axis, then it's a parabola.
}

\thm{De Moivre formula}{
	$$(\cos\theta+i\sin\theta)^n = \cos(n\theta)+i\sin(n\theta).$$
	\pf{Proof}{
		This falls immediately by taking $n^\t{}th$ power of $z=e^i\theta$ and following
		\emph{definition 1.1}.
	}
	\nt{
		By applying binomial expansion to the left hand side of De Moivre formula, we get the following
		consequences immediately:
		\begin{align*}
			\cos^n\theta - {n\choose2} \cos^{n-2}\theta\cdot\sin^2\theta +\cdots & = \cos(n\theta) \\
			i\l(n\cdot\cos^{n-1}\theta\cdot\sin\theta - {n\choose3}\cos^{n-3}\theta\cdot\sin^3\theta + \cdots \r) & = i\sin(n\theta).
		\end{align*}
		From the first equation, we can see $\cos^2\theta - \sin^2\theta = \cos(2\theta)$ is a
		special case of De Moivre formula. From the second equation, we can see
		$2\cos\theta\cdot\sin\theta = \sin(2\theta)$ is a special case as well.
	}
}

\dfn{$n^\text{th}$ roots of complex number}{
	The $n^\t{th}$ roots of complex number are the solutions $z$ to the equation $z^n=x$.
	\nt{
		We call the solutions $z$ to the equation $z^n=1$ the $n^\t{th}$ roots of unity.
	}
}

\mprop{There are $n^\t{th}$ distinct roots of unity}{
	There are $n^\t{th}$ distinct roots $z$ to the equation $z^n=1$, and they are 
	$$1,\cos\frac{2\pi}{n}+i\sin\frac{2\pi}{n}, \cos\frac{4\pi}{n}+i\sin\frac{4\pi}{n},\cdots,
	\cos\frac{(2n-2)\pi}{n} + i\sin\frac{(2n-2)\pi}{n}.$$
	Let $\xi_n = \cos\frac{2\pi}{n} + i\sin\frac{2\pi}{n}$, then the list is 
	$$1,\xi_n, \xi_n^2, \dots, \xi_n^{n-1}.$$
	\nt{
		The roots are essentially cutting the unit circle into $n$ parts, and the cuts lying on the unit
		circle are the roots.
	}
}

\cor{Linear factorization of $x^n-1$}{
	From the previous proposition, we have the linear factorization 
	$$x^n-1=(x-1)(x-\xi_n)(x-\xi_n^2)\cdots(x-\xi_n^{n-1}).$$
	\pf{Proof}{
		From Bezout's theorem, if $x=a$ is a solution to $P(x)=0$, then $(x-a)$ divides $P(x)$. Now
		from \emph{proposition 1.1}, we found all the roots to $x^n-1=0$, so we can linearly factor
		$x^n-1$. 
	}
}

\dfn{Primitive roots of unity}{
	An $n^\t{th}$ root of unity $\xi$ is called primitive if $\xi^n=1$ but $\xi^k\neq 1$ for all
	$0<k<n$. In other words $\xi$ is the $n^\t{th}$ primitive root of unity if $\xi$ raised to the
	power $n$ is the first time equating to 1.
}

\ex{$6^\t{th}$ roots of unity}{
	Consider the equation $z^6=1$ and let $\xi_n = \cos\frac{\pi}{3}+i\sin\frac{\pi}{3}$. From
	\emph{proposition 1.1}, we know all the roots are raising $\xi$ to the power until $5$. Then by
	drawing it geometrically as points on the unit circle, and raising power is equavalent to
	rotating along the circle, the $6^\t{th}$ primitive roots are $\xi$ and $\xi^5$.
}

\thm{$n^\t{th}$ roots of unity can be partition into divisors of $n$}{
	Let $\{d_1, \dots, d_k\}$ be the divisors of $n$, then the set of $n^\t{th}$ roots of unity
	can be partitioned into groups of primitive $d_k^\t{th}$ roots of unity.
}

\thm{Factorization of $x^n-1$ in integer coefficient terms}{
	Define the $d^\t{th}$ cyclotomic polynomial as $\Phi_d(x)$, which is an irreducible polynomial
	over rationals and with integer coefficients. Then, grouping the factors of
	$$x^n-1=(x-1)(x-\xi_n)\cdots(x-\xi_n^{n-1})$$ according to divisors of $n$ with primitive roots
	gives a factorization $$x^n-1=\prod_{d\vert n}\Phi_d(x).$$
}

\ex{Factorization of $x^3-1$ and $x^6-1$}{
	\begin{align*}
		x^3-1 & = \underbrace{(x-1)}_{\Phi_1(x)}\underbrace{(x^2+x+1)}_{\Phi_3(x)} \\
		x^6-1 & = \underbrace{(x-1)}_{\Phi_1(x)}\underbrace{(x+1)}_{\Phi_2(x)}
		\underbrace{(x^2+x+1)}_{\Phi_3(x)}\underbrace{(x^2-x+1)}_{\Phi_6(x)}.
	\end{align*}
}

\dfn{"Cyclotomic integers"}{
	Cyclotomic integers are $\Z[\xi_2,\xi_3,\dots] = \Z[\xi_n]$ for a fixed $n$. Equivalently, it's
	formed by adjoining all roots of units to $\Z$.
}








\chapter{Starting a new chapter}
\section{Demo of commands}
\dfn{Some defintion}{
	yap
}
\qs{Some question}{yap}
\sol{
	\pf{Some proof}{yap}
}
\nt{Some note}
\thm{Some theorem}{
	yap
}
\wc{Some wrong concept}{
	yap
}
\mlemma{Some lemma}{
	yap
}
\mprop{Some proposition}{
	yap
}
\ex{Some example}{
	yap
}
\clm{Some claim}{
	yap
}
\cor{Some corollary}{
	yap
}
\thmcon{Some unlabeled theorem}

This is a new paragraph


\begin{algorithm}
	\caption{Some algorithm}
	\KwIn{input}
	\KwOut{output}
	\SetAlgoLined
	\SetNoFillComment
	\tcc{This is a comment}
	This is first line \tcp*{This is also a comment}
	\uIf{$x > 5$} {
		do nothing
	} \uElseIf {$x < 5$} {
		do nothing
	} \Else {
		do nothing
	}
	\While{$x == 5$}{
		still do nothing
	}
	\ForEach{$x = 1:5$}{
		do nothing
	}
	\Return{return nothing}
\end{algorithm}


\end{document}
