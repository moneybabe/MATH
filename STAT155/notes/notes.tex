\documentclass{book}
\input{preamble.tex}
\newcommand*{\Var}{\ensuremath{\mathrm{Var}}}
\newcommand*{\Cov}{\ensuremath{\mathrm{Cov}}}
\newcommand*{\Corr}{\ensuremath{\mathrm{Corr}}}
\newcommand*{\Bias}{\ensuremath{\mathrm{Bias}}}
\newcommand*{\MSE}{\ensuremath{\mathrm{MSE}}}

\newcommand*{\range}{\ensuremath{\mathrm{range}}\,}
\newcommand*{\spann}{\ensuremath{\mathrm{span}}\,}
\newcommand*{\nul}{\ensuremath{\mathrm{null}}\,}
\newcommand*{\dom}{\ensuremath{\mathrm{dom}}\,}

\newcommand*{\pdv}[3][]{\frac{\partial^{#1}}{\partial#3^{#1}}#2}
\renewcommand*{\implies}{\ensuremath{\Longrightarrow}}
\renewcommand*{\impliedby}{\ensuremath{\Longleftarrow}}
\renewcommand*{\l}{\left}
\renewcommand*{\r}{\right}
\renewcommand{\leq}{\leqslant}
\renewcommand{\geq}{\geqslant}
\newcommand*{\tb}{\textbf}
\renewcommand*{\t}{\text}

\newcommand*{\Z}{\ensuremath{\mathbb{Z}}}
\newcommand*{\Q}{\ensuremath{\mathbb{Q}}}
\newcommand*{\R}{\ensuremath{\mathbb{R}}}
\newcommand*{\F}{\ensuremath{\mathbb{F}}}
\newcommand*{\C}{\ensuremath{\mathbb{C}}}
\newcommand*{\N}{\ensuremath{\mathbb{N}}}
\newcommand*{\E}{\ensuremath{\mathds{E}}}
\renewcommand*{\P}{\ensuremath{\mathds{P}}}
\newcommand*{\p}{\ensuremath{\mathcal{P}}}
\renewcommand*{\L}{\mathcal{L}}


% deliminators
\DeclarePairedDelimiter{\abs}{\lvert}{\rvert}
\DeclarePairedDelimiter{\norm}{\|}{\|}
\DeclarePairedDelimiter{\inner}{\langle}{\rangle}
\DeclarePairedDelimiter{\ceil}{\lceil}{\rceil}
\DeclarePairedDelimiter{\floor}{\lfloor}{\rfloor}
\DeclarePairedDelimiter{\round}{\lfloor}{\rceil}

\let\oldleq\leq % save them in case they're every wanted
\let\oldgeq\geq

\newcommand*{\img}[3][]{
    \begin{figure}[htb!]
         \centering
         \includegraphics[scale=#1]{#2}
         \caption{#3}
    \end{figure}
}
\newcommand*{\imgs}[5][]{
    \begin{figure}[htb]
        \qquad
        \begin{minipage}{.4\textwidth}
            \centering
            \includegraphics[scale=#1]{#2}
            \caption{#3}
        \end{minipage}    
        \qquad
        \begin{minipage}{.4\textwidth}
            \centering
            \includegraphics[scale=#1]{#4}
            \caption{#5}
        \end{minipage}
    \end{figure} 
}


\title{\Huge{STAT 155 Notes}\\\emph{Professor: Adrian Gonzalez Casanova}}
\author{\huge{Neo Lee}}
\date{\huge{Spring 2024}}

\begin{document}

\maketitle
\let\cleardoublepage\clearpage
\pdfbookmark[section]{\contentsname}{toc}
\tableofcontents

\chapter{Combinatorial games}
\section{Lecture 1 (skipped)}
\section{Lecture 2 (skipped)}
\section{Lecture 3}
\dfn{Combinatorial game}{
	A two-player game of perfect information where the players alternate moves and the game
	ends in a finite number of moves. At each position $v\in V$, there is a set of moves 
	$E_v$ available to the player whose turn it is. The game ends when $E_v = \emptyset$. 
	\nt{
		We can think of a combinatorial game as a directed graph $G = (V,E)$ where $V$ is the 
		set of positions and $E$ is the set of edges. The game starts at some vertex $v_0$ and 
		ends when $v_0$ has no outgoing edges.
	}
}
\dfn{Impartial game}{
	A game where the set of moves available to each player is the same at each position. The
	terminal reward to each player is the same. Players alternate moves.
}
\dfn{Partisan game}{
	The complement definition of an impartial game.
}
\ex{Impartial games}{
	\begin{enumerate}
		\item Tic-tac-toe
		\item Chomp 
		\item Subtraction games
		\item Nim
	\end{enumerate}
}
\dfn{Proressively bounded combinatorial game}{
	A combinatorial game is progressively bounded if for every starting position $v_0$, there is a
	finite bound $B(v_0)$ on the number of moves in the game before the game ends.
}
\dfn{Strategy}{
	Let $V_{NT}\subseteq V$ is the set of non-terminal positions. A strategy for a player is a function
	that assigns a legal move to each non-terminal position $v\in V_{NT}$ $$\sigma: V_{NT} \to
	\cup_{v\in V_{NT}} E_v.$$
}
\dfn{Winning strategy}{
	A winning strategy for a player from position $v\in V$ is a strategy that is guaranteed to result in
	a win for that player.
}
\thm{Existence of a winning strategy}{
	In a progressively bounded, impartial combinatorial game, $V=F\sqcup S$. That is, from any initial
	position, one of the players has a winning strategy.
	\nt{
		Denote $F, S$ as the wining positions for first and second player respectively.
	}
	\pf{Proof}{
		We prove this by induction on the finite bound $B(v)$ on the number of moves in the game before
		the game ends. If $B(v) = 0$, then the game ends immediately and $v$ is a winning position either
		for the first or second player. Hence, for $B(v) = 0$, the theorem holds.

		Now suppose that the theorem holds for all $B(v) < k$ and consider a game with $B(v) = k$.
		Let $v\in V$ be the initial position. Without loss of generality, suppose it is the first
		player's turn. The player can move along some edges $e\in E_v$ to $v'$. By inductive
		hypothesis, $V=F\sqcup S$ for $B(v)=k-1$. If $v'\in F$, then the first player has a winning
		strategy at $v$, else if $v'\in S$, then the second player has a winning strategy at $v$.
		Hence, for all $v$, either $v\in F$ or $v\in S$. This proves the theorem.
	}
}
\thm{Existence of winning strategy for first player in rectangular Chomp}{
	In a non-terminal rectangular Chomp game with $m\times n$ for board, the first player always
	has a winning strategy.

	\pf{Proof}{
		First player eats the top right corner to move to $v'$. Then, by \emph{Theorem 1.3.1}, the
		new position $v'$ is either a winning position for the first player or the second player.
		Suppose $v'\in S$, and the next move by the second player moves to $v''$. Then, in fact, the
		first player can choose to not eat the right top corner in the first move, but instead
		directly move to $v''$, which would make $v$ the winning position for the first player.
		\nt{
			\begin{itemize}
				\item 
				$v'$ is the rectangular board with the top right corner eaten.

				\item
				This is called a \emph{strategy stealing}.
			\end{itemize}
		}
	}
}












\chapter{Starting a new chapter}
\section{Demo of commands}
\dfn{Some defintion}{
	yap
}
\qs{Some question}{yap}
\sol{
	\pf{Some proof}{yap}
}
\nt{Some note}
\thm{Some theorem}{
	yap
}
\wc{Some wrong concept}{
	yap
}
\mlemma{Some lemma}{
	yap
}
\mprop{Some proposition}{
	yap
}
\ex{Some example}{
	yap
}
\clm{Some claim}{
	yap
}
\cor{Some corollary}{
	yap
}
\thmcon{Some unlabeled theorem}

This is a new paragraph


\begin{algorithm}
	\caption{Some algorithm}
	\KwIn{input}
	\KwOut{output}
	\SetAlgoLined
	\SetNoFillComment
	\tcc{This is a comment}
	This is first line \tcp*{This is also a comment}
	\uIf{$x > 5$} {
		do nothing
	} \uElseIf {$x < 5$} {
		do nothing
	} \Else {
		do nothing
	}
	\While{$x == 5$}{
		still do nothing
	}
	\ForEach{$x = 1:5$}{
		do nothing
	}
	\Return{return nothing}
\end{algorithm}


\end{document}
