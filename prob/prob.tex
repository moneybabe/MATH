\documentclass{article}
\usepackage{amsfonts, amsmath, amssymb, amsthm} % Math notations imported
\usepackage{enumitem}
\usepackage[margin=1in]{geometry}
\usepackage{blkarray}

\newtheorem{thm}{Theorem}
\newtheorem{prop}[thm]{Proposition}
\newtheorem{cor}[thm]{Corollary}

% title information
\title{Probability Distribution Properties}

% main content
\begin{document} 
\date{}

% placing title information; comment out if using fancyhdr
\maketitle 
\section{Discrete Distributions}
\subsection{Uniform Random Variable}
\begin{align*}
    & E[X] = \frac{x_1 + \dots + x_n}{n}; \quad Var(X) = \frac{x_1^2 + \dots + x_n^2}{n} - \left(\frac{x_1 + \dots + x_n}{n}\right)^2. \\
    & E[X] = \frac{n+1}{2}; \quad Var(X) = \frac{n^2-1}{12}; \quad \emph{only when $x_i \in [1, 2, \dots, n]$}.
\end{align*}

\subsection{Bernoulli Random Variable}
\begin{align*}
    & I = \begin{cases}
        1 & \emph{if $X = 1$}, \\
        0 & \emph{if $X = 0$}.
    \end{cases} \\
    & E[I] = p; \quad Var(I) = p(1-p).
\end{align*}

\subsection{Binomial Random Variable}
\begin{align*}
    & P(X=k) = {n\choose k}p^k(1-p)^{n-k} \quad \emph{for } k \in [0,n]. \\
    & E[X] = np; \quad Var(X) = np(1-p).
\end{align*}

\subsection{Poisson Random Variable}
\begin{align*}
    & P(X=k) = \frac{\lambda^k e^{-\lambda}}{k!} \quad \emph{for } k \in \mathbb{Z}^{^\ge}. \\
    & E[X] = \lambda; \quad Var(X) = \lambda.
\end{align*}

\subsection{Geometric Random Variable}
\begin{align*}
    & P(X=n) = (1-p)^{n-1}p \quad \emph{for } n \in \mathbb{N}. \\
    & E[X] = \frac{1}{p}; \quad Var(X) = \frac{1-p}{p^2}; \quad P(X > n) = (1-p)^n ; \quad P(X>n+k|X>k) = P(X>n).
\end{align*}

\newpage
\section{Continuous Distributions}
\subsection{Uniform Random Variable}
\begin{align*}
    & f(x) = \begin{cases}
        \frac{1}{\beta-\alpha} & \emph{if $x \in [\alpha, \beta]$}, \\
        0 & \emph{otherwise}.
    \end{cases} \\
    & E[X] = \frac{\alpha+\beta}{2}; \quad Var(X) = \frac{(\beta-\alpha)^2}{12}.
\end{align*}

\subsection{Exponential Random Variable}
\begin{align*}
    & f(x) = \begin{cases}
        \lambda e^{-\lambda x} & \emph{if $x \ge 0$}, \\
        0 & \emph{otherwise}.
    \end{cases} \\
    & E[X] = \frac{1}{\lambda}; \quad Var(X) = \frac{1}{\lambda^2}; \quad P(X\ge x) = e^{-\lambda x}; \quad P(X\ge x + y|X\ge y) = e^{-\lambda x}.
\end{align*}

\subsection{Normal Random Variable}
\begin{align*}
    & f(x) = \frac{1}{\sigma\sqrt{2\pi}} e^{-\frac{(x-\mu)^2}{2\sigma^2}} \quad \emph{for $x\in \mathbb{R}$}. \\
    & E[X] = \mu; \quad Var(X) = \sigma^2; \quad Z \sim \emph{N(0,1)} = \frac{X-\mu}{\sigma}.
\end{align*}

\subsection{Gamma Random Variable}
\begin{align*}
    & f(x) = \frac{\lambda}{\Gamma(\alpha)} (\lambda x)^{\alpha-1} e^{-\lambda x} \quad \emph{for $\alpha, \beta, x\in \mathbb{R}^{^+}$}; \quad \Gamma(z) = \int_0^\infty t^{z-1}e^{-t}dt. \\
    & E[X] = \frac{\alpha}{\lambda}; \quad Var(X) = \frac{\alpha}{\lambda^2}.
\end{align*}

\newpage
\section{Multivariate Distributions}
\subsection{Multivariate Normal Distribution}
\begin{align*}
    & f(\vec{x}) = \frac{1}{(2\pi)^{n/2}|\Sigma|^{1/2}} e^{-\frac{1}{2}(\vec{x}-\vec{\mu})^{^\top}\Sigma^{-1}(\vec{x}-\vec{\mu})} \quad \emph{for $x\in \mathbb{R}^n$}. \\
    & E[X] = \vec{\mu}; \quad Var(X) = \Sigma.
\end{align*}
Let random vector $\begin{bmatrix} X_1 \\ \vdots \\ X_n \end{bmatrix}$ have a multivariate normal distribution and $Z$ be the linear combination of the random vector with coefficients $\vec{A} = \begin{bmatrix}
    a_1 \\ 
    \vdots \\
    a_n
\end{bmatrix}$.
Then, $Z\sim \emph{N($\vec{A}\cdot \vec{\mu},\vec{A}^{^\top}\sum\vec{A}$)}$.
For bivariate random vector $\begin{bmatrix}
    X \\ Y
\end{bmatrix}$ with $\vec{A} = \begin{bmatrix}
    a \\ b
\end{bmatrix}$,
\begin{align*}
    Var(Z) = a^2\sigma_X^2+2ab\rho\sigma_X\sigma_Y+b^2\sigma_Y^2.
\end{align*}

\subsection{Multinomial Distribution}
\begin{align*}
    & P(\vec{x}) = \frac{n!}{x_1!\cdots x_k!}p_1^{x_1}\cdots p_k^{x_k} \quad \emph{for $\vec{x} = \begin{bmatrix}
        x_1 \\ \vdots \\ x_k
    \end{bmatrix}$, $x_i\in \mathbb{N}$, $\sum_{i=1}^k x_i = n$, $p_i\in [0,1]$}. \\
    & E[\vec{x}] = n\vec{p}; \quad Var(\vec{x}) = n\overrightarrow{p(1-p)}.
\end{align*}

\subsection{Random Sums}
\begin{align*}
    & X = \xi_1 + \dots + \xi_N; \quad \emph{where $N$ is a R.V. and $\xi_i$ are i.i.d. R.Vs}. \\
    & E[X] = E[\xi]E[N]; \quad Var(X) = E[N]Var(\xi) + E[\xi]^2Var(N).
\end{align*}

\newpage
\section{Theorems}
\subsection{Chebyshev's Inequality}
\begin{align*}
    & P(|X-E[X]| \ge k) \le \frac{Var(X)}{k^2}.
\end{align*}

\subsection{Markov Inequality}
For a non-negative R.V. $X$,
\begin{align*}
    & P(X \ge k) \le \frac{E[X]}{k}.
\end{align*}

\subsection{Central Limit Theorem}
Let $S_n = X_1 + \dots + X_n$. When $n$ is large, $S_n$ is approximately normally distributed with mean $n\mu $ and variance $n\sigma^2$.
\begin{align*}
    & \frac{S_n-n\mu}{\sqrt{n}\sigma} = \frac{\bar{S_n} - \mu}{\sigma / \sqrt{n}} \sim \emph{N(0,1)} \quad \emph{as $n\to\infty$}.
\end{align*}

\subsection{Weak Law of Large Numbers}
Let $S_n = X_1 + \dots + X_n$. 
\begin{align*}
    & \lim_{n\rightarrow\infty}P(|\bar{S_n}-\mu|<\epsilon) = 1.
\end{align*}

\subsection{Strong Law of Large Numbers}
Let $S_n = X_1 + \dots + X_n$.
\begin{align*}
    & P(\lim_{n\rightarrow\infty}\bar{S_n}=\mu) = 1.
\end{align*}

\subsection{Moment Generating Function}
\begin{align*}
    & M_X(t) = E[e^{tX}] = \int_{-\infty}^\infty e^{tx}f(x)dx.
\end{align*}

\newpage
\section{Special Markov Chains}
\subsection{Two State Markov Chain}
\begin{align*}
    P & = 
        \begin{blockarray}{ccc}
            & 0 & 1 \\
            \begin{block}{c||cc||}
                0 & 1-a & a \\
                1 & b & 1-b \\
            \end{block}
        \end{blockarray}\; ; \emph{ state 0, 1 are independent R.V. iff $a=1-b$.}\\
        P^n & = \frac{1}{a+b}
        \begin{bmatrix}
            b & a \\
            b & a \\
        \end{bmatrix} + 
        \frac{(1-a-b)^n}{a+b}
        \begin{bmatrix}
            a & -a \\
            -b & b \\
        \end{bmatrix} ; \quad
        \lim_{n\rightarrow\infty}P^n = 
            \begin{blockarray}{ccc}
                & 0 & 1 \\
                \begin{block}{c||cc||}
                    0 & \frac{b}{a+b} & \frac{a}{a+b} \\
                    1 & \frac{b}{a+b} & \frac{a}{a+b} \\
                \end{block}
            \end{blockarray}\; .
\end{align*} 

\subsection{i.i.d. Markov Chain}
\begin{align*}
    \xi_1, \dots, \xi_n \emph{ are i.i.d. R.Vs.}; \quad P(\xi = k) = a_k \emph{ for } k\in\mathbb{Z^{^\ge}}.
\end{align*}

\subsubsection{Independent Random Variables}
\begin{align*}
    X_n = \xi; P & = 
    \begin{blockarray}{cccc}
        \begin{block}{||cccc||}
            a_0 & a_1 & a_2 & \cdots \\
            a_0 & a_1 & a_2 & \cdots \\
            a_0 & a_1 & a_2 & \cdots \\
            a_0 & a_1 & a_2 & \cdots \\
            \vdots & \vdots & \vdots &  \ddots \\
        \end{block} 
    \end{blockarray}\; .
\end{align*}

\subsubsection{Successive Maxima}
\begin{align*}
    X_n & = max(\xi_1, \dots, \xi_n) \emph{ for } n\in \mathbb{N}.
\end{align*}
\begin{align*}
    P & = 
        \begin{blockarray}{ccccc}
            \begin{block}{||ccccc||}
                A_0 & a_1 & a_2 & a_3 & \cdots \\
                0 & A_1 & a_2 & a_3 & \cdots \\
                0 & 0 & A_2 & a_3 & \cdots \\
                0 & 0 & 0 & A_3 & \cdots \\
                \vdots & \vdots & \vdots & \vdots & \ddots \\
            \end{block} 
        \end{blockarray} \emph{ where }A_k = P(X_{n+1} \le x | X_n = x) = P(\xi_{n+1} \le x).
\end{align*}
Let $T$ be the number of trials needed for $X_n = max(\xi_1, \dots, \xi_n)\ge M$ for an arbitrary value $M$,
\begin{align*}
    E[T] & = \frac{1}{P(\xi \ge M)}.
\end{align*}
\emph{Successive Maxima} can be interpreted as a \emph{Geometric(p)} random variable with $p = P(\xi \ge M)$.

\newpage
\subsubsection{Partial Sums}
\begin{align*}
    X_n & = \xi_1 + \dots + \xi_n \emph{ for } n\in \mathbb{N}.
\end{align*}
\begin{align*}
    P = 
    \begin{blockarray}{ccccc}
        \begin{block}{||ccccc||}
            a_0 & a_1 & a_2 & a_3 & \cdots \\
            0 & a_0 & a_1 & a_2 & \cdots \\
            0 & 0 & a_0 & a_1 & \cdots \\
            0 & 0 & 0 & a_0 & \cdots \\
            \vdots & \vdots & \vdots & \vdots & \ddots \\
        \end{block} 
    \end{blockarray}\; \emph{ if } P(\xi = k) = a_k \emph{ for }k \in \mathbb{Z^{^\ge}}. \\
    P = 
    \begin{blockarray}{cccccc}
        \begin{block}{||cccccc||}
            & \vdots & \vdots & \vdots & \vdots & \\
            \cdots & a_{-1} & a_0 & a_1 & a_2 & \cdots \\
            \cdots & a_{-2} & a_{-1} & a_0 & a_1 & \cdots \\
            \cdots & a_{-3} & a_{-2} & a_{-1} & a_0 & \cdots \\
            & \vdots & \vdots & \vdots & \vdots & \\
        \end{block} 
    \end{blockarray} \emph{ if } P(\xi = k) = a_k \emph{ for }k \in \mathbb{Z}.
\end{align*}

\subsubsection{Branching Processes}
\begin{align*}
    X_{n+1} & = \xi_1^{(n)} + \dots + \xi_{X_n}^{(n)} \emph{ for } n\in \mathbb{N}; \quad X_0 = 1. \\
    u_n & = \emph{probability of extinction by n-th generation} \\
    & = \sum_{k=0}^{\infty}P(\xi = k)(u_{n-1})^k \emph{ for } n\in \mathbb{N}; \\
    u_0 & = 0.
\end{align*}
From random sums, we know
\begin{align*}
    E[X_{n+1}] = E[\xi]E[X_{n}]; \quad Var(X_{n+1}) = Var(\xi)E[X_n] + E[\xi]^2Var(X_n).
\end{align*}
After derivation, 
\begin{align*}
    E[X_n] = E[\xi]^n; \quad Var(X_n) = Var(\xi)E[\xi]^{n-1}\times \begin{cases}
        n & \emph{if } E[\xi] = 1, \\
        \frac{1-E[\xi]^n}{1-E[\xi]} & \emph{otherwise}.
    \end{cases}
\end{align*}
    
\subsection{One-dimensional Random Walk}
\begin{align*}
    P = 
    \begin{blockarray}{cccccc}
        & 0 & 1 & 2 & 3 & \cdots \\
        \begin{block}{c||ccccc||}
            0 & r_0 & p_0 & 0 & 0 & \cdots \\
            1 & q_1 & r_1 & p_1 & 0 & \cdots \\
            2 & 0 & q_2 & r_2 & p_2 & \cdots \\
            3 & 0 & 0 & q_3 & r_3 & \cdots \\
            \vdots & \vdots & \vdots & \vdots & \vdots & \ddots \\
        \end{block} 
    \end{blockarray}\; .
\end{align*}

\subsubsection{Simple Random Walk}
\begin{align*}
    r_i = r = 0; \quad q_i = q = \frac{1}{2}; \quad p_i = p = \frac{1}{2}.
\end{align*}
\subsubsection{Gambler's Ruin}
\begin{align*}
    P = 
    \begin{blockarray}{cccccccc}
        & 0 & 1 & 2 & 3 & \cdots & N-1 & N \\
            \begin{block}{c||ccccccc||}
                0 & 1 & 0 & 0 & 0 & \cdots & 0 & 0 \\
                1 & q & 0 & p & 0 & \cdots & 0 & 0 \\
                2 & 0 & q & 0 & p & \cdots & 0 & 0 \\
                \vdots & \vdots & \vdots & \vdots & \vdots & \ddots & \vdots & \vdots\\
                N-1 & 0 & 0 & 0 & 0 & \cdots & 0 & p \\
                N & 0 & 0 & 0 & 0 & \cdots & 0 & 1 \\
            \end{block} 
        \end{blockarray}\; .
\end{align*}
Let $u_i$ be the probability of $X_n$ reaching state 0 before state $N$ given current state $i$.
\begin{align*}
    u_i & = \begin{cases}
        \frac{N-i}{N} & \emph{if } p = q = \frac{1}{2}, \\
        \frac{(q/p)^i-(q/p)^N}{1-(q/p)^N} & \emph{if } p \neq q.
    \end{cases} \\
    \lim_{N\rightarrow\infty} u_i & = \begin{cases}
        1 & \emph{if } p \le q, \\
        \left(\frac{q}{p}\right)^i & \emph{if } p > q.
    \end{cases}
\end{align*}
Let $T$ be the time needed to reach state 0.
\begin{align*}
    E[T] & = i(N-i) \emph{ if } p = q = \frac{1}{2}.
\end{align*}

\subsection{Success Runs}
\begin{align*}
    P = 
        \begin{blockarray}{ccccccc}
            & 0 & 1 & 2 & 3 & 4 & \cdots \\
            \begin{block}{c||cccccc||}
                0 & p_0 & q_0 & 0 & 0 & 0 & \cdots \\
                1 & p_1 & r_1 & q_1 & 0 & 0 & \cdots \\
                2 & p_2 & 0 & r_2 & q_2 & 0 & \cdots \\
                3 & p_3 & 0 & 0 & r_3 & q_3 & \cdots \\
                \vdots & \vdots & \vdots & \vdots & \vdots & \vdots & \ddots \\
            \end{block} 
        \end{blockarray}\; .
\end{align*}

\subsubsection{Two Outcomes Repeated Trials}
\begin{align*}
    r_i = r = 0; \quad q_i = q = \frac{1}{2}; \quad p_i = p = \frac{1}{2}.
\end{align*}

\subsubsection{Renewal Process} 
Model a light bulb's lifetime as a random variable $\xi$, where
\begin{align*}
    P(\xi=k) = a_k \emph{ for } k \in\mathbb{N},
\end{align*}
and $X_n$ as the age of the bulb in service at time $n$.
\begin{align*}
    r_k = 0; \quad p_k = \frac{a_{k+1}}{a_{k+1} + a_{k+2} + \dots}; \quad q_k = 1-p_k \emph{ for } k \in\mathbb{Z^{^\ge}}.
\end{align*}

\newpage
\section{Estimators}
\subsection{Unbiasedness}
\begin{align*}
    Bias(\hat{\theta}) = E[\hat{\theta}] - \theta.
\end{align*}

\subsection{Efficiency}
Let $\hat{\theta}_1$ and $\hat{\theta}_2$ be two unbiased estimators of $\theta$. 
$\hat{\theta}_1$ is more efficient than $\hat{\theta}_2$ if
\begin{align*}
    Var(\hat{\theta}_1) < Var(\hat{\theta}_2).
\end{align*}
The relative efficiency of $\hat{\theta}_1$ to $\hat{\theta}_2$ is
\begin{align*}
    \frac{Var(\hat{\theta}_2)}{Var(\hat{\theta}_1)}.
\end{align*}

\subsection{Consistency}
$\hat{\theta}$ is a consistent estimator of $\theta$ if
\begin{align*}
    \lim_{n\rightarrow\infty}P(|\hat{\theta}-\theta|\ge\epsilon) = 0.
\end{align*}

\subsection{Sufficiency}
An estimator $\hat{\theta}$ is sufficient for $\theta$ if the conditional distribution of $X_1,\dots,X_n$ given $\hat{\theta}$ does not depend on $\theta$;
equivalently $\hat{\theta}$ is sufficient for $\theta$ if the likelihood function $L(\theta) = f(X;\theta)$ can be factorized as
\begin{align*}
    L(\theta) = g(\hat{\theta}; \theta)b(X_1, \dots, X_n).
\end{align*}

\subsection{Mean Squared Error (MSE)}
\begin{align*}
    MSE(\hat{\theta}) = E[(\hat{\theta}-\theta)^2] = Var(\hat{\theta}) + Bias(\hat{\theta})^2.
\end{align*}

\subsection{Minimum Variance Unbiased Estimator}
\subsubsection{Fishers Information}
\begin{align*}
    I(\theta) = E\left[\left(\frac{\partial}{\partial\theta}\log f(X;\theta)\right)^2\right] = -E\left[\frac{\partial^2}{\partial\theta^2}\log f(X;\theta)\right].
\end{align*}

\subsubsection{Cramer-Rao Lower Bound}
Let $\hat{\theta}$ be an unbiased estimator of $\theta$, then
\begin{align*}
    Var(\hat{\theta}) \ge \frac{1}{I(\theta)}.
\end{align*}
If $Var(\hat{\theta}) = \frac{1}{I(\theta)}$, then $\hat{\theta}$ is a minimum variance unbiased estimator, aka asymptotically efficient.

\subsection{Asymptotic Properties of Maximum Likelihood Estimators}
\subsubsection{Asymptotic Normality}
\begin{align*}
    \lim_{n\rightarrow\infty}\hat{\theta} \sim \emph{N($\theta$, $\frac{1}{nI(\theta)}$)}.
\end{align*}
\subsubsection{Asymptotic Unbiasedness}
\begin{align*}
    \lim_{n\rightarrow\infty}E[\hat{\theta}] = \theta.
\end{align*}
\subsubsection{Asymptotic Efficiency}
\begin{align*}
    \lim_{n\rightarrow\infty}Var(\hat{\theta}) = \frac{1}{nI(\theta)}.
\end{align*}


\end{document}